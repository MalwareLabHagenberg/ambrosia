% Generated by Sphinx.
\def\sphinxdocclass{report}
\documentclass[letterpaper,10pt,english]{sphinxmanual}
\usepackage[utf8]{inputenc}
\DeclareUnicodeCharacter{00A0}{\nobreakspace}
\usepackage{cmap}
\usepackage[T1]{fontenc}
\usepackage{babel}
\usepackage{times}
\usepackage[Bjarne]{fncychap}
\usepackage{longtable}
\usepackage{sphinx}
\usepackage{multirow}


\title{Ambrosia Documentation}
\date{April 18, 2015}
\release{0.9.0}
\author{Wolfgang Ettlinger}
\newcommand{\sphinxlogo}{}
\renewcommand{\releasename}{Release}
\makeindex

\makeatletter
\def\PYG@reset{\let\PYG@it=\relax \let\PYG@bf=\relax%
    \let\PYG@ul=\relax \let\PYG@tc=\relax%
    \let\PYG@bc=\relax \let\PYG@ff=\relax}
\def\PYG@tok#1{\csname PYG@tok@#1\endcsname}
\def\PYG@toks#1+{\ifx\relax#1\empty\else%
    \PYG@tok{#1}\expandafter\PYG@toks\fi}
\def\PYG@do#1{\PYG@bc{\PYG@tc{\PYG@ul{%
    \PYG@it{\PYG@bf{\PYG@ff{#1}}}}}}}
\def\PYG#1#2{\PYG@reset\PYG@toks#1+\relax+\PYG@do{#2}}

\expandafter\def\csname PYG@tok@gd\endcsname{\def\PYG@tc##1{\textcolor[rgb]{0.63,0.00,0.00}{##1}}}
\expandafter\def\csname PYG@tok@gu\endcsname{\let\PYG@bf=\textbf\def\PYG@tc##1{\textcolor[rgb]{0.50,0.00,0.50}{##1}}}
\expandafter\def\csname PYG@tok@gt\endcsname{\def\PYG@tc##1{\textcolor[rgb]{0.00,0.27,0.87}{##1}}}
\expandafter\def\csname PYG@tok@gs\endcsname{\let\PYG@bf=\textbf}
\expandafter\def\csname PYG@tok@gr\endcsname{\def\PYG@tc##1{\textcolor[rgb]{1.00,0.00,0.00}{##1}}}
\expandafter\def\csname PYG@tok@cm\endcsname{\let\PYG@it=\textit\def\PYG@tc##1{\textcolor[rgb]{0.25,0.50,0.56}{##1}}}
\expandafter\def\csname PYG@tok@vg\endcsname{\def\PYG@tc##1{\textcolor[rgb]{0.73,0.38,0.84}{##1}}}
\expandafter\def\csname PYG@tok@m\endcsname{\def\PYG@tc##1{\textcolor[rgb]{0.13,0.50,0.31}{##1}}}
\expandafter\def\csname PYG@tok@mh\endcsname{\def\PYG@tc##1{\textcolor[rgb]{0.13,0.50,0.31}{##1}}}
\expandafter\def\csname PYG@tok@cs\endcsname{\def\PYG@tc##1{\textcolor[rgb]{0.25,0.50,0.56}{##1}}\def\PYG@bc##1{\setlength{\fboxsep}{0pt}\colorbox[rgb]{1.00,0.94,0.94}{\strut ##1}}}
\expandafter\def\csname PYG@tok@ge\endcsname{\let\PYG@it=\textit}
\expandafter\def\csname PYG@tok@vc\endcsname{\def\PYG@tc##1{\textcolor[rgb]{0.73,0.38,0.84}{##1}}}
\expandafter\def\csname PYG@tok@il\endcsname{\def\PYG@tc##1{\textcolor[rgb]{0.13,0.50,0.31}{##1}}}
\expandafter\def\csname PYG@tok@go\endcsname{\def\PYG@tc##1{\textcolor[rgb]{0.20,0.20,0.20}{##1}}}
\expandafter\def\csname PYG@tok@cp\endcsname{\def\PYG@tc##1{\textcolor[rgb]{0.00,0.44,0.13}{##1}}}
\expandafter\def\csname PYG@tok@gi\endcsname{\def\PYG@tc##1{\textcolor[rgb]{0.00,0.63,0.00}{##1}}}
\expandafter\def\csname PYG@tok@gh\endcsname{\let\PYG@bf=\textbf\def\PYG@tc##1{\textcolor[rgb]{0.00,0.00,0.50}{##1}}}
\expandafter\def\csname PYG@tok@ni\endcsname{\let\PYG@bf=\textbf\def\PYG@tc##1{\textcolor[rgb]{0.84,0.33,0.22}{##1}}}
\expandafter\def\csname PYG@tok@nl\endcsname{\let\PYG@bf=\textbf\def\PYG@tc##1{\textcolor[rgb]{0.00,0.13,0.44}{##1}}}
\expandafter\def\csname PYG@tok@nn\endcsname{\let\PYG@bf=\textbf\def\PYG@tc##1{\textcolor[rgb]{0.05,0.52,0.71}{##1}}}
\expandafter\def\csname PYG@tok@no\endcsname{\def\PYG@tc##1{\textcolor[rgb]{0.38,0.68,0.84}{##1}}}
\expandafter\def\csname PYG@tok@na\endcsname{\def\PYG@tc##1{\textcolor[rgb]{0.25,0.44,0.63}{##1}}}
\expandafter\def\csname PYG@tok@nb\endcsname{\def\PYG@tc##1{\textcolor[rgb]{0.00,0.44,0.13}{##1}}}
\expandafter\def\csname PYG@tok@nc\endcsname{\let\PYG@bf=\textbf\def\PYG@tc##1{\textcolor[rgb]{0.05,0.52,0.71}{##1}}}
\expandafter\def\csname PYG@tok@nd\endcsname{\let\PYG@bf=\textbf\def\PYG@tc##1{\textcolor[rgb]{0.33,0.33,0.33}{##1}}}
\expandafter\def\csname PYG@tok@ne\endcsname{\def\PYG@tc##1{\textcolor[rgb]{0.00,0.44,0.13}{##1}}}
\expandafter\def\csname PYG@tok@nf\endcsname{\def\PYG@tc##1{\textcolor[rgb]{0.02,0.16,0.49}{##1}}}
\expandafter\def\csname PYG@tok@si\endcsname{\let\PYG@it=\textit\def\PYG@tc##1{\textcolor[rgb]{0.44,0.63,0.82}{##1}}}
\expandafter\def\csname PYG@tok@s2\endcsname{\def\PYG@tc##1{\textcolor[rgb]{0.25,0.44,0.63}{##1}}}
\expandafter\def\csname PYG@tok@vi\endcsname{\def\PYG@tc##1{\textcolor[rgb]{0.73,0.38,0.84}{##1}}}
\expandafter\def\csname PYG@tok@nt\endcsname{\let\PYG@bf=\textbf\def\PYG@tc##1{\textcolor[rgb]{0.02,0.16,0.45}{##1}}}
\expandafter\def\csname PYG@tok@nv\endcsname{\def\PYG@tc##1{\textcolor[rgb]{0.73,0.38,0.84}{##1}}}
\expandafter\def\csname PYG@tok@s1\endcsname{\def\PYG@tc##1{\textcolor[rgb]{0.25,0.44,0.63}{##1}}}
\expandafter\def\csname PYG@tok@gp\endcsname{\let\PYG@bf=\textbf\def\PYG@tc##1{\textcolor[rgb]{0.78,0.36,0.04}{##1}}}
\expandafter\def\csname PYG@tok@sh\endcsname{\def\PYG@tc##1{\textcolor[rgb]{0.25,0.44,0.63}{##1}}}
\expandafter\def\csname PYG@tok@ow\endcsname{\let\PYG@bf=\textbf\def\PYG@tc##1{\textcolor[rgb]{0.00,0.44,0.13}{##1}}}
\expandafter\def\csname PYG@tok@sx\endcsname{\def\PYG@tc##1{\textcolor[rgb]{0.78,0.36,0.04}{##1}}}
\expandafter\def\csname PYG@tok@bp\endcsname{\def\PYG@tc##1{\textcolor[rgb]{0.00,0.44,0.13}{##1}}}
\expandafter\def\csname PYG@tok@c1\endcsname{\let\PYG@it=\textit\def\PYG@tc##1{\textcolor[rgb]{0.25,0.50,0.56}{##1}}}
\expandafter\def\csname PYG@tok@kc\endcsname{\let\PYG@bf=\textbf\def\PYG@tc##1{\textcolor[rgb]{0.00,0.44,0.13}{##1}}}
\expandafter\def\csname PYG@tok@c\endcsname{\let\PYG@it=\textit\def\PYG@tc##1{\textcolor[rgb]{0.25,0.50,0.56}{##1}}}
\expandafter\def\csname PYG@tok@mf\endcsname{\def\PYG@tc##1{\textcolor[rgb]{0.13,0.50,0.31}{##1}}}
\expandafter\def\csname PYG@tok@err\endcsname{\def\PYG@bc##1{\setlength{\fboxsep}{0pt}\fcolorbox[rgb]{1.00,0.00,0.00}{1,1,1}{\strut ##1}}}
\expandafter\def\csname PYG@tok@mb\endcsname{\def\PYG@tc##1{\textcolor[rgb]{0.13,0.50,0.31}{##1}}}
\expandafter\def\csname PYG@tok@ss\endcsname{\def\PYG@tc##1{\textcolor[rgb]{0.32,0.47,0.09}{##1}}}
\expandafter\def\csname PYG@tok@sr\endcsname{\def\PYG@tc##1{\textcolor[rgb]{0.14,0.33,0.53}{##1}}}
\expandafter\def\csname PYG@tok@mo\endcsname{\def\PYG@tc##1{\textcolor[rgb]{0.13,0.50,0.31}{##1}}}
\expandafter\def\csname PYG@tok@kd\endcsname{\let\PYG@bf=\textbf\def\PYG@tc##1{\textcolor[rgb]{0.00,0.44,0.13}{##1}}}
\expandafter\def\csname PYG@tok@mi\endcsname{\def\PYG@tc##1{\textcolor[rgb]{0.13,0.50,0.31}{##1}}}
\expandafter\def\csname PYG@tok@kn\endcsname{\let\PYG@bf=\textbf\def\PYG@tc##1{\textcolor[rgb]{0.00,0.44,0.13}{##1}}}
\expandafter\def\csname PYG@tok@o\endcsname{\def\PYG@tc##1{\textcolor[rgb]{0.40,0.40,0.40}{##1}}}
\expandafter\def\csname PYG@tok@kr\endcsname{\let\PYG@bf=\textbf\def\PYG@tc##1{\textcolor[rgb]{0.00,0.44,0.13}{##1}}}
\expandafter\def\csname PYG@tok@s\endcsname{\def\PYG@tc##1{\textcolor[rgb]{0.25,0.44,0.63}{##1}}}
\expandafter\def\csname PYG@tok@kp\endcsname{\def\PYG@tc##1{\textcolor[rgb]{0.00,0.44,0.13}{##1}}}
\expandafter\def\csname PYG@tok@w\endcsname{\def\PYG@tc##1{\textcolor[rgb]{0.73,0.73,0.73}{##1}}}
\expandafter\def\csname PYG@tok@kt\endcsname{\def\PYG@tc##1{\textcolor[rgb]{0.56,0.13,0.00}{##1}}}
\expandafter\def\csname PYG@tok@sc\endcsname{\def\PYG@tc##1{\textcolor[rgb]{0.25,0.44,0.63}{##1}}}
\expandafter\def\csname PYG@tok@sb\endcsname{\def\PYG@tc##1{\textcolor[rgb]{0.25,0.44,0.63}{##1}}}
\expandafter\def\csname PYG@tok@k\endcsname{\let\PYG@bf=\textbf\def\PYG@tc##1{\textcolor[rgb]{0.00,0.44,0.13}{##1}}}
\expandafter\def\csname PYG@tok@se\endcsname{\let\PYG@bf=\textbf\def\PYG@tc##1{\textcolor[rgb]{0.25,0.44,0.63}{##1}}}
\expandafter\def\csname PYG@tok@sd\endcsname{\let\PYG@it=\textit\def\PYG@tc##1{\textcolor[rgb]{0.25,0.44,0.63}{##1}}}

\def\PYGZbs{\char`\\}
\def\PYGZus{\char`\_}
\def\PYGZob{\char`\{}
\def\PYGZcb{\char`\}}
\def\PYGZca{\char`\^}
\def\PYGZam{\char`\&}
\def\PYGZlt{\char`\<}
\def\PYGZgt{\char`\>}
\def\PYGZsh{\char`\#}
\def\PYGZpc{\char`\%}
\def\PYGZdl{\char`\$}
\def\PYGZhy{\char`\-}
\def\PYGZsq{\char`\'}
\def\PYGZdq{\char`\"}
\def\PYGZti{\char`\~}
% for compatibility with earlier versions
\def\PYGZat{@}
\def\PYGZlb{[}
\def\PYGZrb{]}
\makeatother

\renewcommand\PYGZsq{\textquotesingle}

\begin{document}

\maketitle
\tableofcontents
\phantomsection\label{index::doc}


Ambrosia is a framework that takes the information from ANANAS reports, applies several operations (time adjustments, correlations, etc) and presents this data in a viewable form.


\chapter{Contents}
\label{index:ambrosia-documentation}\label{index:contents}

\section{Ambrosia Client Documentation}
\label{client::doc}\label{client:ambrosia-client-documentation}

\subsection{Built-In Namespace \_global\_}
\label{_global_::doc}\label{_global_:built-in-namespace-global}

\subsubsection{Methods}
\label{_global_:methods}

\paragraph{Class}
\label{_global_:class}\index{Class() (built-in function)}

\begin{fulllineitems}
\phantomsection\label{_global_:Class}\pysiglinewithargsret{\bfcode{Class}}{\emph{name}, \emph{p1}\optional{, \emph{p2}}}{}~\begin{quote}\begin{description}
\item[{Arguments}] \leavevmode\begin{itemize}
\item {} 
\textbf{name} (\emph{String}) -- the fully qualified name of the new class

\item {} 
\textbf{p1} (\emph{String\textbar{}Object}) -- if two parameters are passed: the object containing class members, else the superclass

\item {} 
\textbf{p2} (\emph{Object}) -- the obj object containing class members

\end{itemize}

\item[{Returns class}] \leavevmode
the newly created class

\end{description}\end{quote}

\end{fulllineitems}


creates a class


\subsection{Namespace ambrosia\_web}
\label{ambrosia_web::doc}\label{ambrosia_web:namespace-ambrosia-web}

\subsubsection{Constructor}
\label{ambrosia_web:constructor}\index{ambrosia\_web() (class)}

\begin{fulllineitems}
\phantomsection\label{ambrosia_web:ambrosia_web}\pysiglinewithargsret{\strong{class }\bfcode{ambrosia\_web}}{}{}
\end{fulllineitems}



\subsubsection{Methods}
\label{ambrosia_web:methods}

\paragraph{init}
\label{ambrosia_web:init}\index{ambrosia\_web.init() (ambrosia\_web method)}

\begin{fulllineitems}
\phantomsection\label{ambrosia_web:ambrosia_web.init}\pysiglinewithargsret{\code{ambrosia\_web.}\bfcode{init}}{}{}
\end{fulllineitems}


initialize Ambrosia


\paragraph{redraw}
\label{ambrosia_web:redraw}\index{ambrosia\_web.redraw() (ambrosia\_web method)}

\begin{fulllineitems}
\phantomsection\label{ambrosia_web:ambrosia_web.redraw}\pysiglinewithargsret{\code{ambrosia\_web.}\bfcode{redraw}}{}{}
\end{fulllineitems}


Redraws all views of the application


\subsection{Namespace ambrosia\_web.entity}
\label{ambrosia_web.entity::doc}\label{ambrosia_web.entity:namespace-ambrosia-web-entity}

\subsubsection{Constructor}
\label{ambrosia_web.entity:constructor}\index{ambrosia\_web.entity() (class)}

\begin{fulllineitems}
\phantomsection\label{ambrosia_web.entity:ambrosia_web.entity}\pysiglinewithargsret{\strong{class }\code{ambrosia\_web.}\bfcode{entity}}{}{}
\end{fulllineitems}



\subsubsection{Methods}
\label{ambrosia_web.entity:methods}

\paragraph{enrich}
\label{ambrosia_web.entity:enrich}\index{ambrosia\_web.entity.enrich() (ambrosia\_web.entity method)}

\begin{fulllineitems}
\phantomsection\label{ambrosia_web.entity:ambrosia_web.entity.enrich}\pysiglinewithargsret{\code{ambrosia\_web.entity.}\bfcode{enrich}}{\emph{el}}{}~\begin{quote}\begin{description}
\item[{Arguments}] \leavevmode\begin{itemize}
\item {} 
\textbf{el} (\emph{object}) -- the deserialized data

\end{itemize}

\end{description}\end{quote}

\end{fulllineitems}


Receives an object containing the deserialized data from the server and returns an instance of the class
{\hyperref[ambrosia_web.entity.Entity:ambrosia_web.entity.Entity]{\code{ambrosia\_web.entity.Entity()}}}


\subsubsection{Attributes}
\label{ambrosia_web.entity:attributes}

\paragraph{onSelectHandler}
\label{ambrosia_web.entity:onselecthandler}\index{onSelectHandler (None attribute)}

\begin{fulllineitems}
\phantomsection\label{ambrosia_web.entity:onSelectHandler}\pysigline{\bfcode{onSelectHandler}}
\end{fulllineitems}


contains all handlers for selecting entities. Any part of the application may listen to those events (i.e. add a
function to this array). If the user select an entity the interface can adapt to this (e.g. the
{\hyperref[ambrosia_web.view.entityview.EntityView:ambrosia_web.view.entityview.EntityView]{\code{ambrosia\_web.view.entityview.EntityView()}}} shows details about this entity).


\subsection{Class ambrosia\_web.entity.Entity}
\label{ambrosia_web.entity.Entity::doc}\label{ambrosia_web.entity.Entity:class-ambrosia-web-entity-entity}
The client side counterpart for an entity


\strong{See also:}


{\hyperref[ambrosia.model:ambrosia.model.Entity]{\code{ambrosia.model.Entity}}}




\subsubsection{Constructor}
\label{ambrosia_web.entity.Entity:constructor}\index{ambrosia\_web.entity.Entity() (class)}

\begin{fulllineitems}
\phantomsection\label{ambrosia_web.entity.Entity:ambrosia_web.entity.Entity}\pysiglinewithargsret{\strong{class }\code{ambrosia\_web.entity.}\bfcode{Entity}}{}{}
\end{fulllineitems}



\subsubsection{Methods}
\label{ambrosia_web.entity.Entity:methods}

\paragraph{getLink}
\label{ambrosia_web.entity.Entity:getlink}\index{ambrosia\_web.entity.Entity.getLink() (ambrosia\_web.entity.Entity method)}

\begin{fulllineitems}
\phantomsection\label{ambrosia_web.entity.Entity:ambrosia_web.entity.Entity.getLink}\pysiglinewithargsret{\code{ambrosia\_web.entity.Entity.}\bfcode{getLink}}{}{}~\begin{quote}\begin{description}
\item[{Returns jQuery}] \leavevmode
the link

\end{description}\end{quote}

\end{fulllineitems}


Returns a jQuery element containing a link that, when clicked, selects the entity.


\paragraph{resolveReference}
\label{ambrosia_web.entity.Entity:resolvereference}\index{ambrosia\_web.entity.Entity.resolveReference() (ambrosia\_web.entity.Entity method)}

\begin{fulllineitems}
\phantomsection\label{ambrosia_web.entity.Entity:ambrosia_web.entity.Entity.resolveReference}\pysiglinewithargsret{\code{ambrosia\_web.entity.Entity.}\bfcode{resolveReference}}{}{}
\end{fulllineitems}


resolves all references


\strong{See also:}


{\hyperref[ambrosia.model:ambrosia.model.Event.to_serializeable]{\code{ambrosia.model.Event.to\_serializeable()}}}




\paragraph{select}
\label{ambrosia_web.entity.Entity:select}\index{ambrosia\_web.entity.Entity.select() (ambrosia\_web.entity.Entity method)}

\begin{fulllineitems}
\phantomsection\label{ambrosia_web.entity.Entity:ambrosia_web.entity.Entity.select}\pysiglinewithargsret{\code{ambrosia\_web.entity.Entity.}\bfcode{select}}{}{}
\end{fulllineitems}


This method should be called when the user selects an entity.


\subsection{Namespace ambrosia\_web.entity.entities}
\label{ambrosia_web.entity.entities:namespace-ambrosia-web-entity-entities}\label{ambrosia_web.entity.entities::doc}

\subsubsection{Constructor}
\label{ambrosia_web.entity.entities:constructor}\index{ambrosia\_web.entity.entities() (class)}

\begin{fulllineitems}
\phantomsection\label{ambrosia_web.entity.entities:ambrosia_web.entity.entities}\pysiglinewithargsret{\strong{class }\code{ambrosia\_web.entity.}\bfcode{entities}}{}{}
\end{fulllineitems}



\subsection{Class ambrosia\_web.entity.entities-App}
\label{ambrosia_web.entity.entities-App:class-ambrosia-web-entity-entities-app}\label{ambrosia_web.entity.entities-App::doc}
Represents {\hyperref[ambrosia.model:ambrosia.model.entities.App]{\code{ambrosia.model.entities.App}}}


\subsubsection{Constructor}
\label{ambrosia_web.entity.entities-App:constructor}\index{ambrosia\_web.entity.entities-App() (class)}

\begin{fulllineitems}
\phantomsection\label{ambrosia_web.entity.entities-App:ambrosia_web.entity.entities-App}\pysiglinewithargsret{\strong{class }\code{ambrosia\_web.entity.}\bfcode{entities-App}}{}{}
\end{fulllineitems}



\subsection{Class ambrosia\_web.entity.entities-File}
\label{ambrosia_web.entity.entities-File::doc}\label{ambrosia_web.entity.entities-File:class-ambrosia-web-entity-entities-file}
Represents {\hyperref[ambrosia.model:ambrosia.model.entities.File]{\code{ambrosia.model.entities.File}}}


\subsubsection{Constructor}
\label{ambrosia_web.entity.entities-File:constructor}\index{ambrosia\_web.entity.entities-File() (class)}

\begin{fulllineitems}
\phantomsection\label{ambrosia_web.entity.entities-File:ambrosia_web.entity.entities-File}\pysiglinewithargsret{\strong{class }\code{ambrosia\_web.entity.}\bfcode{entities-File}}{}{}
\end{fulllineitems}



\subsection{Class ambrosia\_web.entity.entities-ServerEndpoint}
\label{ambrosia_web.entity.entities-ServerEndpoint:class-ambrosia-web-entity-entities-serverendpoint}\label{ambrosia_web.entity.entities-ServerEndpoint::doc}
Represents {\hyperref[ambrosia.model:ambrosia.model.entities.ServerEndpoint]{\code{ambrosia.model.entities.ServerEndpoint}}}


\subsubsection{Constructor}
\label{ambrosia_web.entity.entities-ServerEndpoint:constructor}\index{ambrosia\_web.entity.entities-ServerEndpoint() (class)}

\begin{fulllineitems}
\phantomsection\label{ambrosia_web.entity.entities-ServerEndpoint:ambrosia_web.entity.entities-ServerEndpoint}\pysiglinewithargsret{\strong{class }\code{ambrosia\_web.entity.}\bfcode{entities-ServerEndpoint}}{}{}
\end{fulllineitems}



\subsection{Class ambrosia\_web.entity.entities-Task}
\label{ambrosia_web.entity.entities-Task:class-ambrosia-web-entity-entities-task}\label{ambrosia_web.entity.entities-Task::doc}
Represents {\hyperref[ambrosia.model:ambrosia.model.entities.Task]{\code{ambrosia.model.entities.Task}}}


\subsubsection{Constructor}
\label{ambrosia_web.entity.entities-Task:constructor}\index{ambrosia\_web.entity.entities-Task() (class)}

\begin{fulllineitems}
\phantomsection\label{ambrosia_web.entity.entities-Task:ambrosia_web.entity.entities-Task}\pysiglinewithargsret{\strong{class }\code{ambrosia\_web.entity.}\bfcode{entities-Task}}{}{}
\end{fulllineitems}



\subsection{Namespace ambrosia\_web.event}
\label{ambrosia_web.event::doc}\label{ambrosia_web.event:namespace-ambrosia-web-event}

\subsubsection{Constructor}
\label{ambrosia_web.event:constructor}\index{ambrosia\_web.event() (class)}

\begin{fulllineitems}
\phantomsection\label{ambrosia_web.event:ambrosia_web.event}\pysiglinewithargsret{\strong{class }\code{ambrosia\_web.}\bfcode{event}}{}{}
\end{fulllineitems}



\subsubsection{Methods}
\label{ambrosia_web.event:methods}

\paragraph{clearSelect}
\label{ambrosia_web.event:clearselect}\index{ambrosia\_web.event.clearSelect() (ambrosia\_web.event method)}

\begin{fulllineitems}
\phantomsection\label{ambrosia_web.event:ambrosia_web.event.clearSelect}\pysiglinewithargsret{\code{ambrosia\_web.event.}\bfcode{clearSelect}}{}{}
\end{fulllineitems}


unselect all events


\paragraph{enrich}
\label{ambrosia_web.event:enrich}\index{ambrosia\_web.event.enrich() (ambrosia\_web.event method)}

\begin{fulllineitems}
\phantomsection\label{ambrosia_web.event:ambrosia_web.event.enrich}\pysiglinewithargsret{\code{ambrosia\_web.event.}\bfcode{enrich}}{\emph{el}, \emph{parent}}{}~\begin{quote}\begin{description}
\item[{Arguments}] \leavevmode\begin{itemize}
\item {} 
\textbf{el} (\emph{object}) -- the deserialized data

\item {} 
\textbf{parent} ({\hyperref[ambrosia_web.event.Event:ambrosia_web.event.Event]{\emph{ambrosia\_web.event.Event}}}) -- the events parent event (if exists)

\end{itemize}

\end{description}\end{quote}

\end{fulllineitems}


Receives an object containing the deserialized data from the server and returns an instance of the class
{\hyperref[ambrosia_web.event.Event:ambrosia_web.event.Event]{\code{ambrosia\_web.event.Event()}}}


\paragraph{reset}
\label{ambrosia_web.event:reset}\index{ambrosia\_web.event.reset() (ambrosia\_web.event method)}

\begin{fulllineitems}
\phantomsection\label{ambrosia_web.event:ambrosia_web.event.reset}\pysiglinewithargsret{\code{ambrosia\_web.event.}\bfcode{reset}}{}{}
\end{fulllineitems}


Resets the default \code{A.layout.BlockLayoutManager()}


\subsubsection{Attributes}
\label{ambrosia_web.event:attributes}

\paragraph{BLOCK\_MARGIN\_X}
\label{ambrosia_web.event:block-margin-x}\index{BLOCK\_MARGIN\_X (None attribute)}

\begin{fulllineitems}
\phantomsection\label{ambrosia_web.event:BLOCK_MARGIN_X}\pysigline{\bfcode{BLOCK\_MARGIN\_X}}
\end{fulllineitems}


The horizontal space Ambrosia should keep between two adjacent event


\paragraph{BLOCK\_MARGIN\_Y}
\label{ambrosia_web.event:block-margin-y}\index{BLOCK\_MARGIN\_Y (None attribute)}

\begin{fulllineitems}
\phantomsection\label{ambrosia_web.event:BLOCK_MARGIN_Y}\pysigline{\bfcode{BLOCK\_MARGIN\_Y}}
\end{fulllineitems}


The vertical space Ambrosia should keep between two adjacent event


\paragraph{BLOCK\_PADDING}
\label{ambrosia_web.event:block-padding}\index{BLOCK\_PADDING (None attribute)}

\begin{fulllineitems}
\phantomsection\label{ambrosia_web.event:BLOCK_PADDING}\pysigline{\bfcode{BLOCK\_PADDING}}
\end{fulllineitems}


The horizontal space Ambrosia should keep between the borders of a child event and its parent (in pixel)


\paragraph{BLOCK\_WIDTH}
\label{ambrosia_web.event:block-width}\index{BLOCK\_WIDTH (None attribute)}

\begin{fulllineitems}
\phantomsection\label{ambrosia_web.event:BLOCK_WIDTH}\pysigline{\bfcode{BLOCK\_WIDTH}}
\end{fulllineitems}


The minimum width of a block (in pixel)


\paragraph{DEFAULT\_BLOCK\_HEIGHT}
\label{ambrosia_web.event:default-block-height}\index{DEFAULT\_BLOCK\_HEIGHT (None attribute)}

\begin{fulllineitems}
\phantomsection\label{ambrosia_web.event:DEFAULT_BLOCK_HEIGHT}\pysigline{\bfcode{DEFAULT\_BLOCK\_HEIGHT}}
\end{fulllineitems}


The default height in seconds for an event


\paragraph{DEFAULT\_BLOCK\_LAYOUT\_MANAGER}
\label{ambrosia_web.event:default-block-layout-manager}\index{DEFAULT\_BLOCK\_LAYOUT\_MANAGER (None attribute)}

\begin{fulllineitems}
\phantomsection\label{ambrosia_web.event:DEFAULT_BLOCK_LAYOUT_MANAGER}\pysigline{\bfcode{DEFAULT\_BLOCK\_LAYOUT\_MANAGER}}
\end{fulllineitems}


The default {\hyperref[ambrosia_web.layout.BlockLayoutManager:ambrosia_web.layout.BlockLayoutManager]{\code{ambrosia\_web.layout.BlockLayoutManager()}}} that is used on the top level


\paragraph{onSelectHandler}
\label{ambrosia_web.event:onselecthandler}\index{onSelectHandler (None attribute)}

\begin{fulllineitems}
\phantomsection\label{ambrosia_web.event:onSelectHandler}\pysigline{\bfcode{onSelectHandler}}
\end{fulllineitems}


contains all handlers for selecting events. Any part of the application may listen to those events (i.e. add a
function to this array). If the user select an entity the interface can adapt to this (e.g. the
{\hyperref[ambrosia_web.view.detailsview.DetailsView:ambrosia_web.view.detailsview.DetailsView]{\code{ambrosia\_web.view.detailsview.DetailsView()}}} shows details about this event).


\paragraph{onUnSelectHandler}
\label{ambrosia_web.event:onunselecthandler}\index{onUnSelectHandler (None attribute)}

\begin{fulllineitems}
\phantomsection\label{ambrosia_web.event:onUnSelectHandler}\pysigline{\bfcode{onUnSelectHandler}}
\end{fulllineitems}


contains all handlers for unselecting events. Any part of the application may listen to those events (i.e. add a
function to this array). If the user unselect an entity the interface can adapt to this (e.g. the
{\hyperref[ambrosia_web.view.detailsview.DetailsView:ambrosia_web.view.detailsview.DetailsView]{\code{ambrosia\_web.view.detailsview.DetailsView()}}} shows details about this event).


\subsection{Class ambrosia\_web.event.BlockEvent}
\label{ambrosia_web.event.BlockEvent:class-ambrosia-web-event-blockevent}\label{ambrosia_web.event.BlockEvent::doc}
Base class for all events that are drawn as a block.


\subsubsection{Constructor}
\label{ambrosia_web.event.BlockEvent:constructor}\index{ambrosia\_web.event.BlockEvent() (class)}

\begin{fulllineitems}
\phantomsection\label{ambrosia_web.event.BlockEvent:ambrosia_web.event.BlockEvent}\pysiglinewithargsret{\strong{class }\code{ambrosia\_web.event.}\bfcode{BlockEvent}}{}{}
Bases: {\hyperref[ambrosia_web.event.Event:ambrosia_web.event.Event]{\code{ambrosia\_web.event.Event()}}}

\end{fulllineitems}



\subsubsection{Methods}
\label{ambrosia_web.event.BlockEvent:methods}

\paragraph{calcDimensions}
\label{ambrosia_web.event.BlockEvent:calcdimensions}\index{ambrosia\_web.event.BlockEvent.calcDimensions() (ambrosia\_web.event.BlockEvent method)}

\begin{fulllineitems}
\phantomsection\label{ambrosia_web.event.BlockEvent:ambrosia_web.event.BlockEvent.calcDimensions}\pysiglinewithargsret{\code{ambrosia\_web.event.BlockEvent.}\bfcode{calcDimensions}}{\emph{blockLayoutManager}}{}~\begin{quote}\begin{description}
\item[{Arguments}] \leavevmode\begin{itemize}
\item {} 
\textbf{blockLayoutManager} ({\hyperref[ambrosia_web.layout.BlockLayoutManager:ambrosia_web.layout.BlockLayoutManager]{\emph{ambrosia\_web.layout.BlockLayoutManager}}}) -- the block layout manager to use

\end{itemize}

\end{description}\end{quote}

\end{fulllineitems}


Calculates the dimensions of the visualisation (for block events). The top level events are drawn using the
default block layout manager. Each event that has visible children creates a new block layout manager that
is used to position the children (the children's calcDimensions method is called). The block layout manager
that was used to position the children holds the width and height that is required to draw all children.
Afterwards (using this width/height) the parent event is drawn.


\paragraph{draw}
\label{ambrosia_web.event.BlockEvent:draw}\index{ambrosia\_web.event.BlockEvent.draw() (ambrosia\_web.event.BlockEvent method)}

\begin{fulllineitems}
\phantomsection\label{ambrosia_web.event.BlockEvent:ambrosia_web.event.BlockEvent.draw}\pysiglinewithargsret{\code{ambrosia\_web.event.BlockEvent.}\bfcode{draw}}{\emph{xOffset}}{}~\begin{quote}\begin{description}
\item[{Arguments}] \leavevmode\begin{itemize}
\item {} 
\textbf{xOffset} (\emph{int}) -- (optional) if this is a child object, the x position of the parent

\end{itemize}

\end{description}\end{quote}

\end{fulllineitems}


draws the event


\subsection{Class ambrosia\_web.event.Event}
\label{ambrosia_web.event.Event:class-ambrosia-web-event-event}\label{ambrosia_web.event.Event::doc}
The client side counterpart for an event


\strong{See also:}


{\hyperref[ambrosia.model:ambrosia.model.Event]{\code{ambrosia.model.Event}}}




\subsubsection{Constructor}
\label{ambrosia_web.event.Event:constructor}\index{ambrosia\_web.event.Event() (class)}

\begin{fulllineitems}
\phantomsection\label{ambrosia_web.event.Event:ambrosia_web.event.Event}\pysiglinewithargsret{\strong{class }\code{ambrosia\_web.event.}\bfcode{Event}}{}{}
\end{fulllineitems}



\subsubsection{Methods}
\label{ambrosia_web.event.Event:methods}

\paragraph{calcDimensions}
\label{ambrosia_web.event.Event:calcdimensions}\index{ambrosia\_web.event.Event.calcDimensions() (ambrosia\_web.event.Event method)}

\begin{fulllineitems}
\phantomsection\label{ambrosia_web.event.Event:ambrosia_web.event.Event.calcDimensions}\pysiglinewithargsret{\code{ambrosia\_web.event.Event.}\bfcode{calcDimensions}}{\emph{blockLayoutManager}}{}~\begin{quote}\begin{description}
\item[{Arguments}] \leavevmode\begin{itemize}
\item {} 
\textbf{blockLayoutManager} ({\hyperref[ambrosia_web.layout.BlockLayoutManager:ambrosia_web.layout.BlockLayoutManager]{\emph{ambrosia\_web.layout.BlockLayoutManager}}}) -- the block layout manager to use

\end{itemize}

\end{description}\end{quote}

\end{fulllineitems}


Calculates the dimensions of the visualisation (for block events). Should be called second when drawing.
events.


\paragraph{calcVisible}
\label{ambrosia_web.event.Event:calcvisible}\index{ambrosia\_web.event.Event.calcVisible() (ambrosia\_web.event.Event method)}

\begin{fulllineitems}
\phantomsection\label{ambrosia_web.event.Event:ambrosia_web.event.Event.calcVisible}\pysiglinewithargsret{\code{ambrosia\_web.event.Event.}\bfcode{calcVisible}}{}{}
\end{fulllineitems}


This is the first method called when drawing events. It calculates if an element should be shown and also
considers the visibility of the child elements (a child can force it's parent to show)


\paragraph{draw}
\label{ambrosia_web.event.Event:draw}\index{ambrosia\_web.event.Event.draw() (ambrosia\_web.event.Event method)}

\begin{fulllineitems}
\phantomsection\label{ambrosia_web.event.Event:ambrosia_web.event.Event.draw}\pysiglinewithargsret{\code{ambrosia\_web.event.Event.}\bfcode{draw}}{}{}
\end{fulllineitems}


Draw the event. Should be called third when drawing. Must be implemented by subclass.


\paragraph{getLink}
\label{ambrosia_web.event.Event:getlink}\index{ambrosia\_web.event.Event.getLink() (ambrosia\_web.event.Event method)}

\begin{fulllineitems}
\phantomsection\label{ambrosia_web.event.Event:ambrosia_web.event.Event.getLink}\pysiglinewithargsret{\code{ambrosia\_web.event.Event.}\bfcode{getLink}}{}{}~\begin{quote}\begin{description}
\item[{Returns jQuery}] \leavevmode
the link

\end{description}\end{quote}

\end{fulllineitems}


Returns a jQuery element containing a link that, when clicked, selects the event.


\paragraph{select}
\label{ambrosia_web.event.Event:select}\index{ambrosia\_web.event.Event.select() (ambrosia\_web.event.Event method)}

\begin{fulllineitems}
\phantomsection\label{ambrosia_web.event.Event:ambrosia_web.event.Event.select}\pysiglinewithargsret{\code{ambrosia\_web.event.Event.}\bfcode{select}}{}{}
\end{fulllineitems}


This method should be called when the user selects one event.


\paragraph{selectAdd}
\label{ambrosia_web.event.Event:selectadd}\index{ambrosia\_web.event.Event.selectAdd() (ambrosia\_web.event.Event method)}

\begin{fulllineitems}
\phantomsection\label{ambrosia_web.event.Event:ambrosia_web.event.Event.selectAdd}\pysiglinewithargsret{\code{ambrosia\_web.event.Event.}\bfcode{selectAdd}}{}{}
\end{fulllineitems}


This method should be called when the user adds an event to a selection.


\paragraph{unselect}
\label{ambrosia_web.event.Event:unselect}\index{ambrosia\_web.event.Event.unselect() (ambrosia\_web.event.Event method)}

\begin{fulllineitems}
\phantomsection\label{ambrosia_web.event.Event:ambrosia_web.event.Event.unselect}\pysiglinewithargsret{\code{ambrosia\_web.event.Event.}\bfcode{unselect}}{}{}
\end{fulllineitems}


This method should be called when the user unselects one event.


\subsection{Class ambrosia\_web.event.LineEvent}
\label{ambrosia_web.event.LineEvent::doc}\label{ambrosia_web.event.LineEvent:class-ambrosia-web-event-lineevent}
Base class for all events that are drawn as a horizontal line across the main view.


\subsubsection{Constructor}
\label{ambrosia_web.event.LineEvent:constructor}\index{ambrosia\_web.event.LineEvent() (class)}

\begin{fulllineitems}
\phantomsection\label{ambrosia_web.event.LineEvent:ambrosia_web.event.LineEvent}\pysiglinewithargsret{\strong{class }\code{ambrosia\_web.event.}\bfcode{LineEvent}}{}{}
Bases: {\hyperref[ambrosia_web.event.Event:ambrosia_web.event.Event]{\code{ambrosia\_web.event.Event()}}}

\end{fulllineitems}



\subsubsection{Methods}
\label{ambrosia_web.event.LineEvent:methods}

\paragraph{draw}
\label{ambrosia_web.event.LineEvent:draw}\index{ambrosia\_web.event.LineEvent.draw() (ambrosia\_web.event.LineEvent method)}

\begin{fulllineitems}
\phantomsection\label{ambrosia_web.event.LineEvent:ambrosia_web.event.LineEvent.draw}\pysiglinewithargsret{\code{ambrosia\_web.event.LineEvent.}\bfcode{draw}}{}{}
\end{fulllineitems}


draws the line


\subsection{Namespace ambrosia\_web.event.events}
\label{ambrosia_web.event.events:namespace-ambrosia-web-event-events}\label{ambrosia_web.event.events::doc}

\subsubsection{Constructor}
\label{ambrosia_web.event.events:constructor}\index{ambrosia\_web.event.events() (class)}

\begin{fulllineitems}
\phantomsection\label{ambrosia_web.event.events:ambrosia_web.event.events}\pysiglinewithargsret{\strong{class }\code{ambrosia\_web.event.}\bfcode{events}}{}{}
\end{fulllineitems}



\subsection{Class ambrosia\_web.event.events-ANANASAdbShellExec}
\label{ambrosia_web.event.events-ANANASAdbShellExec:class-ambrosia-web-event-events-ananasadbshellexec}\label{ambrosia_web.event.events-ANANASAdbShellExec::doc}
Represents \code{ambrosia\_plugins.lkm.events.ANANASAdbShellExec}


\subsubsection{Constructor}
\label{ambrosia_web.event.events-ANANASAdbShellExec:constructor}\index{ambrosia\_web.event.events-ANANASAdbShellExec() (class)}

\begin{fulllineitems}
\phantomsection\label{ambrosia_web.event.events-ANANASAdbShellExec:ambrosia_web.event.events-ANANASAdbShellExec}\pysiglinewithargsret{\strong{class }\code{ambrosia\_web.event.}\bfcode{events-ANANASAdbShellExec}}{}{}
\end{fulllineitems}



\subsection{Class ambrosia\_web.event.events-ANANASAdbShellExecEvent}
\label{ambrosia_web.event.events-ANANASAdbShellExecEvent:class-ambrosia-web-event-events-ananasadbshellexecevent}\label{ambrosia_web.event.events-ANANASAdbShellExecEvent::doc}
Represents {\hyperref[ambrosia_plugins.lkm:ambrosia_plugins.lkm.events.ANANASAdbShellExecEvent]{\code{ambrosia\_plugins.lkm.events.ANANASAdbShellExecEvent}}}


\subsubsection{Constructor}
\label{ambrosia_web.event.events-ANANASAdbShellExecEvent:constructor}\index{ambrosia\_web.event.events-ANANASAdbShellExecEvent() (class)}

\begin{fulllineitems}
\phantomsection\label{ambrosia_web.event.events-ANANASAdbShellExecEvent:ambrosia_web.event.events-ANANASAdbShellExecEvent}\pysiglinewithargsret{\strong{class }\code{ambrosia\_web.event.}\bfcode{events-ANANASAdbShellExecEvent}}{}{}
\end{fulllineitems}



\subsection{Class ambrosia\_web.event.events-ANANASEvent}
\label{ambrosia_web.event.events-ANANASEvent:class-ambrosia-web-event-events-ananasevent}\label{ambrosia_web.event.events-ANANASEvent::doc}
Represents {\hyperref[ambrosia_plugins.events:ambrosia_plugins.events.ANANASEvent]{\code{ambrosia\_plugins.events.ANANASEvent}}}


\subsubsection{Constructor}
\label{ambrosia_web.event.events-ANANASEvent:constructor}\index{ambrosia\_web.event.events-ANANASEvent() (class)}

\begin{fulllineitems}
\phantomsection\label{ambrosia_web.event.events-ANANASEvent:ambrosia_web.event.events-ANANASEvent}\pysiglinewithargsret{\strong{class }\code{ambrosia\_web.event.}\bfcode{events-ANANASEvent}}{}{}
\end{fulllineitems}



\subsection{Class ambrosia\_web.event.events-APKInstallEvent}
\label{ambrosia_web.event.events-APKInstallEvent:class-ambrosia-web-event-events-apkinstallevent}\label{ambrosia_web.event.events-APKInstallEvent::doc}
Represents {\hyperref[ambrosia_plugins.lkm:ambrosia_plugins.lkm.events.APKInstallEvent]{\code{ambrosia\_plugins.lkm.events.APKInstallEvent}}}


\subsubsection{Constructor}
\label{ambrosia_web.event.events-APKInstallEvent:constructor}\index{ambrosia\_web.event.events-APKInstallEvent() (class)}

\begin{fulllineitems}
\phantomsection\label{ambrosia_web.event.events-APKInstallEvent:ambrosia_web.event.events-APKInstallEvent}\pysiglinewithargsret{\strong{class }\code{ambrosia\_web.event.}\bfcode{events-APKInstallEvent}}{}{}
\end{fulllineitems}



\subsection{Class ambrosia\_web.event.events-AndroidApicall}
\label{ambrosia_web.event.events-AndroidApicall::doc}\label{ambrosia_web.event.events-AndroidApicall:class-ambrosia-web-event-events-androidapicall}
Represents \code{ambrosia\_plugins.apimonitor.AndroidApicall}


\subsubsection{Constructor}
\label{ambrosia_web.event.events-AndroidApicall:constructor}\index{ambrosia\_web.event.events-AndroidApicall() (class)}

\begin{fulllineitems}
\phantomsection\label{ambrosia_web.event.events-AndroidApicall:ambrosia_web.event.events-AndroidApicall}\pysiglinewithargsret{\strong{class }\code{ambrosia\_web.event.}\bfcode{events-AndroidApicall}}{}{}
\end{fulllineitems}



\subsection{Class ambrosia\_web.event.events-AndroidApicallEvent}
\label{ambrosia_web.event.events-AndroidApicallEvent:class-ambrosia-web-event-events-androidapicallevent}\label{ambrosia_web.event.events-AndroidApicallEvent::doc}
Represents {\hyperref[ambrosia_plugins.apimonitor:ambrosia_plugins.apimonitor.AndroidApicallEvent]{\code{ambrosia\_plugins.apimonitor.AndroidApicallEvent}}}


\subsubsection{Constructor}
\label{ambrosia_web.event.events-AndroidApicallEvent:constructor}\index{ambrosia\_web.event.events-AndroidApicallEvent() (class)}

\begin{fulllineitems}
\phantomsection\label{ambrosia_web.event.events-AndroidApicallEvent:ambrosia_web.event.events-AndroidApicallEvent}\pysiglinewithargsret{\strong{class }\code{ambrosia\_web.event.}\bfcode{events-AndroidApicallEvent}}{}{}
\end{fulllineitems}



\subsection{Class ambrosia\_web.event.events-AnonymousFileEvent}
\label{ambrosia_web.event.events-AnonymousFileEvent:class-ambrosia-web-event-events-anonymousfileevent}\label{ambrosia_web.event.events-AnonymousFileEvent::doc}
Represents {\hyperref[ambrosia_plugins.lkm:ambrosia_plugins.lkm.events.AnonymousFileEvent]{\code{ambrosia\_plugins.lkm.events.AnonymousFileEvent}}}


\subsubsection{Constructor}
\label{ambrosia_web.event.events-AnonymousFileEvent:constructor}\index{ambrosia\_web.event.events-AnonymousFileEvent() (class)}

\begin{fulllineitems}
\phantomsection\label{ambrosia_web.event.events-AnonymousFileEvent:ambrosia_web.event.events-AnonymousFileEvent}\pysiglinewithargsret{\strong{class }\code{ambrosia\_web.event.}\bfcode{events-AnonymousFileEvent}}{}{}
\end{fulllineitems}



\subsection{Class ambrosia\_web.event.events-CallLogAccess}
\label{ambrosia_web.event.events-CallLogAccess::doc}\label{ambrosia_web.event.events-CallLogAccess:class-ambrosia-web-event-events-calllogaccess}
Represents \code{ambrosia\_plugins.apimonitor.CallLogAccess}


\subsubsection{Constructor}
\label{ambrosia_web.event.events-CallLogAccess:constructor}\index{ambrosia\_web.event.events-CallLogAccess() (class)}

\begin{fulllineitems}
\phantomsection\label{ambrosia_web.event.events-CallLogAccess:ambrosia_web.event.events-CallLogAccess}\pysiglinewithargsret{\strong{class }\code{ambrosia\_web.event.}\bfcode{events-CallLogAccess}}{}{}
\end{fulllineitems}



\subsection{Class ambrosia\_web.event.events-CallLogAccessEvent}
\label{ambrosia_web.event.events-CallLogAccessEvent:class-ambrosia-web-event-events-calllogaccessevent}\label{ambrosia_web.event.events-CallLogAccessEvent::doc}
Represents {\hyperref[ambrosia_plugins.apimonitor:ambrosia_plugins.apimonitor.CallLogAccessEvent]{\code{ambrosia\_plugins.apimonitor.CallLogAccessEvent}}}


\subsubsection{Constructor}
\label{ambrosia_web.event.events-CallLogAccessEvent:constructor}\index{ambrosia\_web.event.events-CallLogAccessEvent() (class)}

\begin{fulllineitems}
\phantomsection\label{ambrosia_web.event.events-CallLogAccessEvent:ambrosia_web.event.events-CallLogAccessEvent}\pysiglinewithargsret{\strong{class }\code{ambrosia\_web.event.}\bfcode{events-CallLogAccessEvent}}{}{}
\end{fulllineitems}



\subsection{Class ambrosia\_web.event.events-CommandExecuteEvent}
\label{ambrosia_web.event.events-CommandExecuteEvent::doc}\label{ambrosia_web.event.events-CommandExecuteEvent:class-ambrosia-web-event-events-commandexecuteevent}
Represents {\hyperref[ambrosia_plugins.lkm:ambrosia_plugins.lkm.events.CommandExecuteEvent]{\code{ambrosia\_plugins.lkm.events.CommandExecuteEvent}}}


\subsubsection{Constructor}
\label{ambrosia_web.event.events-CommandExecuteEvent:constructor}\index{ambrosia\_web.event.events-CommandExecuteEvent() (class)}

\begin{fulllineitems}
\phantomsection\label{ambrosia_web.event.events-CommandExecuteEvent:ambrosia_web.event.events-CommandExecuteEvent}\pysiglinewithargsret{\strong{class }\code{ambrosia\_web.event.}\bfcode{events-CommandExecuteEvent}}{}{}
\end{fulllineitems}



\subsection{Class ambrosia\_web.event.events-ContactAccessEvent}
\label{ambrosia_web.event.events-ContactAccessEvent::doc}\label{ambrosia_web.event.events-ContactAccessEvent:class-ambrosia-web-event-events-contactaccessevent}
Represents {\hyperref[ambrosia_plugins.apimonitor:ambrosia_plugins.apimonitor.ContactAccessEvent]{\code{ambrosia\_plugins.apimonitor.ContactAccessEvent}}}


\subsubsection{Constructor}
\label{ambrosia_web.event.events-ContactAccessEvent:constructor}\index{ambrosia\_web.event.events-ContactAccessEvent() (class)}

\begin{fulllineitems}
\phantomsection\label{ambrosia_web.event.events-ContactAccessEvent:ambrosia_web.event.events-ContactAccessEvent}\pysiglinewithargsret{\strong{class }\code{ambrosia\_web.event.}\bfcode{events-ContactAccessEvent}}{}{}
\end{fulllineitems}



\subsection{Class ambrosia\_web.event.events-ContactsAccess}
\label{ambrosia_web.event.events-ContactsAccess::doc}\label{ambrosia_web.event.events-ContactsAccess:class-ambrosia-web-event-events-contactsaccess}
Represents \code{ambrosia\_plugins.apimonitor.ContactsAccess}


\subsubsection{Constructor}
\label{ambrosia_web.event.events-ContactsAccess:constructor}\index{ambrosia\_web.event.events-ContactsAccess() (class)}

\begin{fulllineitems}
\phantomsection\label{ambrosia_web.event.events-ContactsAccess:ambrosia_web.event.events-ContactsAccess}\pysiglinewithargsret{\strong{class }\code{ambrosia\_web.event.}\bfcode{events-ContactsAccess}}{}{}
\end{fulllineitems}



\subsection{Class ambrosia\_web.event.events-CreateDir}
\label{ambrosia_web.event.events-CreateDir::doc}\label{ambrosia_web.event.events-CreateDir:class-ambrosia-web-event-events-createdir}
Represents \code{ambrosia\_plugins.lkm.events.CreateDir}


\subsubsection{Constructor}
\label{ambrosia_web.event.events-CreateDir:constructor}\index{ambrosia\_web.event.events-CreateDir() (class)}

\begin{fulllineitems}
\phantomsection\label{ambrosia_web.event.events-CreateDir:ambrosia_web.event.events-CreateDir}\pysiglinewithargsret{\strong{class }\code{ambrosia\_web.event.}\bfcode{events-CreateDir}}{}{}
\end{fulllineitems}



\subsection{Class ambrosia\_web.event.events-CreateDirEvent}
\label{ambrosia_web.event.events-CreateDirEvent:class-ambrosia-web-event-events-createdirevent}\label{ambrosia_web.event.events-CreateDirEvent::doc}
Represents {\hyperref[ambrosia_plugins.lkm:ambrosia_plugins.lkm.events.CreateDirEvent]{\code{ambrosia\_plugins.lkm.events.CreateDirEvent}}}


\subsubsection{Constructor}
\label{ambrosia_web.event.events-CreateDirEvent:constructor}\index{ambrosia\_web.event.events-CreateDirEvent() (class)}

\begin{fulllineitems}
\phantomsection\label{ambrosia_web.event.events-CreateDirEvent:ambrosia_web.event.events-CreateDirEvent}\pysiglinewithargsret{\strong{class }\code{ambrosia\_web.event.}\bfcode{events-CreateDirEvent}}{}{}
\end{fulllineitems}



\subsection{Class ambrosia\_web.event.events-DeleteFileEvent}
\label{ambrosia_web.event.events-DeleteFileEvent:class-ambrosia-web-event-events-deletefileevent}\label{ambrosia_web.event.events-DeleteFileEvent::doc}
Represents \code{ambrosia\_plugins.lkm.events.DeleteFileEvent}


\subsubsection{Constructor}
\label{ambrosia_web.event.events-DeleteFileEvent:constructor}\index{ambrosia\_web.event.events-DeleteFileEvent() (class)}

\begin{fulllineitems}
\phantomsection\label{ambrosia_web.event.events-DeleteFileEvent:ambrosia_web.event.events-DeleteFileEvent}\pysiglinewithargsret{\strong{class }\code{ambrosia\_web.event.}\bfcode{events-DeleteFileEvent}}{}{}
\end{fulllineitems}



\subsection{Class ambrosia\_web.event.events-DeletePathEvent}
\label{ambrosia_web.event.events-DeletePathEvent:class-ambrosia-web-event-events-deletepathevent}\label{ambrosia_web.event.events-DeletePathEvent::doc}
Represents {\hyperref[ambrosia_plugins.lkm:ambrosia_plugins.lkm.events.DeletePathEvent]{\code{ambrosia\_plugins.lkm.events.DeletePathEvent}}}


\subsubsection{Constructor}
\label{ambrosia_web.event.events-DeletePathEvent:constructor}\index{ambrosia\_web.event.events-DeletePathEvent() (class)}

\begin{fulllineitems}
\phantomsection\label{ambrosia_web.event.events-DeletePathEvent:ambrosia_web.event.events-DeletePathEvent}\pysiglinewithargsret{\strong{class }\code{ambrosia\_web.event.}\bfcode{events-DeletePathEvent}}{}{}
\end{fulllineitems}



\subsection{Class ambrosia\_web.event.events-ExecEvent}
\label{ambrosia_web.event.events-ExecEvent::doc}\label{ambrosia_web.event.events-ExecEvent:class-ambrosia-web-event-events-execevent}
Represents {\hyperref[ambrosia_plugins.lkm:ambrosia_plugins.lkm.events.ExecEvent]{\code{ambrosia\_plugins.lkm.events.ExecEvent}}}


\subsubsection{Constructor}
\label{ambrosia_web.event.events-ExecEvent:constructor}\index{ambrosia\_web.event.events-ExecEvent() (class)}

\begin{fulllineitems}
\phantomsection\label{ambrosia_web.event.events-ExecEvent:ambrosia_web.event.events-ExecEvent}\pysiglinewithargsret{\strong{class }\code{ambrosia\_web.event.}\bfcode{events-ExecEvent}}{}{}
\end{fulllineitems}



\subsection{Class ambrosia\_web.event.events-FileEvent}
\label{ambrosia_web.event.events-FileEvent::doc}\label{ambrosia_web.event.events-FileEvent:class-ambrosia-web-event-events-fileevent}
Represents {\hyperref[ambrosia_plugins.lkm:ambrosia_plugins.lkm.events.FileEvent]{\code{ambrosia\_plugins.lkm.events.FileEvent}}}


\subsubsection{Constructor}
\label{ambrosia_web.event.events-FileEvent:constructor}\index{ambrosia\_web.event.events-FileEvent() (class)}

\begin{fulllineitems}
\phantomsection\label{ambrosia_web.event.events-FileEvent:ambrosia_web.event.events-FileEvent}\pysiglinewithargsret{\strong{class }\code{ambrosia\_web.event.}\bfcode{events-FileEvent}}{}{}
\end{fulllineitems}



\subsection{Class ambrosia\_web.event.events-JavaLibraryLoadEvent}
\label{ambrosia_web.event.events-JavaLibraryLoadEvent::doc}\label{ambrosia_web.event.events-JavaLibraryLoadEvent:class-ambrosia-web-event-events-javalibraryloadevent}
Represents {\hyperref[ambrosia_plugins.lkm:ambrosia_plugins.lkm.events.JavaLibraryLoadEvent]{\code{ambrosia\_plugins.lkm.events.JavaLibraryLoadEvent}}}


\subsubsection{Constructor}
\label{ambrosia_web.event.events-JavaLibraryLoadEvent:constructor}\index{ambrosia\_web.event.events-JavaLibraryLoadEvent() (class)}

\begin{fulllineitems}
\phantomsection\label{ambrosia_web.event.events-JavaLibraryLoadEvent:ambrosia_web.event.events-JavaLibraryLoadEvent}\pysiglinewithargsret{\strong{class }\code{ambrosia\_web.event.}\bfcode{events-JavaLibraryLoadEvent}}{}{}
\end{fulllineitems}



\subsection{Class ambrosia\_web.event.events-LibraryLoad}
\label{ambrosia_web.event.events-LibraryLoad:class-ambrosia-web-event-events-libraryload}\label{ambrosia_web.event.events-LibraryLoad::doc}
Represents \code{ambrosia\_plugins.lkm.events.LibraryLoad}


\subsubsection{Constructor}
\label{ambrosia_web.event.events-LibraryLoad:constructor}\index{ambrosia\_web.event.events-LibraryLoad() (class)}

\begin{fulllineitems}
\phantomsection\label{ambrosia_web.event.events-LibraryLoad:ambrosia_web.event.events-LibraryLoad}\pysiglinewithargsret{\strong{class }\code{ambrosia\_web.event.}\bfcode{events-LibraryLoad}}{}{}
\end{fulllineitems}



\subsection{Class ambrosia\_web.event.events-LibraryLoadEvent}
\label{ambrosia_web.event.events-LibraryLoadEvent:class-ambrosia-web-event-events-libraryloadevent}\label{ambrosia_web.event.events-LibraryLoadEvent::doc}
Represents {\hyperref[ambrosia_plugins.lkm:ambrosia_plugins.lkm.events.LibraryLoadEvent]{\code{ambrosia\_plugins.lkm.events.LibraryLoadEvent}}}


\subsubsection{Constructor}
\label{ambrosia_web.event.events-LibraryLoadEvent:constructor}\index{ambrosia\_web.event.events-LibraryLoadEvent() (class)}

\begin{fulllineitems}
\phantomsection\label{ambrosia_web.event.events-LibraryLoadEvent:ambrosia_web.event.events-LibraryLoadEvent}\pysiglinewithargsret{\strong{class }\code{ambrosia\_web.event.}\bfcode{events-LibraryLoadEvent}}{}{}
\end{fulllineitems}



\subsection{Class ambrosia\_web.event.events-MemoryMapEvent}
\label{ambrosia_web.event.events-MemoryMapEvent::doc}\label{ambrosia_web.event.events-MemoryMapEvent:class-ambrosia-web-event-events-memorymapevent}
Represents {\hyperref[ambrosia_plugins.lkm:ambrosia_plugins.lkm.events.MemoryMapEvent]{\code{ambrosia\_plugins.lkm.events.MemoryMapEvent}}}


\subsubsection{Constructor}
\label{ambrosia_web.event.events-MemoryMapEvent:constructor}\index{ambrosia\_web.event.events-MemoryMapEvent() (class)}

\begin{fulllineitems}
\phantomsection\label{ambrosia_web.event.events-MemoryMapEvent:ambrosia_web.event.events-MemoryMapEvent}\pysiglinewithargsret{\strong{class }\code{ambrosia\_web.event.}\bfcode{events-MemoryMapEvent}}{}{}
\end{fulllineitems}



\subsection{Class ambrosia\_web.event.events-PhoneCall}
\label{ambrosia_web.event.events-PhoneCall:class-ambrosia-web-event-events-phonecall}\label{ambrosia_web.event.events-PhoneCall::doc}
Represents \code{ambrosia\_plugins.apimonitor.PhoneCall}


\subsubsection{Constructor}
\label{ambrosia_web.event.events-PhoneCall:constructor}\index{ambrosia\_web.event.events-PhoneCall() (class)}

\begin{fulllineitems}
\phantomsection\label{ambrosia_web.event.events-PhoneCall:ambrosia_web.event.events-PhoneCall}\pysiglinewithargsret{\strong{class }\code{ambrosia\_web.event.}\bfcode{events-PhoneCall}}{}{}
\end{fulllineitems}



\subsection{Class ambrosia\_web.event.events-PhoneCallEvent}
\label{ambrosia_web.event.events-PhoneCallEvent:class-ambrosia-web-event-events-phonecallevent}\label{ambrosia_web.event.events-PhoneCallEvent::doc}
Represents {\hyperref[ambrosia_plugins.apimonitor:ambrosia_plugins.apimonitor.PhoneCallEvent]{\code{ambrosia\_plugins.apimonitor.PhoneCallEvent}}}


\subsubsection{Constructor}
\label{ambrosia_web.event.events-PhoneCallEvent:constructor}\index{ambrosia\_web.event.events-PhoneCallEvent() (class)}

\begin{fulllineitems}
\phantomsection\label{ambrosia_web.event.events-PhoneCallEvent:ambrosia_web.event.events-PhoneCallEvent}\pysiglinewithargsret{\strong{class }\code{ambrosia\_web.event.}\bfcode{events-PhoneCallEvent}}{}{}
\end{fulllineitems}



\subsection{Class ambrosia\_web.event.events-SMSAccess}
\label{ambrosia_web.event.events-SMSAccess:class-ambrosia-web-event-events-smsaccess}\label{ambrosia_web.event.events-SMSAccess::doc}
Represents \code{ambrosia\_plugins.apimonitor.SMSAccess}


\subsubsection{Constructor}
\label{ambrosia_web.event.events-SMSAccess:constructor}\index{ambrosia\_web.event.events-SMSAccess() (class)}

\begin{fulllineitems}
\phantomsection\label{ambrosia_web.event.events-SMSAccess:ambrosia_web.event.events-SMSAccess}\pysiglinewithargsret{\strong{class }\code{ambrosia\_web.event.}\bfcode{events-SMSAccess}}{}{}
\end{fulllineitems}



\subsection{Class ambrosia\_web.event.events-SMSAccessEvent}
\label{ambrosia_web.event.events-SMSAccessEvent::doc}\label{ambrosia_web.event.events-SMSAccessEvent:class-ambrosia-web-event-events-smsaccessevent}
Represents {\hyperref[ambrosia_plugins.apimonitor:ambrosia_plugins.apimonitor.SMSAccessEvent]{\code{ambrosia\_plugins.apimonitor.SMSAccessEvent}}}


\subsubsection{Constructor}
\label{ambrosia_web.event.events-SMSAccessEvent:constructor}\index{ambrosia\_web.event.events-SMSAccessEvent() (class)}

\begin{fulllineitems}
\phantomsection\label{ambrosia_web.event.events-SMSAccessEvent:ambrosia_web.event.events-SMSAccessEvent}\pysiglinewithargsret{\strong{class }\code{ambrosia\_web.event.}\bfcode{events-SMSAccessEvent}}{}{}
\end{fulllineitems}



\subsection{Class ambrosia\_web.event.events-SendSignal}
\label{ambrosia_web.event.events-SendSignal::doc}\label{ambrosia_web.event.events-SendSignal:class-ambrosia-web-event-events-sendsignal}
Represents \code{ambrosia\_plugins.lkm.events.SendSignal}


\subsubsection{Constructor}
\label{ambrosia_web.event.events-SendSignal:constructor}\index{ambrosia\_web.event.events-SendSignal() (class)}

\begin{fulllineitems}
\phantomsection\label{ambrosia_web.event.events-SendSignal:ambrosia_web.event.events-SendSignal}\pysiglinewithargsret{\strong{class }\code{ambrosia\_web.event.}\bfcode{events-SendSignal}}{}{}
\end{fulllineitems}



\subsection{Class ambrosia\_web.event.events-SendSignalEvent}
\label{ambrosia_web.event.events-SendSignalEvent:class-ambrosia-web-event-events-sendsignalevent}\label{ambrosia_web.event.events-SendSignalEvent::doc}
Represents {\hyperref[ambrosia_plugins.lkm:ambrosia_plugins.lkm.events.SendSignalEvent]{\code{ambrosia\_plugins.lkm.events.SendSignalEvent}}}


\subsubsection{Constructor}
\label{ambrosia_web.event.events-SendSignalEvent:constructor}\index{ambrosia\_web.event.events-SendSignalEvent() (class)}

\begin{fulllineitems}
\phantomsection\label{ambrosia_web.event.events-SendSignalEvent:ambrosia_web.event.events-SendSignalEvent}\pysiglinewithargsret{\strong{class }\code{ambrosia\_web.event.}\bfcode{events-SendSignalEvent}}{}{}
\end{fulllineitems}



\subsection{Class ambrosia\_web.event.events-SocketAccept}
\label{ambrosia_web.event.events-SocketAccept:class-ambrosia-web-event-events-socketaccept}\label{ambrosia_web.event.events-SocketAccept::doc}
Represents \code{ambrosia\_plugins.lkm.events.SocketAccept}


\subsubsection{Constructor}
\label{ambrosia_web.event.events-SocketAccept:constructor}\index{ambrosia\_web.event.events-SocketAccept() (class)}

\begin{fulllineitems}
\phantomsection\label{ambrosia_web.event.events-SocketAccept:ambrosia_web.event.events-SocketAccept}\pysiglinewithargsret{\strong{class }\code{ambrosia\_web.event.}\bfcode{events-SocketAccept}}{}{}
\end{fulllineitems}



\subsection{Class ambrosia\_web.event.events-SocketAcceptEvent}
\label{ambrosia_web.event.events-SocketAcceptEvent::doc}\label{ambrosia_web.event.events-SocketAcceptEvent:class-ambrosia-web-event-events-socketacceptevent}
Represents {\hyperref[ambrosia_plugins.lkm:ambrosia_plugins.lkm.events.SocketAcceptEvent]{\code{ambrosia\_plugins.lkm.events.SocketAcceptEvent}}}


\subsubsection{Constructor}
\label{ambrosia_web.event.events-SocketAcceptEvent:constructor}\index{ambrosia\_web.event.events-SocketAcceptEvent() (class)}

\begin{fulllineitems}
\phantomsection\label{ambrosia_web.event.events-SocketAcceptEvent:ambrosia_web.event.events-SocketAcceptEvent}\pysiglinewithargsret{\strong{class }\code{ambrosia\_web.event.}\bfcode{events-SocketAcceptEvent}}{}{}
\end{fulllineitems}



\subsection{Class ambrosia\_web.event.events-SocketEvent}
\label{ambrosia_web.event.events-SocketEvent:class-ambrosia-web-event-events-socketevent}\label{ambrosia_web.event.events-SocketEvent::doc}
Represents {\hyperref[ambrosia_plugins.lkm:ambrosia_plugins.lkm.events.SocketEvent]{\code{ambrosia\_plugins.lkm.events.SocketEvent}}}


\subsubsection{Constructor}
\label{ambrosia_web.event.events-SocketEvent:constructor}\index{ambrosia\_web.event.events-SocketEvent() (class)}

\begin{fulllineitems}
\phantomsection\label{ambrosia_web.event.events-SocketEvent:ambrosia_web.event.events-SocketEvent}\pysiglinewithargsret{\strong{class }\code{ambrosia\_web.event.}\bfcode{events-SocketEvent}}{}{}
\end{fulllineitems}



\subsection{Class ambrosia\_web.event.events-StartTaskEvent}
\label{ambrosia_web.event.events-StartTaskEvent:class-ambrosia-web-event-events-starttaskevent}\label{ambrosia_web.event.events-StartTaskEvent::doc}
Represents {\hyperref[ambrosia_plugins.lkm:ambrosia_plugins.lkm.events.StartTaskEvent]{\code{ambrosia\_plugins.lkm.events.StartTaskEvent}}}


\subsubsection{Constructor}
\label{ambrosia_web.event.events-StartTaskEvent:constructor}\index{ambrosia\_web.event.events-StartTaskEvent() (class)}

\begin{fulllineitems}
\phantomsection\label{ambrosia_web.event.events-StartTaskEvent:ambrosia_web.event.events-StartTaskEvent}\pysiglinewithargsret{\strong{class }\code{ambrosia\_web.event.}\bfcode{events-StartTaskEvent}}{}{}
\end{fulllineitems}



\subsection{Class ambrosia\_web.event.events-SuperUserRequest}
\label{ambrosia_web.event.events-SuperUserRequest:class-ambrosia-web-event-events-superuserrequest}\label{ambrosia_web.event.events-SuperUserRequest::doc}
Represents \code{ambrosia\_plugins.lkm.events.SuperUserRequest}


\subsubsection{Constructor}
\label{ambrosia_web.event.events-SuperUserRequest:constructor}\index{ambrosia\_web.event.events-SuperUserRequest() (class)}

\begin{fulllineitems}
\phantomsection\label{ambrosia_web.event.events-SuperUserRequest:ambrosia_web.event.events-SuperUserRequest}\pysiglinewithargsret{\strong{class }\code{ambrosia\_web.event.}\bfcode{events-SuperUserRequest}}{}{}
\end{fulllineitems}



\subsection{Class ambrosia\_web.event.events-SuperUserRequestEvent}
\label{ambrosia_web.event.events-SuperUserRequestEvent::doc}\label{ambrosia_web.event.events-SuperUserRequestEvent:class-ambrosia-web-event-events-superuserrequestevent}
Represents {\hyperref[ambrosia_plugins.lkm:ambrosia_plugins.lkm.events.SuperUserRequestEvent]{\code{ambrosia\_plugins.lkm.events.SuperUserRequestEvent}}}


\subsubsection{Constructor}
\label{ambrosia_web.event.events-SuperUserRequestEvent:constructor}\index{ambrosia\_web.event.events-SuperUserRequestEvent() (class)}

\begin{fulllineitems}
\phantomsection\label{ambrosia_web.event.events-SuperUserRequestEvent:ambrosia_web.event.events-SuperUserRequestEvent}\pysiglinewithargsret{\strong{class }\code{ambrosia\_web.event.}\bfcode{events-SuperUserRequestEvent}}{}{}
\end{fulllineitems}



\subsection{Class ambrosia\_web.event.events-SyscallEvent}
\label{ambrosia_web.event.events-SyscallEvent:class-ambrosia-web-event-events-syscallevent}\label{ambrosia_web.event.events-SyscallEvent::doc}
Represents {\hyperref[ambrosia_plugins.lkm:ambrosia_plugins.lkm.events.SyscallEvent]{\code{ambrosia\_plugins.lkm.events.SyscallEvent}}}


\subsubsection{Constructor}
\label{ambrosia_web.event.events-SyscallEvent:constructor}\index{ambrosia\_web.event.events-SyscallEvent() (class)}

\begin{fulllineitems}
\phantomsection\label{ambrosia_web.event.events-SyscallEvent:ambrosia_web.event.events-SyscallEvent}\pysiglinewithargsret{\strong{class }\code{ambrosia\_web.event.}\bfcode{events-SyscallEvent}}{}{}
\end{fulllineitems}



\subsection{Class ambrosia\_web.event.events-UnknownFdEvent}
\label{ambrosia_web.event.events-UnknownFdEvent:class-ambrosia-web-event-events-unknownfdevent}\label{ambrosia_web.event.events-UnknownFdEvent::doc}
Represents {\hyperref[ambrosia_plugins.lkm:ambrosia_plugins.lkm.events.UnknownFdEvent]{\code{ambrosia\_plugins.lkm.events.UnknownFdEvent}}}


\subsubsection{Constructor}
\label{ambrosia_web.event.events-UnknownFdEvent:constructor}\index{ambrosia\_web.event.events-UnknownFdEvent() (class)}

\begin{fulllineitems}
\phantomsection\label{ambrosia_web.event.events-UnknownFdEvent:ambrosia_web.event.events-UnknownFdEvent}\pysiglinewithargsret{\strong{class }\code{ambrosia\_web.event.}\bfcode{events-UnknownFdEvent}}{}{}
\end{fulllineitems}



\subsection{Class ambrosia\_web.event.events-ZygoteForkEvent}
\label{ambrosia_web.event.events-ZygoteForkEvent::doc}\label{ambrosia_web.event.events-ZygoteForkEvent:class-ambrosia-web-event-events-zygoteforkevent}
Represents {\hyperref[ambrosia_plugins.lkm:ambrosia_plugins.lkm.events.ZygoteForkEvent]{\code{ambrosia\_plugins.lkm.events.ZygoteForkEvent}}}


\subsubsection{Constructor}
\label{ambrosia_web.event.events-ZygoteForkEvent:constructor}\index{ambrosia\_web.event.events-ZygoteForkEvent() (class)}

\begin{fulllineitems}
\phantomsection\label{ambrosia_web.event.events-ZygoteForkEvent:ambrosia_web.event.events-ZygoteForkEvent}\pysiglinewithargsret{\strong{class }\code{ambrosia\_web.event.}\bfcode{events-ZygoteForkEvent}}{}{}
\end{fulllineitems}



\subsection{Namespace ambrosia\_web.filter}
\label{ambrosia_web.filter:namespace-ambrosia-web-filter}\label{ambrosia_web.filter::doc}

\subsubsection{Constructor}
\label{ambrosia_web.filter:constructor}\index{ambrosia\_web.filter() (class)}

\begin{fulllineitems}
\phantomsection\label{ambrosia_web.filter:ambrosia_web.filter}\pysiglinewithargsret{\strong{class }\code{ambrosia\_web.}\bfcode{filter}}{}{}
\end{fulllineitems}



\subsubsection{Methods}
\label{ambrosia_web.filter:methods}

\paragraph{handleLogicalOperation}
\label{ambrosia_web.filter:handlelogicaloperation}\index{ambrosia\_web.filter.handleLogicalOperation() (ambrosia\_web.filter method)}

\begin{fulllineitems}
\phantomsection\label{ambrosia_web.filter:ambrosia_web.filter.handleLogicalOperation}\pysiglinewithargsret{\code{ambrosia\_web.filter.}\bfcode{handleLogicalOperation}}{\emph{ex1}, \emph{rest}}{}~\begin{quote}\begin{description}
\item[{Arguments}] \leavevmode\begin{itemize}
\item {} 
\textbf{ex1} -- an expression

\item {} 
\textbf{rest} -- an array containing a logical operation and a second expression or undefined

\end{itemize}

\item[{Returns *}] \leavevmode
\end{description}\end{quote}

\end{fulllineitems}


Helper function for the parser.


\subsubsection{Attributes}
\label{ambrosia_web.filter:attributes}

\paragraph{addFilterHandler}
\label{ambrosia_web.filter:addfilterhandler}\index{addFilterHandler (None attribute)}

\begin{fulllineitems}
\phantomsection\label{ambrosia_web.filter:addFilterHandler}\pysigline{\bfcode{addFilterHandler}}
\end{fulllineitems}


contains all handlers for adding filters to an  event class. Any part of the application may listen to those
events (i.e. add a function to this array). If the user select an entity the interface can adapt to this.


\paragraph{removeFilterHandler}
\label{ambrosia_web.filter:removefilterhandler}\index{removeFilterHandler (None attribute)}

\begin{fulllineitems}
\phantomsection\label{ambrosia_web.filter:removeFilterHandler}\pysigline{\bfcode{removeFilterHandler}}
\end{fulllineitems}


contains all handlers for removing filters from an  event class. Any part of the application may listen to those
events (i.e. add a function to this array). If the user select an entity the interface can adapt to this.


\subsection{Class ambrosia\_web.filter.BlacklistFilter}
\label{ambrosia_web.filter.BlacklistFilter:class-ambrosia-web-filter-blacklistfilter}\label{ambrosia_web.filter.BlacklistFilter::doc}
A blacklisting filter


\subsubsection{Constructor}
\label{ambrosia_web.filter.BlacklistFilter:constructor}\index{ambrosia\_web.filter.BlacklistFilter() (class)}

\begin{fulllineitems}
\phantomsection\label{ambrosia_web.filter.BlacklistFilter:ambrosia_web.filter.BlacklistFilter}\pysiglinewithargsret{\strong{class }\code{ambrosia\_web.filter.}\bfcode{BlacklistFilter}}{\emph{rule}, \emph{description}, \emph{enabled}}{}~\begin{quote}\begin{description}
\item[{Arguments}] \leavevmode\begin{itemize}
\item {} 
\textbf{rule} (\emph{String}) -- the condition for the filter

\item {} 
\textbf{description} (\emph{String}) -- a string describing the filter

\item {} 
\textbf{enabled} (\emph{bool}) -- (optional) whether the filter is effective

\end{itemize}

\end{description}\end{quote}

\end{fulllineitems}



\subsection{Class ambrosia\_web.filter.Comparison}
\label{ambrosia_web.filter.Comparison:class-ambrosia-web-filter-comparison}\label{ambrosia_web.filter.Comparison::doc}
A comparison. Used by the parser.


\subsubsection{Constructor}
\label{ambrosia_web.filter.Comparison:constructor}\index{ambrosia\_web.filter.Comparison() (class)}

\begin{fulllineitems}
\phantomsection\label{ambrosia_web.filter.Comparison:ambrosia_web.filter.Comparison}\pysiglinewithargsret{\strong{class }\code{ambrosia\_web.filter.}\bfcode{Comparison}}{\emph{p1}, \emph{op}, \emph{p2}}{}~\begin{quote}\begin{description}
\item[{Arguments}] \leavevmode\begin{itemize}
\item {} 
\textbf{p1} -- the first value that is compared

\item {} 
\textbf{op} -- the compare operation

\item {} 
\textbf{p2} -- the sencond value

\end{itemize}

\end{description}\end{quote}

\end{fulllineitems}



\subsection{Class ambrosia\_web.filter.Filter}
\label{ambrosia_web.filter.Filter:class-ambrosia-web-filter-filter}\label{ambrosia_web.filter.Filter::doc}
A Filter represents a single condition (either entered by the user or a default condition).

The following shows example for the filter syntax:


\subsubsection{Examples}
\label{ambrosia_web.filter.Filter:examples}
\begin{Verbatim}[commandchars=\\\{\}]
\PYG{o}{!}\PYG{p}{(}\PYG{n+nx}{test} \PYG{o}{==} \PYG{l+m+mf}{1.2} \PYG{o}{\textbar{}\textbar{}} \PYG{p}{(}\PYG{n+nx}{test} \PYG{o}{\PYGZgt{}} \PYG{l+m+mi}{2} \PYG{o}{\PYGZam{}\PYGZam{}} \PYG{n+nx}{p}\PYG{p}{.}\PYG{n+nx}{bar} \PYG{o}{!=} \PYG{l+s+s2}{\PYGZdq{}foobar\PYGZdq{}}\PYG{p}{)} \PYG{o}{\textbar{}\textbar{}} \PYG{k+kc}{true} \PYG{p}{)} \PYG{o}{\PYGZam{}\PYGZam{}} \PYG{o}{!}\PYG{k+kc}{false}
\end{Verbatim}

The logical operations ``\&\&'' and `!!' as well as the unary logical operation ''!'' are allowed. Parentheses may be
used to change the default precedence of the operations.

These logical operations manage ``comparisons''. A ``comparison'' may compare two values using the operators ``=='',
''!='', ``\textgreater{}='', ``\textless{}='', ``\textless{}'', ``\textasciitilde{}'' (the first value matches a regex defined by the second value), '':'' (the second value
is an array and the first element is contained in the second one) and ''!:'' (the first value is not contained
in the second value).

A value may be a string in the form of ``string'', a number in the form of 1.0 or 1, true or false or a property.
A property is a string describing an attribute of an event (e.g. abspath, successful). Moreover a property may
also match a specific reference (e.g. r.process.pid, r.file.abspath). The reference defined in a property may be
a specific reference (like r.file or r.process). Moreover the string ``*'' may be used to get all values
(e.g. r.*.id). Since multiple values are returned, the value  must be treated as an array (Array operations '':''
and ''!:'' must be used). A general filter (that is applied to all events regardless of their type) can therefore
be used to find all events related to a certain entity (e.g. ``someidofanentity'' : r.*.id).


\subsubsection{Constructor}
\label{ambrosia_web.filter.Filter:constructor}\index{ambrosia\_web.filter.Filter() (class)}

\begin{fulllineitems}
\phantomsection\label{ambrosia_web.filter.Filter:ambrosia_web.filter.Filter}\pysiglinewithargsret{\strong{class }\code{ambrosia\_web.filter.}\bfcode{Filter}}{}{}
\end{fulllineitems}



\subsubsection{Methods}
\label{ambrosia_web.filter.Filter:methods}

\paragraph{evaluate}
\label{ambrosia_web.filter.Filter:evaluate}\index{ambrosia\_web.filter.Filter.evaluate() (ambrosia\_web.filter.Filter method)}

\begin{fulllineitems}
\phantomsection\label{ambrosia_web.filter.Filter:ambrosia_web.filter.Filter.evaluate}\pysiglinewithargsret{\code{ambrosia\_web.filter.Filter.}\bfcode{evaluate}}{}{}~\begin{quote}\begin{description}
\item[{Returns bool}] \leavevmode
true if the event matches

\end{description}\end{quote}

\end{fulllineitems}


Evaluate if an an event matches this filter


\paragraph{isEnabled}
\label{ambrosia_web.filter.Filter:isenabled}\index{ambrosia\_web.filter.Filter.isEnabled() (ambrosia\_web.filter.Filter method)}

\begin{fulllineitems}
\phantomsection\label{ambrosia_web.filter.Filter:ambrosia_web.filter.Filter.isEnabled}\pysiglinewithargsret{\code{ambrosia\_web.filter.Filter.}\bfcode{isEnabled}}{}{}~\begin{quote}\begin{description}
\item[{Returns bool}] \leavevmode
true if enabled

\end{description}\end{quote}

\end{fulllineitems}


Checks whether this filter is enabled


\paragraph{setDescription}
\label{ambrosia_web.filter.Filter:setdescription}\index{ambrosia\_web.filter.Filter.setDescription() (ambrosia\_web.filter.Filter method)}

\begin{fulllineitems}
\phantomsection\label{ambrosia_web.filter.Filter:ambrosia_web.filter.Filter.setDescription}\pysiglinewithargsret{\code{ambrosia\_web.filter.Filter.}\bfcode{setDescription}}{\emph{d}}{}~\begin{quote}\begin{description}
\item[{Arguments}] \leavevmode\begin{itemize}
\item {} 
\textbf{d} (\emph{String}) -- the description

\end{itemize}

\end{description}\end{quote}

\end{fulllineitems}


set the description


\paragraph{setEnabled}
\label{ambrosia_web.filter.Filter:setenabled}\index{ambrosia\_web.filter.Filter.setEnabled() (ambrosia\_web.filter.Filter method)}

\begin{fulllineitems}
\phantomsection\label{ambrosia_web.filter.Filter:ambrosia_web.filter.Filter.setEnabled}\pysiglinewithargsret{\code{ambrosia\_web.filter.Filter.}\bfcode{setEnabled}}{\emph{b}}{}~\begin{quote}\begin{description}
\item[{Arguments}] \leavevmode\begin{itemize}
\item {} 
\textbf{b} (\emph{bool}) -- whether the filter should be enabled

\end{itemize}

\end{description}\end{quote}

\end{fulllineitems}


enable or disable the filter


\paragraph{setRule}
\label{ambrosia_web.filter.Filter:setrule}\index{ambrosia\_web.filter.Filter.setRule() (ambrosia\_web.filter.Filter method)}

\begin{fulllineitems}
\phantomsection\label{ambrosia_web.filter.Filter:ambrosia_web.filter.Filter.setRule}\pysiglinewithargsret{\code{ambrosia\_web.filter.Filter.}\bfcode{setRule}}{\emph{r}}{}~\begin{quote}\begin{description}
\item[{Arguments}] \leavevmode\begin{itemize}
\item {} 
\textbf{r} (\emph{String}) -- the new rule in filter syntax

\end{itemize}

\end{description}\end{quote}

\end{fulllineitems}


replaces the current rule with a new one


\subsection{Class ambrosia\_web.filter.LogicalOperation}
\label{ambrosia_web.filter.LogicalOperation:class-ambrosia-web-filter-logicaloperation}\label{ambrosia_web.filter.LogicalOperation::doc}
Logical operations like ``\&\&'' and ''!!''. Used by the parser


\subsubsection{Constructor}
\label{ambrosia_web.filter.LogicalOperation:constructor}\index{ambrosia\_web.filter.LogicalOperation() (class)}

\begin{fulllineitems}
\phantomsection\label{ambrosia_web.filter.LogicalOperation:ambrosia_web.filter.LogicalOperation}\pysiglinewithargsret{\strong{class }\code{ambrosia\_web.filter.}\bfcode{LogicalOperation}}{\emph{p1}, \emph{op}, \emph{p2}}{}~\begin{quote}\begin{description}
\item[{Arguments}] \leavevmode\begin{itemize}
\item {} 
\textbf{p1} -- the first expression

\item {} 
\textbf{op} -- the operation

\item {} 
\textbf{p2} -- the second expression

\end{itemize}

\end{description}\end{quote}

\end{fulllineitems}



\subsection{Class ambrosia\_web.filter.Property}
\label{ambrosia_web.filter.Property:class-ambrosia-web-filter-property}\label{ambrosia_web.filter.Property::doc}
A property used in a filter. Used by the parser.


\subsubsection{Constructor}
\label{ambrosia_web.filter.Property:constructor}\index{ambrosia\_web.filter.Property() (class)}

\begin{fulllineitems}
\phantomsection\label{ambrosia_web.filter.Property:ambrosia_web.filter.Property}\pysiglinewithargsret{\strong{class }\code{ambrosia\_web.filter.}\bfcode{Property}}{\emph{s}}{}~\begin{quote}\begin{description}
\item[{Arguments}] \leavevmode\begin{itemize}
\item {} 
\textbf{s} -- the property string

\end{itemize}

\end{description}\end{quote}

\end{fulllineitems}



\subsection{Class ambrosia\_web.filter.UnaryOperator}
\label{ambrosia_web.filter.UnaryOperator:class-ambrosia-web-filter-unaryoperator}\label{ambrosia_web.filter.UnaryOperator::doc}
Unary operators. Used by the parser


\subsubsection{Constructor}
\label{ambrosia_web.filter.UnaryOperator:constructor}\index{ambrosia\_web.filter.UnaryOperator() (class)}

\begin{fulllineitems}
\phantomsection\label{ambrosia_web.filter.UnaryOperator:ambrosia_web.filter.UnaryOperator}\pysiglinewithargsret{\strong{class }\code{ambrosia\_web.filter.}\bfcode{UnaryOperator}}{\emph{op}, \emph{expression}}{}~\begin{quote}\begin{description}
\item[{Arguments}] \leavevmode\begin{itemize}
\item {} 
\textbf{op} -- the operation e.g. NOT

\item {} 
\textbf{expression} -- the expression the operator is applied to

\end{itemize}

\end{description}\end{quote}

\end{fulllineitems}



\subsection{Namespace ambrosia\_web.layout}
\label{ambrosia_web.layout:namespace-ambrosia-web-layout}\label{ambrosia_web.layout::doc}

\subsubsection{Constructor}
\label{ambrosia_web.layout:constructor}\index{ambrosia\_web.layout() (class)}

\begin{fulllineitems}
\phantomsection\label{ambrosia_web.layout:ambrosia_web.layout}\pysiglinewithargsret{\strong{class }\code{ambrosia\_web.}\bfcode{layout}}{}{}
\end{fulllineitems}



\subsection{Class ambrosia\_web.layout.BlockLayoutManager}
\label{ambrosia_web.layout.BlockLayoutManager:class-ambrosia-web-layout-blocklayoutmanager}\label{ambrosia_web.layout.BlockLayoutManager::doc}
The block layout manager is used to position event block in the main view.


\strong{See also:}


{\hyperref[ambrosia_web.layout.BlockLayoutManager:ambrosia_web.layout.BlockLayoutManager.fitBlock]{\code{ambrosia\_web.layout.BlockLayoutManager.fitBlock()}}} for details.



Note: in order for the block layout manager to properly work, the events have to be fitted in ascending order (x
position)


\subsubsection{Constructor}
\label{ambrosia_web.layout.BlockLayoutManager:constructor}\index{ambrosia\_web.layout.BlockLayoutManager() (class)}

\begin{fulllineitems}
\phantomsection\label{ambrosia_web.layout.BlockLayoutManager:ambrosia_web.layout.BlockLayoutManager}\pysiglinewithargsret{\strong{class }\code{ambrosia\_web.layout.}\bfcode{BlockLayoutManager}}{}{}
\end{fulllineitems}



\subsubsection{Methods}
\label{ambrosia_web.layout.BlockLayoutManager:methods}

\paragraph{fitBlock}
\label{ambrosia_web.layout.BlockLayoutManager:fitblock}\index{ambrosia\_web.layout.BlockLayoutManager.fitBlock() (ambrosia\_web.layout.BlockLayoutManager method)}

\begin{fulllineitems}
\phantomsection\label{ambrosia_web.layout.BlockLayoutManager:ambrosia_web.layout.BlockLayoutManager.fitBlock}\pysiglinewithargsret{\code{ambrosia\_web.layout.BlockLayoutManager.}\bfcode{fitBlock}}{\emph{dim}, \emph{margin\_x}, \emph{margin\_y}}{}~\begin{quote}\begin{description}
\item[{Arguments}] \leavevmode\begin{itemize}
\item {} 
\textbf{dim} ({\hyperref[ambrosia_web.layout.Dimensions:ambrosia_web.layout.Dimensions]{\emph{ambrosia\_web.layout.Dimensions}}}) -- the dimensions of the block (may overlap other events)

\item {} 
\textbf{margin\_x} (\emph{int}) -- the horizontal margin that should be left

\item {} 
\textbf{margin\_y} (\emph{int}) -- the vertical margin that should be left

\end{itemize}

\item[{Returns ambrosia\_web.layout.Dimensions}] \leavevmode
the new dimensions of the non-overlapping block

\end{description}\end{quote}

\end{fulllineitems}


Takes a {\hyperref[ambrosia_web.layout.Dimensions:ambrosia_web.layout.Dimensions]{\code{ambrosia\_web.layout.Dimensions()}}} object and tries to fit it considering the previously
fitted blocks.


\paragraph{getEndY}
\label{ambrosia_web.layout.BlockLayoutManager:getendy}\index{ambrosia\_web.layout.BlockLayoutManager.getEndY() (ambrosia\_web.layout.BlockLayoutManager method)}

\begin{fulllineitems}
\phantomsection\label{ambrosia_web.layout.BlockLayoutManager:ambrosia_web.layout.BlockLayoutManager.getEndY}\pysiglinewithargsret{\code{ambrosia\_web.layout.BlockLayoutManager.}\bfcode{getEndY}}{}{}~\begin{quote}\begin{description}
\item[{Returns number}] \leavevmode
\end{description}\end{quote}

\end{fulllineitems}


position bottom border of the block layout manager (considering all fitted events)


\paragraph{getWidth}
\label{ambrosia_web.layout.BlockLayoutManager:getwidth}\index{ambrosia\_web.layout.BlockLayoutManager.getWidth() (ambrosia\_web.layout.BlockLayoutManager method)}

\begin{fulllineitems}
\phantomsection\label{ambrosia_web.layout.BlockLayoutManager:ambrosia_web.layout.BlockLayoutManager.getWidth}\pysiglinewithargsret{\code{ambrosia\_web.layout.BlockLayoutManager.}\bfcode{getWidth}}{}{}~\begin{quote}\begin{description}
\item[{Returns number}] \leavevmode
\end{description}\end{quote}

\end{fulllineitems}


get the width of the whole block layout manager (considering all fitted events)


\subsection{Class ambrosia\_web.layout.Dimensions}
\label{ambrosia_web.layout.Dimensions:class-ambrosia-web-layout-dimensions}\label{ambrosia_web.layout.Dimensions::doc}
Helper class that represents the dimensions of a block


\subsubsection{Constructor}
\label{ambrosia_web.layout.Dimensions:constructor}\index{ambrosia\_web.layout.Dimensions() (class)}

\begin{fulllineitems}
\phantomsection\label{ambrosia_web.layout.Dimensions:ambrosia_web.layout.Dimensions}\pysiglinewithargsret{\strong{class }\code{ambrosia\_web.layout.}\bfcode{Dimensions}}{\emph{x}, \emph{y}, \emph{width}, \emph{height}}{}~\begin{quote}\begin{description}
\item[{Arguments}] \leavevmode\begin{itemize}
\item {} 
\textbf{x} -- the x position

\item {} 
\textbf{y} -- the y position

\item {} 
\textbf{width} -- the width

\item {} 
\textbf{height} -- the height

\end{itemize}

\end{description}\end{quote}

\end{fulllineitems}



\subsection{Namespace ambrosia\_web.util}
\label{ambrosia_web.util:namespace-ambrosia-web-util}\label{ambrosia_web.util::doc}

\subsubsection{Constructor}
\label{ambrosia_web.util:constructor}\index{ambrosia\_web.util() (class)}

\begin{fulllineitems}
\phantomsection\label{ambrosia_web.util:ambrosia_web.util}\pysiglinewithargsret{\strong{class }\code{ambrosia\_web.}\bfcode{util}}{}{}
\end{fulllineitems}



\subsubsection{Methods}
\label{ambrosia_web.util:methods}

\paragraph{assert}
\label{ambrosia_web.util:assert}\index{ambrosia\_web.util.assert() (ambrosia\_web.util method)}

\begin{fulllineitems}
\phantomsection\label{ambrosia_web.util:ambrosia_web.util.assert}\pysiglinewithargsret{\code{ambrosia\_web.util.}\bfcode{assert}}{\emph{b}}{}~\begin{quote}\begin{description}
\item[{Arguments}] \leavevmode\begin{itemize}
\item {} 
\textbf{b} (\emph{bool}) -- 

\end{itemize}

\end{description}\end{quote}

\end{fulllineitems}


Simple helper function that raises an exception when false is passed


\paragraph{deserialize}
\label{ambrosia_web.util:deserialize}\index{ambrosia\_web.util.deserialize() (ambrosia\_web.util method)}

\begin{fulllineitems}
\phantomsection\label{ambrosia_web.util:ambrosia_web.util.deserialize}\pysiglinewithargsret{\code{ambrosia\_web.util.}\bfcode{deserialize}}{\emph{obj}, \emph{objs}}{}~\begin{quote}\begin{description}
\item[{Arguments}] \leavevmode\begin{itemize}
\item {} 
\textbf{obj} -- the obj from Ambrosia

\item {} 
\textbf{objs} -- the objs from Ambrosia

\end{itemize}

\end{description}\end{quote}

\end{fulllineitems}


deserialize results from Ambrosia


\subsection{Class ambrosia\_web.util.Log}
\label{ambrosia_web.util.Log:class-ambrosia-web-util-log}\label{ambrosia_web.util.Log::doc}
The class that handles logging


\subsubsection{Constructor}
\label{ambrosia_web.util.Log:constructor}\index{ambrosia\_web.util.Log() (class)}

\begin{fulllineitems}
\phantomsection\label{ambrosia_web.util.Log:ambrosia_web.util.Log}\pysiglinewithargsret{\strong{class }\code{ambrosia\_web.util.}\bfcode{Log}}{}{}
\end{fulllineitems}



\subsubsection{Methods}
\label{ambrosia_web.util.Log:methods}

\paragraph{D}
\label{ambrosia_web.util.Log:d}\index{ambrosia\_web.util.Log.D() (ambrosia\_web.util.Log method)}

\begin{fulllineitems}
\phantomsection\label{ambrosia_web.util.Log:ambrosia_web.util.Log.D}\pysiglinewithargsret{\code{ambrosia\_web.util.Log.}\bfcode{D}}{\emph{str}}{}~\begin{quote}\begin{description}
\item[{Arguments}] \leavevmode\begin{itemize}
\item {} 
\textbf{str} (\emph{String}) -- the message to log

\end{itemize}

\end{description}\end{quote}

\end{fulllineitems}


shortcut for debug logging


\paragraph{E}
\label{ambrosia_web.util.Log:e}\index{ambrosia\_web.util.Log.E() (ambrosia\_web.util.Log method)}

\begin{fulllineitems}
\phantomsection\label{ambrosia_web.util.Log:ambrosia_web.util.Log.E}\pysiglinewithargsret{\code{ambrosia\_web.util.Log.}\bfcode{E}}{\emph{str}}{}~\begin{quote}\begin{description}
\item[{Arguments}] \leavevmode\begin{itemize}
\item {} 
\textbf{str} (\emph{String}) -- the message to log

\end{itemize}

\end{description}\end{quote}

\end{fulllineitems}


shortcut for error logging


\paragraph{I}
\label{ambrosia_web.util.Log:i}\index{ambrosia\_web.util.Log.I() (ambrosia\_web.util.Log method)}

\begin{fulllineitems}
\phantomsection\label{ambrosia_web.util.Log:ambrosia_web.util.Log.I}\pysiglinewithargsret{\code{ambrosia\_web.util.Log.}\bfcode{I}}{\emph{str}}{}~\begin{quote}\begin{description}
\item[{Arguments}] \leavevmode\begin{itemize}
\item {} 
\textbf{str} (\emph{String}) -- the message to log

\end{itemize}

\end{description}\end{quote}

\end{fulllineitems}


shortcut for info logging


\paragraph{log}
\label{ambrosia_web.util.Log:log}\index{ambrosia\_web.util.Log.log() (ambrosia\_web.util.Log method)}

\begin{fulllineitems}
\phantomsection\label{ambrosia_web.util.Log:ambrosia_web.util.Log.log}\pysiglinewithargsret{\code{ambrosia\_web.util.Log.}\bfcode{log}}{\emph{str}, \emph{level}}{}~\begin{quote}\begin{description}
\item[{Arguments}] \leavevmode\begin{itemize}
\item {} 
\textbf{str} (\emph{String}) -- the message to log

\item {} 
\textbf{level} (\emph{String}) -- the level: DEBUG, INFO, WARN, ERROR

\end{itemize}

\end{description}\end{quote}

\end{fulllineitems}


log an event


\paragraph{W}
\label{ambrosia_web.util.Log:w}\index{ambrosia\_web.util.Log.W() (ambrosia\_web.util.Log method)}

\begin{fulllineitems}
\phantomsection\label{ambrosia_web.util.Log:ambrosia_web.util.Log.W}\pysiglinewithargsret{\code{ambrosia\_web.util.Log.}\bfcode{W}}{\emph{str}}{}~\begin{quote}\begin{description}
\item[{Arguments}] \leavevmode\begin{itemize}
\item {} 
\textbf{str} (\emph{String}) -- the message to log

\end{itemize}

\end{description}\end{quote}

\end{fulllineitems}


shortcut for warn logging


\subsection{Namespace ambrosia\_web.view}
\label{ambrosia_web.view:namespace-ambrosia-web-view}\label{ambrosia_web.view::doc}

\subsubsection{Constructor}
\label{ambrosia_web.view:constructor}\index{ambrosia\_web.view() (class)}

\begin{fulllineitems}
\phantomsection\label{ambrosia_web.view:ambrosia_web.view}\pysiglinewithargsret{\strong{class }\code{ambrosia\_web.}\bfcode{view}}{}{}
\end{fulllineitems}



\subsubsection{Methods}
\label{ambrosia_web.view:methods}

\paragraph{hideAllPanels}
\label{ambrosia_web.view:hideallpanels}\index{ambrosia\_web.view.hideAllPanels() (ambrosia\_web.view method)}

\begin{fulllineitems}
\phantomsection\label{ambrosia_web.view:ambrosia_web.view.hideAllPanels}\pysiglinewithargsret{\code{ambrosia\_web.view.}\bfcode{hideAllPanels}}{}{}
\end{fulllineitems}


hide all panels


\subsection{Class ambrosia\_web.view.Panel}
\label{ambrosia_web.view.Panel:class-ambrosia-web-view-panel}\label{ambrosia_web.view.Panel::doc}
Base class for all panels (DetailsView, EntityView, FilterView)


\subsubsection{Constructor}
\label{ambrosia_web.view.Panel:constructor}\index{ambrosia\_web.view.Panel() (class)}

\begin{fulllineitems}
\phantomsection\label{ambrosia_web.view.Panel:ambrosia_web.view.Panel}\pysiglinewithargsret{\strong{class }\code{ambrosia\_web.view.}\bfcode{Panel}}{\emph{name}, \emph{element}}{}~\begin{quote}\begin{description}
\item[{Arguments}] \leavevmode\begin{itemize}
\item {} 
\textbf{name} (\emph{String}) -- the caption of the panel

\item {} 
\textbf{element} (\emph{jQuery}) -- the element to draw the panel into

\end{itemize}

\end{description}\end{quote}

\end{fulllineitems}



\subsection{Namespace ambrosia\_web.view.detailsview}
\label{ambrosia_web.view.detailsview:namespace-ambrosia-web-view-detailsview}\label{ambrosia_web.view.detailsview::doc}

\subsubsection{Constructor}
\label{ambrosia_web.view.detailsview:constructor}\index{ambrosia\_web.view.detailsview() (class)}

\begin{fulllineitems}
\phantomsection\label{ambrosia_web.view.detailsview:ambrosia_web.view.detailsview}\pysiglinewithargsret{\strong{class }\code{ambrosia\_web.view.}\bfcode{detailsview}}{}{}
\end{fulllineitems}



\subsection{Class ambrosia\_web.view.detailsview.DetailsView}
\label{ambrosia_web.view.detailsview.DetailsView::doc}\label{ambrosia_web.view.detailsview.DetailsView:class-ambrosia-web-view-detailsview-detailsview}
Implements a simple view that shows details about the last event that has been selected


\subsubsection{Constructor}
\label{ambrosia_web.view.detailsview.DetailsView:constructor}\index{ambrosia\_web.view.detailsview.DetailsView() (class)}

\begin{fulllineitems}
\phantomsection\label{ambrosia_web.view.detailsview.DetailsView:ambrosia_web.view.detailsview.DetailsView}\pysiglinewithargsret{\strong{class }\code{ambrosia\_web.view.detailsview.}\bfcode{DetailsView}}{\emph{element}}{}~\begin{quote}\begin{description}
\item[{Arguments}] \leavevmode\begin{itemize}
\item {} 
\textbf{element} (\emph{jQuery}) -- the jQuery element the view should be located

\end{itemize}

\end{description}\end{quote}

\end{fulllineitems}



\subsubsection{Methods}
\label{ambrosia_web.view.detailsview.DetailsView:methods}

\paragraph{setup}
\label{ambrosia_web.view.detailsview.DetailsView:setup}\index{ambrosia\_web.view.detailsview.DetailsView.setup() (ambrosia\_web.view.detailsview.DetailsView method)}

\begin{fulllineitems}
\phantomsection\label{ambrosia_web.view.detailsview.DetailsView:ambrosia_web.view.detailsview.DetailsView.setup}\pysiglinewithargsret{\code{ambrosia\_web.view.detailsview.DetailsView.}\bfcode{setup}}{}{}
\end{fulllineitems}


set up the details view


\subsection{Namespace ambrosia\_web.view.entityview}
\label{ambrosia_web.view.entityview:namespace-ambrosia-web-view-entityview}\label{ambrosia_web.view.entityview::doc}

\subsubsection{Constructor}
\label{ambrosia_web.view.entityview:constructor}\index{ambrosia\_web.view.entityview() (class)}

\begin{fulllineitems}
\phantomsection\label{ambrosia_web.view.entityview:ambrosia_web.view.entityview}\pysiglinewithargsret{\strong{class }\code{ambrosia\_web.view.}\bfcode{entityview}}{}{}
\end{fulllineitems}



\subsection{Class ambrosia\_web.view.entityview.EntityView}
\label{ambrosia_web.view.entityview.EntityView:class-ambrosia-web-view-entityview-entityview}\label{ambrosia_web.view.entityview.EntityView::doc}
Implements a simple view that shows details about the selected entity


\subsubsection{Constructor}
\label{ambrosia_web.view.entityview.EntityView:constructor}\index{ambrosia\_web.view.entityview.EntityView() (class)}

\begin{fulllineitems}
\phantomsection\label{ambrosia_web.view.entityview.EntityView:ambrosia_web.view.entityview.EntityView}\pysiglinewithargsret{\strong{class }\code{ambrosia\_web.view.entityview.}\bfcode{EntityView}}{\emph{element}}{}~\begin{quote}\begin{description}
\item[{Arguments}] \leavevmode\begin{itemize}
\item {} 
\textbf{element} (\emph{jQuery}) -- the jQuery element the view should be located

\end{itemize}

\end{description}\end{quote}

\end{fulllineitems}



\subsection{Namespace ambrosia\_web.view.filterview}
\label{ambrosia_web.view.filterview:namespace-ambrosia-web-view-filterview}\label{ambrosia_web.view.filterview::doc}

\subsubsection{Constructor}
\label{ambrosia_web.view.filterview:constructor}\index{ambrosia\_web.view.filterview() (class)}

\begin{fulllineitems}
\phantomsection\label{ambrosia_web.view.filterview:ambrosia_web.view.filterview}\pysiglinewithargsret{\strong{class }\code{ambrosia\_web.view.}\bfcode{filterview}}{}{}
\end{fulllineitems}



\subsection{Class ambrosia\_web.view.filterview.FilterView}
\label{ambrosia_web.view.filterview.FilterView::doc}\label{ambrosia_web.view.filterview.FilterView:class-ambrosia-web-view-filterview-filterview}
Implements a view that allows to view and modify filters


\subsubsection{Constructor}
\label{ambrosia_web.view.filterview.FilterView:constructor}\index{ambrosia\_web.view.filterview.FilterView() (class)}

\begin{fulllineitems}
\phantomsection\label{ambrosia_web.view.filterview.FilterView:ambrosia_web.view.filterview.FilterView}\pysiglinewithargsret{\strong{class }\code{ambrosia\_web.view.filterview.}\bfcode{FilterView}}{}{}
\end{fulllineitems}



\subsubsection{Methods}
\label{ambrosia_web.view.filterview.FilterView:methods}

\paragraph{redraw}
\label{ambrosia_web.view.filterview.FilterView:redraw}\index{ambrosia\_web.view.filterview.FilterView.redraw() (ambrosia\_web.view.filterview.FilterView method)}

\begin{fulllineitems}
\phantomsection\label{ambrosia_web.view.filterview.FilterView:ambrosia_web.view.filterview.FilterView.redraw}\pysiglinewithargsret{\code{ambrosia\_web.view.filterview.FilterView.}\bfcode{redraw}}{}{}
\end{fulllineitems}


redraws the filter view


\paragraph{setup}
\label{ambrosia_web.view.filterview.FilterView:setup}\index{ambrosia\_web.view.filterview.FilterView.setup() (ambrosia\_web.view.filterview.FilterView method)}

\begin{fulllineitems}
\phantomsection\label{ambrosia_web.view.filterview.FilterView:ambrosia_web.view.filterview.FilterView.setup}\pysiglinewithargsret{\code{ambrosia\_web.view.filterview.FilterView.}\bfcode{setup}}{}{}
\end{fulllineitems}


sets up the filterview


\subsection{Namespace ambrosia\_web.view.mainview}
\label{ambrosia_web.view.mainview::doc}\label{ambrosia_web.view.mainview:namespace-ambrosia-web-view-mainview}

\subsubsection{Constructor}
\label{ambrosia_web.view.mainview:constructor}\index{ambrosia\_web.view.mainview() (class)}

\begin{fulllineitems}
\phantomsection\label{ambrosia_web.view.mainview:ambrosia_web.view.mainview}\pysiglinewithargsret{\strong{class }\code{ambrosia\_web.view.}\bfcode{mainview}}{}{}
\end{fulllineitems}



\subsubsection{Attributes}
\label{ambrosia_web.view.mainview:attributes}

\paragraph{EXTRA\_WIDTH}
\label{ambrosia_web.view.mainview:extra-width}\index{EXTRA\_WIDTH (None attribute)}

\begin{fulllineitems}
\phantomsection\label{ambrosia_web.view.mainview:EXTRA_WIDTH}\pysigline{\bfcode{EXTRA\_WIDTH}}
\end{fulllineitems}


the extra horizontal space that should be left after the last event


\paragraph{X\_OFFSET}
\label{ambrosia_web.view.mainview:x-offset}\index{X\_OFFSET (None attribute)}

\begin{fulllineitems}
\phantomsection\label{ambrosia_web.view.mainview:X_OFFSET}\pysigline{\bfcode{X\_OFFSET}}
\end{fulllineitems}


the x offset where events may be drawn


\subsection{Class ambrosia\_web.view.mainview.MainView}
\label{ambrosia_web.view.mainview.MainView::doc}\label{ambrosia_web.view.mainview.MainView:class-ambrosia-web-view-mainview-mainview}
the main view showing all events in a timeline


\subsubsection{Constructor}
\label{ambrosia_web.view.mainview.MainView:constructor}\index{ambrosia\_web.view.mainview.MainView() (class)}

\begin{fulllineitems}
\phantomsection\label{ambrosia_web.view.mainview.MainView:ambrosia_web.view.mainview.MainView}\pysiglinewithargsret{\strong{class }\code{ambrosia\_web.view.mainview.}\bfcode{MainView}}{}{}
\end{fulllineitems}



\subsubsection{Methods}
\label{ambrosia_web.view.mainview.MainView:methods}

\paragraph{getHeight}
\label{ambrosia_web.view.mainview.MainView:getheight}\index{ambrosia\_web.view.mainview.MainView.getHeight() (ambrosia\_web.view.mainview.MainView method)}

\begin{fulllineitems}
\phantomsection\label{ambrosia_web.view.mainview.MainView:ambrosia_web.view.mainview.MainView.getHeight}\pysiglinewithargsret{\code{ambrosia\_web.view.mainview.MainView.}\bfcode{getHeight}}{}{}~\begin{quote}\begin{description}
\item[{Returns number}] \leavevmode
the height

\end{description}\end{quote}

\end{fulllineitems}


get the height of the main view


\paragraph{getWidth}
\label{ambrosia_web.view.mainview.MainView:getwidth}\index{ambrosia\_web.view.mainview.MainView.getWidth() (ambrosia\_web.view.mainview.MainView method)}

\begin{fulllineitems}
\phantomsection\label{ambrosia_web.view.mainview.MainView:ambrosia_web.view.mainview.MainView.getWidth}\pysiglinewithargsret{\code{ambrosia\_web.view.mainview.MainView.}\bfcode{getWidth}}{}{}~\begin{quote}\begin{description}
\item[{Returns number}] \leavevmode
the width

\end{description}\end{quote}

\end{fulllineitems}


get the width of the main view


\paragraph{redraw}
\label{ambrosia_web.view.mainview.MainView:redraw}\index{ambrosia\_web.view.mainview.MainView.redraw() (ambrosia\_web.view.mainview.MainView method)}

\begin{fulllineitems}
\phantomsection\label{ambrosia_web.view.mainview.MainView:ambrosia_web.view.mainview.MainView.redraw}\pysiglinewithargsret{\code{ambrosia\_web.view.mainview.MainView.}\bfcode{redraw}}{}{}
\end{fulllineitems}


redraw the main view


\paragraph{setup}
\label{ambrosia_web.view.mainview.MainView:setup}\index{ambrosia\_web.view.mainview.MainView.setup() (ambrosia\_web.view.mainview.MainView method)}

\begin{fulllineitems}
\phantomsection\label{ambrosia_web.view.mainview.MainView:ambrosia_web.view.mainview.MainView.setup}\pysiglinewithargsret{\code{ambrosia\_web.view.mainview.MainView.}\bfcode{setup}}{}{}
\end{fulllineitems}


set up the main view


\paragraph{setWidth}
\label{ambrosia_web.view.mainview.MainView:setwidth}\index{ambrosia\_web.view.mainview.MainView.setWidth() (ambrosia\_web.view.mainview.MainView method)}

\begin{fulllineitems}
\phantomsection\label{ambrosia_web.view.mainview.MainView:ambrosia_web.view.mainview.MainView.setWidth}\pysiglinewithargsret{\code{ambrosia\_web.view.mainview.MainView.}\bfcode{setWidth}}{\emph{val}}{}~\begin{quote}\begin{description}
\item[{Arguments}] \leavevmode\begin{itemize}
\item {} 
\textbf{val} (\emph{number}) -- the width

\end{itemize}

\end{description}\end{quote}

\end{fulllineitems}


set the width of the main view


\subsection{Overview}
\label{client:overview}
This section gives a short overview of the internal workings of Ambrosia Web. For a detailed description please see the
documentation for the packages.

The function {\hyperref[ambrosia_web:ambrosia_web.init]{\code{ambrosia\_web.init()}}} loads the serialized data (specified after the hash symbol in the URL),
enriches the events and entities ({\hyperref[ambrosia_web.entity:ambrosia_web.entity.enrich]{\code{ambrosia\_web.entity.enrich()}}} and {\hyperref[ambrosia_web.event:ambrosia_web.event.enrich]{\code{ambrosia\_web.event.enrich()}}}) and
resolves all references of the entities (\code{ambrosia\_web.entity.Entity.resolveReferences()}). Afterwards all views
are set up ({\hyperref[ambrosia_web.view.mainview.MainView:ambrosia_web.view.mainview.MainView]{\code{ambrosia\_web.view.mainview.MainView()}}}, {\hyperref[ambrosia_web.view.entityview.EntityView:ambrosia_web.view.entityview.EntityView]{\code{ambrosia\_web.view.entityview.EntityView()}}},
{\hyperref[ambrosia_web.view.detailsview.DetailsView:ambrosia_web.view.detailsview.DetailsView]{\code{ambrosia\_web.view.detailsview.DetailsView()}}}, :js:class:{\color{red}\bfseries{}{}`}ambrosia\_web.view.filterview.FilterView).

The main view shows all events on a timeline. The method {\hyperref[ambrosia_web.view.mainview.MainView:ambrosia_web.view.mainview.MainView.redraw]{\code{ambrosia\_web.view.mainview.MainView.redraw()}}} uses the
following methods to draw the events:
\begin{itemize}
\item {} 
{\hyperref[ambrosia_web.event.Event:ambrosia_web.event.Event.calcVisible]{\code{ambrosia\_web.event.Event.calcVisible()}}}: calculates if the event should be drawn at all (i.e. whether it is
filtered)

\item {} 
{\hyperref[ambrosia_web.event.Event:ambrosia_web.event.Event.calcDimensions]{\code{ambrosia\_web.event.Event.calcDimensions()}}}: calculates where the event should be drawn and how big it should
be

\item {} 
{\hyperref[ambrosia_web.event.Event:ambrosia_web.event.Event.draw]{\code{ambrosia\_web.event.Event.draw()}}}: draws the element.

\end{itemize}

Each of theses methods may call the corresponding methods on child events (e.g. a parent event needs to know about the
positions of the children to decide how big it should be).

Ambrosia defines two types of events:
\begin{itemize}
\item {} 
a {\hyperref[ambrosia_web.event.BlockEvent:ambrosia_web.event.BlockEvent]{\code{ambrosia\_web.event.BlockEvent()}}} is drawn as a block in the main view

\item {} 
a {\hyperref[ambrosia_web.event.LineEvent:ambrosia_web.event.LineEvent]{\code{ambrosia\_web.event.LineEvent()}}} is drawn as a line across the main view (children are not drawn)

\end{itemize}

In order for a block event to decide where it should be drawn the {\hyperref[ambrosia_web.layout.BlockLayoutManager:ambrosia_web.layout.BlockLayoutManager]{\code{ambrosia\_web.layout.BlockLayoutManager()}}} is
used. This class remembers the relevant block events that have already been drawn and allows an event to find a position
where enough free space is available. the block layout manager is used on the top level and to position children of an
event. Each event with children creates a new block layout manager.

Events and entities can be selected (see {\hyperref[ambrosia_web.event.Event:ambrosia_web.event.Event.select]{\code{ambrosia\_web.event.Event.select()}}}
{\hyperref[ambrosia_web.event.Event:ambrosia_web.event.Event.selectAdd]{\code{ambrosia\_web.event.Event.selectAdd()}}}, {\hyperref[ambrosia_web.event.Event:ambrosia_web.event.Event.unselect]{\code{ambrosia\_web.event.Event.unselect()}}}.
{\hyperref[ambrosia_web.event:ambrosia_web.event.clearSelect]{\code{ambrosia\_web.event.clearSelect()}}}, {\hyperref[ambrosia_web.entity.Entity:ambrosia_web.entity.Entity.select]{\code{ambrosia\_web.entity.Entity.select()}}}). Any part of the application
may select an entity or an event and all parts of the application may register to select and unselect events (see
\code{ambrosia\_web.event.onSelectHandler}, \code{ambrosia\_web.event.onUnSelectHandler},
\code{ambrosia\_web.entity.onSelectHandler}). Multiple events may be selected but only one entity can be selected.

Each event class specifies filters. For an event all filters have to match for the event to be shown. General filters
are applied to all events (see {\hyperref[ambrosia_web.event.Event:ambrosia_web.event.Event]{\code{ambrosia\_web.event.Event()}}}). Those the rules for these filters follow a
specific syntax (see {\hyperref[ambrosia_web.filter.Filter:ambrosia_web.filter.Filter]{\code{ambrosia\_web.filter.Filter()}}}).


\section{Ambrosia Server Documentation}
\label{server:ambrosia-server-documentation}\label{server::doc}
==


\subsection{ambrosia package}
\label{ambrosia::doc}\label{ambrosia:ambrosia-package}

\subsubsection{Subpackages}
\label{ambrosia:subpackages}

\paragraph{ambrosia.clocks package}
\label{ambrosia.clocks:ambrosia-clocks-package}\label{ambrosia.clocks::doc}

\subparagraph{Module contents}
\label{ambrosia.clocks:module-ambrosia.clocks}\label{ambrosia.clocks:module-contents}\index{ambrosia.clocks (module)}\index{ClockSyncer (class in ambrosia.clocks)}

\begin{fulllineitems}
\phantomsection\label{ambrosia.clocks:ambrosia.clocks.ClockSyncer}\pysiglinewithargsret{\strong{class }\code{ambrosia.clocks.}\bfcode{ClockSyncer}}{\emph{context}}{}
Bases: \code{object}

Used to synchronize all events.

This class manages the \textbf{translate\_table}. This Array has the following structure:

\begin{Verbatim}[commandchars=\\\{\}]
\PYG{p}{[}
    \PYG{p}{(}\PYG{n}{time}\PYG{p}{,} \PYG{n}{error}\PYG{p}{)}
\PYG{p}{]}
\end{Verbatim}

where
* \emph{time} is a timestamp (datetime.datetime) when the emulator time has changed (in \textbf{emulator time}) and
* \emph{error} is the datetime.timedelta of how much the emulator time is in the future

The entries have to be sorted by \emph{time}.

\begin{notice}{warning}{Warning:}
This class assumes that when the emulator is started, the times are synchronized.
\end{notice}

\begin{notice}{warning}{Warning:}
This class assumes that the emulator clock is always turned ahead (and never back). Currently this is the case
since ANANAS tries to trigger behaviour that occur when the sample has been installed for a while.

This also poses a theoretical issue e.g. if the emulator time is 17:00 at boot and at 17:02 the clock is turned
back to 17:00. An event occurring at 17:01 can either have happened at 17:01 or 17:03.
\end{notice}

\begin{notice}{warning}{Warning:}
This class assumes that all timestamps have the same time zone (local time).
\end{notice}
\begin{quote}\begin{description}
\item[{Parameters}] \leavevmode
\textbf{context} (\emph{ambrosia\_web.context.AmbrosiaContext}) -- The current context.

\end{description}\end{quote}
\index{emu\_time() (ambrosia.clocks.ClockSyncer method)}

\begin{fulllineitems}
\phantomsection\label{ambrosia.clocks:ambrosia.clocks.ClockSyncer.emu_time}\pysiglinewithargsret{\bfcode{emu\_time}}{\emph{t}}{}
Calculate host time from a given emulator timestamp.

The method goes through all entries and finds the first entry where the given emulator timestamp is greater than
the \emph{time}. This means that the timestamp occurs after this emulator clock change. If no such entry is found,
the emulator clock is assumed to be in sync with the host clock.

\end{fulllineitems}


\end{fulllineitems}



\paragraph{ambrosia.config package}
\label{ambrosia.config:ambrosia-config-package}\label{ambrosia.config::doc}

\subparagraph{Module contents}
\label{ambrosia.config:module-ambrosia.config}\label{ambrosia.config:module-contents}\index{ambrosia.config (module)}\index{Config (class in ambrosia.config)}

\begin{fulllineitems}
\phantomsection\label{ambrosia.config:ambrosia.config.Config}\pysiglinewithargsret{\strong{class }\code{ambrosia.config.}\bfcode{Config}}{\emph{configfile}}{}
Bases: \code{ConfigParser.SafeConfigParser}

Allows simple access to the configuration file (currently not used or implemented)

\end{fulllineitems}



\paragraph{ambrosia.context package}
\label{ambrosia.context::doc}\label{ambrosia.context:ambrosia-context-package}

\subparagraph{Module contents}
\label{ambrosia.context:module-ambrosia.context}\label{ambrosia.context:module-contents}\index{ambrosia.context (module)}\index{AmbrosiaContext (class in ambrosia.context)}

\begin{fulllineitems}
\phantomsection\label{ambrosia.context:ambrosia.context.AmbrosiaContext}\pysiglinewithargsret{\strong{class }\code{ambrosia.context.}\bfcode{AmbrosiaContext}}{\emph{configfile}}{}
Bases: \code{object}

Objects of this class hold all relevant information for \textbf{one} run of Ambrosia:
\begin{itemize}
\item {} 
\emph{config} (\code{ambrosia\_web.config.Config}): the configuration

\item {} 
\emph{db} (\code{ambrosia\_web.db.AmbrosiaDb}): the database (currently not used)

\item {} 
\emph{analysis} (\code{ambrosia\_web.model.Analysis}): the object containing the Analysis results.

\item {} 
\emph{clock\_syncer} (\code{ambrosia\_web.clocks.ClockSyncer}): used to syncronize clocks (emulator \textless{}-\textgreater{} host)

\item {} 
\emph{plugin\_manager} (\code{ambrosia\_web.plugins.PluginManager}): the object holding information about the Ambrosia
plugins

\end{itemize}
\begin{quote}\begin{description}
\item[{Parameters}] \leavevmode
\textbf{configfile} (\emph{str}) -- path to configuration file

\end{description}\end{quote}

\end{fulllineitems}



\paragraph{ambrosia.db package}
\label{ambrosia.db:ambrosia-db-package}\label{ambrosia.db::doc}

\subparagraph{Module contents}
\label{ambrosia.db:module-ambrosia.db}\label{ambrosia.db:module-contents}\index{ambrosia.db (module)}\index{AmbrosiaDb (class in ambrosia.db)}

\begin{fulllineitems}
\phantomsection\label{ambrosia.db:ambrosia.db.AmbrosiaDb}\pysiglinewithargsret{\strong{class }\code{ambrosia.db.}\bfcode{AmbrosiaDb}}{\emph{context}}{}
Bases: \code{object}

For future use: persistently store objects in Memory using ZODB.

Currently the memory-footprint of Ambrosia is reasonable. However, Ambrosia is designed to be stored in ZODB.
This database allows transparent storage to disk if memory becomes scarce. ZODB also uses certain data structures
optimized (e.g. BTree module). Ambrosia already uses these data structures. The following classes are already
designed to be stored in ZODB:
\begin{itemize}
\item {} 
\code{ambrosia\_web.model.Analysis}

\item {} 
\code{ambrosia\_web.model.Entity}

\item {} 
\code{ambrosia\_web.model.Event}

\end{itemize}
\begin{quote}\begin{description}
\item[{Parameters}] \leavevmode
\textbf{context} (\emph{ambrosia\_web.context.AmbrosiaContext}) -- The current context.

\end{description}\end{quote}

\end{fulllineitems}



\paragraph{ambrosia.model package}
\label{ambrosia.model:ambrosia-model-package}\label{ambrosia.model::doc}

\subparagraph{Submodules}
\label{ambrosia.model:submodules}

\subparagraph{ambrosia.model.entities module}
\label{ambrosia.model:module-ambrosia.model.entities}\label{ambrosia.model:ambrosia-model-entities-module}\index{ambrosia.model.entities (module)}\index{App (class in ambrosia.model.entities)}

\begin{fulllineitems}
\phantomsection\label{ambrosia.model:ambrosia.model.entities.App}\pysiglinewithargsret{\strong{class }\code{ambrosia.model.entities.}\bfcode{App}}{\emph{context}, \emph{package}}{}
Bases: {\hyperref[ambrosia.model:ambrosia.model.Entity]{\code{ambrosia.model.Entity}}}
\index{find() (ambrosia.model.entities.App static method)}

\begin{fulllineitems}
\phantomsection\label{ambrosia.model:ambrosia.model.entities.App.find}\pysiglinewithargsret{\strong{static }\bfcode{find}}{\emph{context}, \emph{entities}, \emph{identifier\_btree}, \emph{package}}{}
\end{fulllineitems}

\index{get\_serializeable\_properties() (ambrosia.model.entities.App method)}

\begin{fulllineitems}
\phantomsection\label{ambrosia.model:ambrosia.model.entities.App.get_serializeable_properties}\pysiglinewithargsret{\bfcode{get\_serializeable\_properties}}{}{}
\end{fulllineitems}


\end{fulllineitems}

\index{File (class in ambrosia.model.entities)}

\begin{fulllineitems}
\phantomsection\label{ambrosia.model:ambrosia.model.entities.File}\pysiglinewithargsret{\strong{class }\code{ambrosia.model.entities.}\bfcode{File}}{\emph{context}, \emph{abspath}}{}
Bases: {\hyperref[ambrosia.model:ambrosia.model.Entity]{\code{ambrosia.model.Entity}}}

Represents file (existing or not) on the emulator.
\begin{quote}\begin{description}
\item[{Parameters}] \leavevmode\begin{itemize}
\item {} 
\textbf{context} (\emph{ambrosia\_web.context.AmbrosiaContext}) -- the current context

\item {} 
\textbf{abspath} (\emph{str}) -- the absolute path of the file

\end{itemize}

\end{description}\end{quote}
\index{find() (ambrosia.model.entities.File static method)}

\begin{fulllineitems}
\phantomsection\label{ambrosia.model:ambrosia.model.entities.File.find}\pysiglinewithargsret{\strong{static }\bfcode{find}}{\emph{context}, \emph{entities}, \emph{identifier\_btree}, \emph{abspath}}{}
\end{fulllineitems}

\index{get\_serializeable\_properties() (ambrosia.model.entities.File method)}

\begin{fulllineitems}
\phantomsection\label{ambrosia.model:ambrosia.model.entities.File.get_serializeable_properties}\pysiglinewithargsret{\bfcode{get\_serializeable\_properties}}{}{}
\end{fulllineitems}

\index{matches\_entity() (ambrosia.model.entities.File method)}

\begin{fulllineitems}
\phantomsection\label{ambrosia.model:ambrosia.model.entities.File.matches_entity}\pysiglinewithargsret{\bfcode{matches\_entity}}{\emph{abspath}}{}
\end{fulllineitems}

\index{unknown() (ambrosia.model.entities.File static method)}

\begin{fulllineitems}
\phantomsection\label{ambrosia.model:ambrosia.model.entities.File.unknown}\pysiglinewithargsret{\strong{static }\bfcode{unknown}}{\emph{context}}{}
Get the file representing unknonw files
\begin{quote}\begin{description}
\item[{Parameters}] \leavevmode
\textbf{context} (\emph{ambrosia\_web.context.AmbrosiaContext}) -- the current context

\end{description}\end{quote}

\end{fulllineitems}


\end{fulllineitems}

\index{ServerEndpoint (class in ambrosia.model.entities)}

\begin{fulllineitems}
\phantomsection\label{ambrosia.model:ambrosia.model.entities.ServerEndpoint}\pysiglinewithargsret{\strong{class }\code{ambrosia.model.entities.}\bfcode{ServerEndpoint}}{\emph{context}, \emph{protocol}, \emph{address}, \emph{port=None}}{}
Bases: {\hyperref[ambrosia.model:ambrosia.model.Entity]{\code{ambrosia.model.Entity}}}

Represents a server endpoint i.e. a server and port.
\begin{quote}\begin{description}
\item[{Parameters}] \leavevmode\begin{itemize}
\item {} 
\textbf{context} (\emph{ambrosia\_web.context.AmbrosiaContext}) -- the current context

\item {} 
\textbf{protocol} (\emph{str}) -- the network protocol used (e.g. TCP)

\end{itemize}

\end{description}\end{quote}
\index{find() (ambrosia.model.entities.ServerEndpoint static method)}

\begin{fulllineitems}
\phantomsection\label{ambrosia.model:ambrosia.model.entities.ServerEndpoint.find}\pysiglinewithargsret{\strong{static }\bfcode{find}}{\emph{context}, \emph{entities}, \emph{identifier\_btree}, \emph{protocol}, \emph{address}, \emph{port}}{}
\end{fulllineitems}

\index{get\_serializeable\_properties() (ambrosia.model.entities.ServerEndpoint method)}

\begin{fulllineitems}
\phantomsection\label{ambrosia.model:ambrosia.model.entities.ServerEndpoint.get_serializeable_properties}\pysiglinewithargsret{\bfcode{get\_serializeable\_properties}}{}{}
\end{fulllineitems}


\end{fulllineitems}

\index{Task (class in ambrosia.model.entities)}

\begin{fulllineitems}
\phantomsection\label{ambrosia.model:ambrosia.model.entities.Task}\pysiglinewithargsret{\strong{class }\code{ambrosia.model.entities.}\bfcode{Task}}{\emph{context}, \emph{pid}, \emph{start\_ts}, \emph{end\_ts}}{}
Bases: {\hyperref[ambrosia.model:ambrosia.model.Entity]{\code{ambrosia.model.Entity}}}

Represents a process or thread running on the emulator.
\begin{quote}\begin{description}
\item[{Parameters}] \leavevmode\begin{itemize}
\item {} 
\textbf{context} (\emph{ambrosia\_web.context.AmbrosiaContext}) -- the current context

\item {} 
\textbf{pid} (\emph{int}) -- the PID/TID of the task

\item {} 
\textbf{start\_ts} (\emph{datetime.datetime}) -- the timestamp the task started or \emph{None} if unknown

\item {} 
\textbf{end\_ts} (\emph{datetime.datetime}) -- the timestamp the task ended or \emph{None} if unknown

\end{itemize}

\end{description}\end{quote}
\index{find() (ambrosia.model.entities.Task static method)}

\begin{fulllineitems}
\phantomsection\label{ambrosia.model:ambrosia.model.entities.Task.find}\pysiglinewithargsret{\strong{static }\bfcode{find}}{\emph{context}, \emph{entities}, \emph{identifier\_btree}, \emph{pid}, \emph{start\_ts}, \emph{end\_ts}}{}
\end{fulllineitems}

\index{get\_serializeable\_properties() (ambrosia.model.entities.Task method)}

\begin{fulllineitems}
\phantomsection\label{ambrosia.model:ambrosia.model.entities.Task.get_serializeable_properties}\pysiglinewithargsret{\bfcode{get\_serializeable\_properties}}{}{}
\end{fulllineitems}

\index{is\_process (ambrosia.model.entities.Task attribute)}

\begin{fulllineitems}
\phantomsection\label{ambrosia.model:ambrosia.model.entities.Task.is_process}\pysigline{\bfcode{is\_process}}
whether this task is a process rather than a thread

\end{fulllineitems}


\end{fulllineitems}



\subparagraph{Module contents}
\label{ambrosia.model:module-ambrosia.model}\label{ambrosia.model:module-contents}\index{ambrosia.model (module)}\index{Analysis (class in ambrosia.model)}

\begin{fulllineitems}
\phantomsection\label{ambrosia.model:ambrosia.model.Analysis}\pysigline{\strong{class }\code{ambrosia.model.}\bfcode{Analysis}}
Bases: \code{persistent.Persistent}

An Analysis object (and the referenced objects) stores all information the Ambrosia analysis found out.

Analysis also manages all (top-level) Events and Entities (see \code{ambrosia\_web.model.Event},
\code{ambrosia\_web.model.Entity}) and tries to optimize for searching performance.
\index{add\_entity() (ambrosia.model.Analysis method)}

\begin{fulllineitems}
\phantomsection\label{ambrosia.model:ambrosia.model.Analysis.add_entity}\pysiglinewithargsret{\bfcode{add\_entity}}{\emph{context}, \emph{cls}, \emph{*args}}{}
Add an entity (alias for {\hyperref[ambrosia.model:ambrosia.model.Analysis.get_entity]{\code{Analysis.get\_entity()}}})

\end{fulllineitems}

\index{add\_event() (ambrosia.model.Analysis method)}

\begin{fulllineitems}
\phantomsection\label{ambrosia.model:ambrosia.model.Analysis.add_event}\pysiglinewithargsret{\bfcode{add\_event}}{\emph{evt}}{}
Add event and generate indices
\begin{quote}\begin{description}
\item[{Parameters}] \leavevmode
\textbf{evt} (\emph{Event}) -- the event to add

\end{description}\end{quote}

\begin{notice}{warning}{Warning:}
The indexed attributes of an event may not be altered after the event has been added (otherwise the indices
are out of date). This means that only static values may be indexed.
\end{notice}

\end{fulllineitems}

\index{adjust\_times() (ambrosia.model.Analysis method)}

\begin{fulllineitems}
\phantomsection\label{ambrosia.model:ambrosia.model.Analysis.adjust_times}\pysiglinewithargsret{\bfcode{adjust\_times}}{\emph{context}}{}
Goes through all events and calls adjust\_times on all Events

\end{fulllineitems}

\index{del\_event() (ambrosia.model.Analysis method)}

\begin{fulllineitems}
\phantomsection\label{ambrosia.model:ambrosia.model.Analysis.del_event}\pysiglinewithargsret{\bfcode{del\_event}}{\emph{evt}}{}
Delete event and update indices
\begin{quote}\begin{description}
\item[{Parameters}] \leavevmode
\textbf{evt} (\emph{Event}) -- the event to remove

\end{description}\end{quote}

\end{fulllineitems}

\index{get\_entity() (ambrosia.model.Analysis method)}

\begin{fulllineitems}
\phantomsection\label{ambrosia.model:ambrosia.model.Analysis.get_entity}\pysiglinewithargsret{\bfcode{get\_entity}}{\emph{context}, \emph{cls}, \emph{*args}}{}
Search for a specific entity, if it does not exist, create a new one
\begin{quote}\begin{description}
\item[{Parameters}] \leavevmode\begin{itemize}
\item {} 
\textbf{context} (\emph{ambrosia\_web.context.AmbrosiaContext}) -- the current context

\item {} 
\textbf{cls} (\emph{class}) -- the class of the entity we are looking for

\item {} 
\textbf{*args} -- the arguments that would construct an entity

\end{itemize}

\end{description}\end{quote}

This method uses {\hyperref[ambrosia.model:ambrosia.model.Entity.find]{\code{Entity.find()}}} (of the specific entity class) to search for entities. This method
receives a List of all known entitites of that class as well as a \code{BTrees.OOBTree.BTree} also containing
all entities (indexed by their \emph{primary\_identifier} to allow more efficient searching). Moreover this method
relieves the *args argument.

The *args argument contains all information that identifies a certain entity. This could be e.g. the IP address
and the port of a server. Those values are passed to the find method. If the server is already known, the entity
representing it is returned. If no such server entity exist a new one is created using those two parameters.

This behaviour makes sure that multiple event referencing the same entity all have references to the exact same
entity in memory.

\end{fulllineitems}

\index{iter\_all\_events() (ambrosia.model.Analysis method)}

\begin{fulllineitems}
\phantomsection\label{ambrosia.model:ambrosia.model.Analysis.iter_all_events}\pysiglinewithargsret{\bfcode{iter\_all\_events}}{\emph{context}, \emph{key=None}, \emph{min\_value=None}, \emph{max\_value=None}, \emph{value=None}}{}
\end{fulllineitems}

\index{iter\_entities() (ambrosia.model.Analysis method)}

\begin{fulllineitems}
\phantomsection\label{ambrosia.model:ambrosia.model.Analysis.iter_entities}\pysiglinewithargsret{\bfcode{iter\_entities}}{\emph{context}, \emph{cls}}{}
iterate all known entities of a specific class.
\begin{quote}\begin{description}
\item[{Parameters}] \leavevmode\begin{itemize}
\item {} 
\textbf{context} (\emph{ambrosia\_web.context.AmbrosiaContext}) -- the current context

\item {} 
\textbf{cls} (\emph{class}) -- the class of the entity we are looking for

\end{itemize}

\end{description}\end{quote}

\end{fulllineitems}

\index{iter\_events() (ambrosia.model.Analysis method)}

\begin{fulllineitems}
\phantomsection\label{ambrosia.model:ambrosia.model.Analysis.iter_events}\pysiglinewithargsret{\bfcode{iter\_events}}{\emph{context}, \emph{cls=None}, \emph{key=None}, \emph{min\_value=None}, \emph{max\_value=None}, \emph{value=None}}{}
Iterates over all events matching specific conditions in an efficient manner.
\begin{quote}\begin{description}
\item[{Parameters}] \leavevmode\begin{itemize}
\item {} 
\textbf{context} (\emph{ambrosia\_web.context.AmbrosiaContext}) -- the current context

\item {} 
\textbf{cls} (\emph{class}) -- the class of the events we are looking for

\item {} 
\textbf{key} -- the key we are searching for

\item {} 
\textbf{min\_value} -- the minimum value

\item {} 
\textbf{max\_value} -- the maximum value

\item {} 
\textbf{value} -- the specific value (to search for exactly one value)

\end{itemize}

\end{description}\end{quote}

This method uses an internal indizes to efficiently select specific events. Each event class defines attributes
that should be indexed ({\hyperref[ambrosia.model:ambrosia.model.Event]{\code{Event}}}.indices). This class makes sure that those attributes can be searched for
very fast.

The method accepts the following combinations of argument:
* nothing: return all events
* \emph{cls}: return all events of a specific class (inefficient)
* \emph{cls}, \emph{key}, \emph{min\_value} and/or \emph{max\_value}: search for all events of a specific class where the attribute
\emph{key} is within the defined value constraints
* \emph{cls}, \emph{key}, \emph{value}: search all events of a specific class where the attribute \emph{key} has the value \emph{value}

\end{fulllineitems}

\index{to\_serializeable() (ambrosia.model.Analysis method)}

\begin{fulllineitems}
\phantomsection\label{ambrosia.model:ambrosia.model.Analysis.to_serializeable}\pysiglinewithargsret{\bfcode{to\_serializeable}}{}{}
Returns all results in a serializable form

\end{fulllineitems}


\end{fulllineitems}

\index{Entity (class in ambrosia.model)}

\begin{fulllineitems}
\phantomsection\label{ambrosia.model:ambrosia.model.Entity}\pysiglinewithargsret{\strong{class }\code{ambrosia.model.}\bfcode{Entity}}{\emph{primary\_identifier}}{}
Bases: \code{persistent.Persistent}

An Entity represents a static element without a timestamp e.g. a file or a server.
\begin{quote}\begin{description}
\item[{Parameters}] \leavevmode
\textbf{primary\_identifier} (\emph{str}) -- A identifier that identifies the entity. This does not have to be unique
(e.g. PID).

\end{description}\end{quote}
\index{find() (ambrosia.model.Entity static method)}

\begin{fulllineitems}
\phantomsection\label{ambrosia.model:ambrosia.model.Entity.find}\pysiglinewithargsret{\strong{static }\bfcode{find}}{\emph{context}, \emph{entities}, \emph{identifier\_btree}, \emph{*args}}{}
Should find and return an entity based on the *args. Must be implemented by subclass.
\begin{quote}\begin{description}
\item[{Parameters}] \leavevmode\begin{itemize}
\item {} 
\textbf{entities} (\emph{list}) -- all entities known

\item {} 
\textbf{identifier\_btree} (\emph{BTrees.OOBTree.BTree}) -- a binary tree where the keys are the primary identifier and the
values are a list containg the matching entity.

\item {} 
\textbf{*args} -- the arguments identifying the entity. Must be identical to the constructor parameters.

\end{itemize}

\end{description}\end{quote}

\end{fulllineitems}

\index{get\_serializeable\_properties() (ambrosia.model.Entity method)}

\begin{fulllineitems}
\phantomsection\label{ambrosia.model:ambrosia.model.Entity.get_serializeable_properties}\pysiglinewithargsret{\bfcode{get\_serializeable\_properties}}{}{}
Should return all information relevant about the specific entity. Must be implemented by subclass.

\end{fulllineitems}

\index{primary\_identifier (ambrosia.model.Entity attribute)}

\begin{fulllineitems}
\phantomsection\label{ambrosia.model:ambrosia.model.Entity.primary_identifier}\pysigline{\bfcode{primary\_identifier}}
Returns the primary identifier for the entity.

\end{fulllineitems}

\index{primary\_key (ambrosia.model.Entity attribute)}

\begin{fulllineitems}
\phantomsection\label{ambrosia.model:ambrosia.model.Entity.primary_key}\pysigline{\bfcode{primary\_key}\strong{ = None}}
A generated unique key

\end{fulllineitems}

\index{to\_serializeable() (ambrosia.model.Entity method)}

\begin{fulllineitems}
\phantomsection\label{ambrosia.model:ambrosia.model.Entity.to_serializeable}\pysiglinewithargsret{\bfcode{to\_serializeable}}{}{}
Returns a dict containing all relevant information about the entity.

\end{fulllineitems}


\end{fulllineitems}

\index{Event (class in ambrosia.model)}

\begin{fulllineitems}
\phantomsection\label{ambrosia.model:ambrosia.model.Event}\pysiglinewithargsret{\strong{class }\code{ambrosia.model.}\bfcode{Event}}{\emph{start\_ts=None}, \emph{end\_ts=None}}{}
Bases: \code{persistent.Persistent}

Event represents any event with a start-time and/or end-time
\begin{quote}\begin{description}
\item[{Parameters}] \leavevmode\begin{itemize}
\item {} 
\textbf{start\_ts} (\emph{datetime.datetime}) -- the time the event began

\item {} 
\textbf{end\_ts} (\emph{datetime.datetime}) -- the time the event ended

\end{itemize}

\end{description}\end{quote}
\index{add\_child() (ambrosia.model.Event method)}

\begin{fulllineitems}
\phantomsection\label{ambrosia.model:ambrosia.model.Event.add_child}\pysiglinewithargsret{\bfcode{add\_child}}{\emph{c}}{}
Add child to this event. Also checks whether new child already has a parent (this is not allowed in a tree
structure) and updates timestamps.

\end{fulllineitems}

\index{adjust\_times() (ambrosia.model.Event method)}

\begin{fulllineitems}
\phantomsection\label{ambrosia.model:ambrosia.model.Event.adjust_times}\pysiglinewithargsret{\bfcode{adjust\_times}}{\emph{context}}{}
Adjust times (e.g. emulator time -\textgreater{} system time)
\begin{quote}\begin{description}
\item[{Parameters}] \leavevmode
\textbf{context} (\emph{ambrosia\_web.context.AmbrosiaContext}) -- the current context

\end{description}\end{quote}

\end{fulllineitems}

\index{children (ambrosia.model.Event attribute)}

\begin{fulllineitems}
\phantomsection\label{ambrosia.model:ambrosia.model.Event.children}\pysigline{\bfcode{children}}
Iterates over all children.

\end{fulllineitems}

\index{cmp\_by\_time() (ambrosia.model.Event method)}

\begin{fulllineitems}
\phantomsection\label{ambrosia.model:ambrosia.model.Event.cmp_by_time}\pysiglinewithargsret{\bfcode{cmp\_by\_time}}{\emph{other}}{}
Compares two events by start timestamp
\begin{quote}\begin{description}
\item[{Parameters}] \leavevmode
\textbf{other} (\emph{Event}) -- the other event

\end{description}\end{quote}

\end{fulllineitems}

\index{end\_ts (ambrosia.model.Event attribute)}

\begin{fulllineitems}
\phantomsection\label{ambrosia.model:ambrosia.model.Event.end_ts}\pysigline{\bfcode{end\_ts}}
The end timestamp if set else the start timestamp (assuming that start timestamp = end timestamp)
\begin{quote}\begin{description}
\item[{Returns}] \leavevmode
The end timestamp

\end{description}\end{quote}

\end{fulllineitems}

\index{get\_serializeable\_properties() (ambrosia.model.Event method)}

\begin{fulllineitems}
\phantomsection\label{ambrosia.model:ambrosia.model.Event.get_serializeable_properties}\pysiglinewithargsret{\bfcode{get\_serializeable\_properties}}{}{}
This method is used for serialisation, has to be implemented by the subclass. Should return a dict with all
important information about the event.

\end{fulllineitems}

\index{indices (ambrosia.model.Event attribute)}

\begin{fulllineitems}
\phantomsection\label{ambrosia.model:ambrosia.model.Event.indices}\pysigline{\bfcode{indices}\strong{ = set({[}{]})}}~\begin{quote}

This set contains all attributes that can be searched for; these attributes MUST NOT be CHANGED after the event has
been added
\end{quote}

\end{fulllineitems}

\index{sort() (ambrosia.model.Event method)}

\begin{fulllineitems}
\phantomsection\label{ambrosia.model:ambrosia.model.Event.sort}\pysiglinewithargsret{\bfcode{sort}}{}{}
Sort events by start timestamp

\end{fulllineitems}

\index{start\_ts (ambrosia.model.Event attribute)}

\begin{fulllineitems}
\phantomsection\label{ambrosia.model:ambrosia.model.Event.start_ts}\pysigline{\bfcode{start\_ts}}
The start timestamp if set else the end timestamp (assuming that start timestamp = end timestamp)
\begin{quote}\begin{description}
\item[{Returns}] \leavevmode
The start timestamp

\end{description}\end{quote}

\end{fulllineitems}

\index{to\_serializeable() (ambrosia.model.Event method)}

\begin{fulllineitems}
\phantomsection\label{ambrosia.model:ambrosia.model.Event.to_serializeable}\pysiglinewithargsret{\bfcode{to\_serializeable}}{}{}
Returns a dict that can be used for serialization.

The primary keys of entities this entity refers to (e.g. parent process) are stored in the attribute
``references''. This way any entity only has to be transmitted once, when the entity is referenced only the
primary key is used.

\end{fulllineitems}


\end{fulllineitems}



\paragraph{ambrosia.plugins package}
\label{ambrosia.plugins:ambrosia-plugins-package}\label{ambrosia.plugins::doc}

\subparagraph{Module contents}
\label{ambrosia.plugins:module-contents}\label{ambrosia.plugins:module-ambrosia.plugins}\index{ambrosia.plugins (module)}\index{PluginInfoTop (class in ambrosia.plugins)}

\begin{fulllineitems}
\phantomsection\label{ambrosia.plugins:ambrosia.plugins.PluginInfoTop}\pysigline{\strong{class }\code{ambrosia.plugins.}\bfcode{PluginInfoTop}}
Bases: \code{object}

The base class to all PluginInfo classes. Every plugin must define a class named \emph{PluginInfo} in the base module
of the plugin.
\index{correlators() (ambrosia.plugins.PluginInfoTop static method)}

\begin{fulllineitems}
\phantomsection\label{ambrosia.plugins:ambrosia.plugins.PluginInfoTop.correlators}\pysiglinewithargsret{\strong{static }\bfcode{correlators}}{}{}
Should return a list with tuples containing a \code{ambrosia\_web.Correlator} and the priority (int)

\end{fulllineitems}

\index{parsers() (ambrosia.plugins.PluginInfoTop static method)}

\begin{fulllineitems}
\phantomsection\label{ambrosia.plugins:ambrosia.plugins.PluginInfoTop.parsers}\pysiglinewithargsret{\strong{static }\bfcode{parsers}}{}{}
Should return a list with all defined \code{ambrosia\_web.ResultParser} classes.

\end{fulllineitems}


\end{fulllineitems}

\index{PluginManager (class in ambrosia.plugins)}

\begin{fulllineitems}
\phantomsection\label{ambrosia.plugins:ambrosia.plugins.PluginManager}\pysigline{\strong{class }\code{ambrosia.plugins.}\bfcode{PluginManager}}
Bases: \code{object}

Manages all Ambrosia plugins
\index{correlators() (ambrosia.plugins.PluginManager method)}

\begin{fulllineitems}
\phantomsection\label{ambrosia.plugins:ambrosia.plugins.PluginManager.correlators}\pysiglinewithargsret{\bfcode{correlators}}{}{}
Iterate all correlators (sorted by priority)

\end{fulllineitems}

\index{find() (ambrosia.plugins.PluginManager method)}

\begin{fulllineitems}
\phantomsection\label{ambrosia.plugins:ambrosia.plugins.PluginManager.find}\pysiglinewithargsret{\bfcode{find}}{}{}
Finds all plugins and gathers information about them.

\end{fulllineitems}

\index{parsers() (ambrosia.plugins.PluginManager method)}

\begin{fulllineitems}
\phantomsection\label{ambrosia.plugins:ambrosia.plugins.PluginManager.parsers}\pysiglinewithargsret{\bfcode{parsers}}{}{}
Returs a set with all parsers

\end{fulllineitems}


\end{fulllineitems}



\paragraph{ambrosia.util package}
\label{ambrosia.util::doc}\label{ambrosia.util:ambrosia-util-package}

\subparagraph{Submodules}
\label{ambrosia.util:submodules}

\subparagraph{ambrosia.util.log module}
\label{ambrosia.util:module-ambrosia.util.log}\label{ambrosia.util:ambrosia-util-log-module}\index{ambrosia.util.log (module)}\index{AmbrosiaFormater (class in ambrosia.util.log)}

\begin{fulllineitems}
\phantomsection\label{ambrosia.util:ambrosia.util.log.AmbrosiaFormater}\pysiglinewithargsret{\strong{class }\code{ambrosia.util.log.}\bfcode{AmbrosiaFormater}}{\emph{use\_colors}}{}
Bases: \code{logging.Formatter}

A custom log formatter that can use colors
\index{color\_mapping (ambrosia.util.log.AmbrosiaFormater attribute)}

\begin{fulllineitems}
\phantomsection\label{ambrosia.util:ambrosia.util.log.AmbrosiaFormater.color_mapping}\pysigline{\bfcode{color\_mapping}\strong{ = \{`INFO': `1;35', `CRITICAL': `1;31', `WARN': `1;33', `WARNING': `1;33', `ERROR': `1;31', `DEBUG': `1;32'\}}}
\end{fulllineitems}

\index{format() (ambrosia.util.log.AmbrosiaFormater method)}

\begin{fulllineitems}
\phantomsection\label{ambrosia.util:ambrosia.util.log.AmbrosiaFormater.format}\pysiglinewithargsret{\bfcode{format}}{\emph{record}}{}
\end{fulllineitems}


\end{fulllineitems}

\index{init\_logging() (in module ambrosia.util.log)}

\begin{fulllineitems}
\phantomsection\label{ambrosia.util:ambrosia.util.log.init_logging}\pysiglinewithargsret{\code{ambrosia.util.log.}\bfcode{init\_logging}}{\emph{log\_level}}{}
Initialize logging to stderr
\begin{quote}\begin{description}
\item[{Parameters}] \leavevmode
\textbf{log\_level} (\emph{str}) -- the minimum log level

\end{description}\end{quote}

\end{fulllineitems}



\subparagraph{Module contents}
\label{ambrosia.util:module-contents}\label{ambrosia.util:module-ambrosia.util}\index{ambrosia.util (module)}\index{SerializationError}

\begin{fulllineitems}
\phantomsection\label{ambrosia.util:ambrosia.util.SerializationError}\pysigline{\strong{exception }\code{ambrosia.util.}\bfcode{SerializationError}}
Bases: \code{exceptions.Exception}

Indicates that something went wrong during serialization

\end{fulllineitems}

\index{classname() (in module ambrosia.util)}

\begin{fulllineitems}
\phantomsection\label{ambrosia.util:ambrosia.util.classname}\pysiglinewithargsret{\code{ambrosia.util.}\bfcode{classname}}{\emph{cls}}{}
Returns the full class name of a class

\end{fulllineitems}

\index{get\_class() (in module ambrosia.util)}

\begin{fulllineitems}
\phantomsection\label{ambrosia.util:ambrosia.util.get_class}\pysiglinewithargsret{\code{ambrosia.util.}\bfcode{get\_class}}{\emph{name}}{}
\end{fulllineitems}

\index{get\_logger() (in module ambrosia.util)}

\begin{fulllineitems}
\phantomsection\label{ambrosia.util:ambrosia.util.get_logger}\pysiglinewithargsret{\code{ambrosia.util.}\bfcode{get\_logger}}{\emph{o}}{}
Create a logger for a object.
\begin{quote}\begin{description}
\item[{Parameters}] \leavevmode
\textbf{o} (\emph{object}) -- the \emph{self} reference of a object

\end{description}\end{quote}

\end{fulllineitems}

\index{join\_command() (in module ambrosia.util)}

\begin{fulllineitems}
\phantomsection\label{ambrosia.util:ambrosia.util.join_command}\pysiglinewithargsret{\code{ambrosia.util.}\bfcode{join\_command}}{\emph{lst}}{}
Convert a list of arguments (argv) to a command line

\end{fulllineitems}

\index{js\_date() (in module ambrosia.util)}

\begin{fulllineitems}
\phantomsection\label{ambrosia.util:ambrosia.util.js_date}\pysiglinewithargsret{\code{ambrosia.util.}\bfcode{js\_date}}{\emph{date}}{}
Converts a datetime.datetime to a float timestamp for javascript

\end{fulllineitems}

\index{obj\_classname() (in module ambrosia.util)}

\begin{fulllineitems}
\phantomsection\label{ambrosia.util:ambrosia.util.obj_classname}\pysiglinewithargsret{\code{ambrosia.util.}\bfcode{obj\_classname}}{\emph{o}}{}
Returns the full class name of an object

\end{fulllineitems}

\index{serialize\_obj() (in module ambrosia.util)}

\begin{fulllineitems}
\phantomsection\label{ambrosia.util:ambrosia.util.serialize_obj}\pysiglinewithargsret{\code{ambrosia.util.}\bfcode{serialize\_obj}}{\emph{obj}, \emph{fp}}{}
Serialize an object
\begin{quote}\begin{description}
\item[{Parameters}] \leavevmode
\textbf{obj} (\emph{object}) -- the object to serialize

\end{description}\end{quote}

Returns a JSON-string containing the ``hollow'' object and a list with objects. All actual data is striped from the
object and appended to the objects list.

For example this function converts the dict:

\begin{Verbatim}[commandchars=\\\{\}]
\PYG{p}{\PYGZob{}}
    \PYG{l+s}{\PYGZsq{}}\PYG{l+s}{test}\PYG{l+s}{\PYGZsq{}}\PYG{p}{:} \PYG{p}{[}\PYG{n+nb+bp}{None}\PYG{p}{,} \PYG{l+m+mi}{1}\PYG{p}{,} \PYG{l+s}{\PYGZsq{}}\PYG{l+s}{test}\PYG{l+s}{\PYGZsq{}}\PYG{p}{]}
\PYG{p}{\PYGZcb{}}
\end{Verbatim}

into the following ``hollow'' object:

\begin{Verbatim}[commandchars=\\\{\}]
\PYG{p}{\PYGZob{}}
    \PYG{l+m+mi}{1}\PYG{p}{:} \PYG{p}{[}\PYG{l+m+mi}{0}\PYG{p}{,} \PYG{l+m+mi}{2}\PYG{p}{,} \PYG{l+m+mi}{1}\PYG{p}{]}
\PYG{p}{\PYGZcb{}}
\end{Verbatim}

and the following objects list:
.. code-block:: python
\begin{quote}

{[}None, `test', 1{]}
\end{quote}

All the data in the ``hollow'' object references data in the objects list. E.g. \emph{1} references `test'.

This type of is used for compression. Since Ambrosia generates a lot of data containing the same string multiple
times this serialization should reduce the size of the serialized data (since a string only has to be stored once
in the objects list. E.g. in the example above the string `test' is contained two times in the original data but
only once in the objects list.

\end{fulllineitems}

\index{unique\_id() (in module ambrosia.util)}

\begin{fulllineitems}
\phantomsection\label{ambrosia.util:ambrosia.util.unique_id}\pysiglinewithargsret{\code{ambrosia.util.}\bfcode{unique\_id}}{}{}
Generates a uniqe id

\end{fulllineitems}



\subsubsection{Module contents}
\label{ambrosia:module-contents}\label{ambrosia:module-ambrosia}\index{ambrosia (module)}\index{Ambrosia (class in ambrosia)}

\begin{fulllineitems}
\phantomsection\label{ambrosia:ambrosia.Ambrosia}\pysiglinewithargsret{\strong{class }\code{ambrosia.}\bfcode{Ambrosia}}{\emph{root}, \emph{configfile}}{}
Bases: \code{object}

This class is the main class that performs starts all actions
\begin{quote}\begin{description}
\item[{Parameters}] \leavevmode\begin{itemize}
\item {} 
\textbf{root} (\emph{xml.etree.Element}) -- The document root of the XML report

\item {} 
\textbf{configfile} (\emph{str}) -- the config file path

\end{itemize}

\end{description}\end{quote}

Upon object creation the report is being parsed. General information (such as the APK filename) as well as
Plugin-specific values are obtained. Plugin-specific values are parsed using {\hyperref[ambrosia:ambrosia.ResultParser]{\code{ResultParser}}} instances.
\index{adjust\_times() (ambrosia.Ambrosia method)}

\begin{fulllineitems}
\phantomsection\label{ambrosia:ambrosia.Ambrosia.adjust_times}\pysiglinewithargsret{\bfcode{adjust\_times}}{}{}
This method adjusts the timestamps of all events.

Since the emulator clock and the clock of the analysis machine may be different (e.g. when the simulation plugin
turns time ahead) the timestamps of several Events (with timestamps comming from the emulator) have to be
adjusted (to the clock of the analysis machine). See \code{ambrosia\_web.clocks.ClockSyncer}.

This method should be called right after the \code{ambrosia\_web.Ambrosia} class has been created.

\end{fulllineitems}

\index{correlate() (ambrosia.Ambrosia method)}

\begin{fulllineitems}
\phantomsection\label{ambrosia:ambrosia.Ambrosia.correlate}\pysiglinewithargsret{\bfcode{correlate}}{}{}
Correlates the events

This method finds all Correlaters (see \code{ambrosia\_web.plugins.PluginManager}) and starts them.

A {\hyperref[ambrosia:ambrosia.Correlator]{\code{Correlator}}} searches for specific events (at top level) and wraps them into other events. E.g. a
open(), read() and close() SyscallEvents are wrapped into a FileEvent. The {\hyperref[ambrosia:ambrosia.Correlator]{\code{Correlator}}} can also do
several passes (e.g. wrap 3 events of type A into a event B, then wrap several B events and wrap them into a C
event).

Should be called after {\hyperref[ambrosia:ambrosia.Ambrosia.adjust_times]{\code{Ambrosia.adjust\_times()}}}.

\end{fulllineitems}

\index{serialize() (ambrosia.Ambrosia method)}

\begin{fulllineitems}
\phantomsection\label{ambrosia:ambrosia.Ambrosia.serialize}\pysiglinewithargsret{\bfcode{serialize}}{\emph{fp}}{}
Serialize Events into a compact text format (see \code{ambrosia\_web.util.serialize\_obj()}).

Should be called after {\hyperref[ambrosia:ambrosia.Ambrosia.correlate]{\code{Ambrosia.correlate()}}}.
\begin{quote}\begin{description}
\item[{Returns}] \leavevmode
the serialized string

\end{description}\end{quote}

\end{fulllineitems}


\end{fulllineitems}

\index{Correlator (class in ambrosia)}

\begin{fulllineitems}
\phantomsection\label{ambrosia:ambrosia.Correlator}\pysiglinewithargsret{\strong{class }\code{ambrosia.}\bfcode{Correlator}}{\emph{context}}{}
Bases: \code{object}

Base class for Correlators.

A Correlator is called after all primitive events (like Syscalls, API calls etc.) have been acquired. The Correlator
is responsible to find matching primitive events (or events generated by other Correlators) and wrap them into
higher-level Events.

The \code{ambrosia\_web.plugins.PluginInfoTop} specifies a priority for each Correlator. This allows to force a
specific order in which the Correlators are called (e.g. if a Correlator relies on Events generated by another
Correlator).
\index{correlate() (ambrosia.Correlator method)}

\begin{fulllineitems}
\phantomsection\label{ambrosia:ambrosia.Correlator.correlate}\pysiglinewithargsret{\bfcode{correlate}}{}{}
\textbf{Must} be implemented by the specific class.

\end{fulllineitems}

\index{update\_tree() (ambrosia.Correlator method)}

\begin{fulllineitems}
\phantomsection\label{ambrosia:ambrosia.Correlator.update_tree}\pysiglinewithargsret{\bfcode{update\_tree}}{}{}
This method may be used by subclasses to update the result event tree.

If the subclass uses the \code{ambrosia\_web.model.Event.iter\_events()} in a loop it may not add or remove events from
the event tree. Otherwise events may be skipped or processed twice. Therefore the subclass may use the \emph{to\_add}
and \emph{to\_remove} attributes to store events that should be added and removed from the top level of the event
tree. Afterwards this method can be used to process the pending adds/removes.

\end{fulllineitems}


\end{fulllineitems}

\index{ResultParser (class in ambrosia)}

\begin{fulllineitems}
\phantomsection\label{ambrosia:ambrosia.ResultParser}\pysigline{\strong{class }\code{ambrosia.}\bfcode{ResultParser}}
Bases: \code{object}

Allows a plugin to implement parsers for the results in the XML report (Abstract base class).

When the \emph{result} section of a report is parsed \textbf{all} ResultParsers of all plugins are called for each result
section. Each ResultParser may generate primitive events from the supplied XML Element.
\index{finish() (ambrosia.ResultParser method)}

\begin{fulllineitems}
\phantomsection\label{ambrosia:ambrosia.ResultParser.finish}\pysiglinewithargsret{\bfcode{finish}}{\emph{context}}{}
Called after all parsing has been done
\begin{quote}\begin{description}
\item[{Parameters}] \leavevmode
\textbf{context} (\emph{ambrosia\_web.context.AmbrosiaContext}) -- The current context.

\end{description}\end{quote}

\end{fulllineitems}

\index{parse() (ambrosia.ResultParser method)}

\begin{fulllineitems}
\phantomsection\label{ambrosia:ambrosia.ResultParser.parse}\pysiglinewithargsret{\bfcode{parse}}{\emph{name}, \emph{el}, \emph{context}}{}
The actual parsing routine \textbf{must} be implemented by the specific class.
\begin{quote}\begin{description}
\item[{Parameters}] \leavevmode\begin{itemize}
\item {} 
\textbf{name} (\emph{str}) -- The name of the tag (child of the \emph{results} element).

\item {} 
\textbf{el} (\emph{xml.etree.Element}) -- The result element to parse.

\item {} 
\textbf{context} (\emph{ambrosia\_web.context.AmbrosiaContext}) -- The current context.

\end{itemize}

\end{description}\end{quote}

\end{fulllineitems}

\index{prepare() (ambrosia.ResultParser method)}

\begin{fulllineitems}
\phantomsection\label{ambrosia:ambrosia.ResultParser.prepare}\pysiglinewithargsret{\bfcode{prepare}}{\emph{context}}{}
Called before any parsing is done by any ResultParser. \textbf{May} be overwritten by specific class.
\begin{quote}\begin{description}
\item[{Parameters}] \leavevmode
\textbf{context} (\emph{ambrosia\_web.context.AmbrosiaContext}) -- The current context.

\end{description}\end{quote}

\end{fulllineitems}

\index{start\_parsers() (ambrosia.ResultParser static method)}

\begin{fulllineitems}
\phantomsection\label{ambrosia:ambrosia.ResultParser.start_parsers}\pysiglinewithargsret{\strong{static }\bfcode{start\_parsers}}{\emph{el}, \emph{context}}{}
Starts all ResultParsers registered in the \code{ambrosia\_web.plugins.PluginManager}.
\begin{quote}\begin{description}
\item[{Parameters}] \leavevmode
\textbf{context} (\emph{ambrosia\_web.context.AmbrosiaContext}) -- The current context.

\end{description}\end{quote}

\end{fulllineitems}


\end{fulllineitems}



\subsection{ambrosia\_plugins package}
\label{ambrosia_plugins:ambrosia-plugins-package}\label{ambrosia_plugins::doc}

\subsubsection{Subpackages}
\label{ambrosia_plugins:subpackages}

\paragraph{ambrosia\_plugins.apimonitor package}
\label{ambrosia_plugins.apimonitor:ambrosia-plugins-apimonitor-package}\label{ambrosia_plugins.apimonitor::doc}

\subparagraph{Module contents}
\label{ambrosia_plugins.apimonitor:module-contents}\label{ambrosia_plugins.apimonitor:module-ambrosia_plugins.apimonitor}\index{ambrosia\_plugins.apimonitor (module)}\index{AndroidApicallEvent (class in ambrosia\_plugins.apimonitor)}

\begin{fulllineitems}
\phantomsection\label{ambrosia_plugins.apimonitor:ambrosia_plugins.apimonitor.AndroidApicallEvent}\pysiglinewithargsret{\strong{class }\code{ambrosia\_plugins.apimonitor.}\bfcode{AndroidApicallEvent}}{\emph{api}, \emph{method}, \emph{params}, \emph{returnval}, \emph{start\_ts}}{}
Bases: {\hyperref[ambrosia.model:ambrosia.model.Event]{\code{ambrosia.model.Event}}}

Represents an API call of the App
\begin{quote}\begin{description}
\item[{Parameters}] \leavevmode\begin{itemize}
\item {} 
\textbf{api} (\emph{str}) -- the class referenced by this API call

\item {} 
\textbf{method} (\emph{str}) -- the method called

\item {} 
\textbf{returnval} (\emph{str}) -- the return value

\item {} 
\textbf{start\_ts} (\emph{datetime.datetime}) -- the time the API call occurred (emulator clock)

\end{itemize}

\end{description}\end{quote}
\index{adjust\_times() (ambrosia\_plugins.apimonitor.AndroidApicallEvent method)}

\begin{fulllineitems}
\phantomsection\label{ambrosia_plugins.apimonitor:ambrosia_plugins.apimonitor.AndroidApicallEvent.adjust_times}\pysiglinewithargsret{\bfcode{adjust\_times}}{\emph{context}}{}
\end{fulllineitems}

\index{get\_serializeable\_properties() (ambrosia\_plugins.apimonitor.AndroidApicallEvent method)}

\begin{fulllineitems}
\phantomsection\label{ambrosia_plugins.apimonitor:ambrosia_plugins.apimonitor.AndroidApicallEvent.get_serializeable_properties}\pysiglinewithargsret{\bfcode{get\_serializeable\_properties}}{}{}
\end{fulllineitems}

\index{indices (ambrosia\_plugins.apimonitor.AndroidApicallEvent attribute)}

\begin{fulllineitems}
\phantomsection\label{ambrosia_plugins.apimonitor:ambrosia_plugins.apimonitor.AndroidApicallEvent.indices}\pysigline{\bfcode{indices}\strong{ = \{\}}}
\end{fulllineitems}


\end{fulllineitems}

\index{ApiCallCorrelator (class in ambrosia\_plugins.apimonitor)}

\begin{fulllineitems}
\phantomsection\label{ambrosia_plugins.apimonitor:ambrosia_plugins.apimonitor.ApiCallCorrelator}\pysiglinewithargsret{\strong{class }\code{ambrosia\_plugins.apimonitor.}\bfcode{ApiCallCorrelator}}{\emph{context}}{}
Bases: {\hyperref[ambrosia:ambrosia.Correlator]{\code{ambrosia.Correlator}}}

Goes through all API calls and wraps known API calls into higher-level events.
\begin{quote}\begin{description}
\item[{Parameters}] \leavevmode
\textbf{context} (\emph{ambrosia\_web.context.AmbrosiaContext}) -- the current context.

\end{description}\end{quote}
\index{correlate() (ambrosia\_plugins.apimonitor.ApiCallCorrelator method)}

\begin{fulllineitems}
\phantomsection\label{ambrosia_plugins.apimonitor:ambrosia_plugins.apimonitor.ApiCallCorrelator.correlate}\pysiglinewithargsret{\bfcode{correlate}}{}{}
\end{fulllineitems}


\end{fulllineitems}

\index{ApimonitorPluginParser (class in ambrosia\_plugins.apimonitor)}

\begin{fulllineitems}
\phantomsection\label{ambrosia_plugins.apimonitor:ambrosia_plugins.apimonitor.ApimonitorPluginParser}\pysigline{\strong{class }\code{ambrosia\_plugins.apimonitor.}\bfcode{ApimonitorPluginParser}}
Bases: {\hyperref[ambrosia:ambrosia.ResultParser]{\code{ambrosia.ResultParser}}}

The plugin parser that parses the apimonitor tag
\index{finish() (ambrosia\_plugins.apimonitor.ApimonitorPluginParser method)}

\begin{fulllineitems}
\phantomsection\label{ambrosia_plugins.apimonitor:ambrosia_plugins.apimonitor.ApimonitorPluginParser.finish}\pysiglinewithargsret{\bfcode{finish}}{\emph{context}}{}
\end{fulllineitems}

\index{parse() (ambrosia\_plugins.apimonitor.ApimonitorPluginParser method)}

\begin{fulllineitems}
\phantomsection\label{ambrosia_plugins.apimonitor:ambrosia_plugins.apimonitor.ApimonitorPluginParser.parse}\pysiglinewithargsret{\bfcode{parse}}{\emph{name}, \emph{el}, \emph{context}}{}
\end{fulllineitems}


\end{fulllineitems}

\index{CallLogAccessEvent (class in ambrosia\_plugins.apimonitor)}

\begin{fulllineitems}
\phantomsection\label{ambrosia_plugins.apimonitor:ambrosia_plugins.apimonitor.CallLogAccessEvent}\pysigline{\strong{class }\code{ambrosia\_plugins.apimonitor.}\bfcode{CallLogAccessEvent}}
Bases: {\hyperref[ambrosia.model:ambrosia.model.Event]{\code{ambrosia.model.Event}}}

App accesses call logs
\index{get\_serializeable\_properties() (ambrosia\_plugins.apimonitor.CallLogAccessEvent method)}

\begin{fulllineitems}
\phantomsection\label{ambrosia_plugins.apimonitor:ambrosia_plugins.apimonitor.CallLogAccessEvent.get_serializeable_properties}\pysiglinewithargsret{\bfcode{get\_serializeable\_properties}}{}{}
\end{fulllineitems}

\index{indices (ambrosia\_plugins.apimonitor.CallLogAccessEvent attribute)}

\begin{fulllineitems}
\phantomsection\label{ambrosia_plugins.apimonitor:ambrosia_plugins.apimonitor.CallLogAccessEvent.indices}\pysigline{\bfcode{indices}\strong{ = \{\}}}
\end{fulllineitems}


\end{fulllineitems}

\index{ContactAccessEvent (class in ambrosia\_plugins.apimonitor)}

\begin{fulllineitems}
\phantomsection\label{ambrosia_plugins.apimonitor:ambrosia_plugins.apimonitor.ContactAccessEvent}\pysigline{\strong{class }\code{ambrosia\_plugins.apimonitor.}\bfcode{ContactAccessEvent}}
Bases: {\hyperref[ambrosia.model:ambrosia.model.Event]{\code{ambrosia.model.Event}}}

App accesses contacts
\index{get\_serializeable\_properties() (ambrosia\_plugins.apimonitor.ContactAccessEvent method)}

\begin{fulllineitems}
\phantomsection\label{ambrosia_plugins.apimonitor:ambrosia_plugins.apimonitor.ContactAccessEvent.get_serializeable_properties}\pysiglinewithargsret{\bfcode{get\_serializeable\_properties}}{}{}
\end{fulllineitems}

\index{indices (ambrosia\_plugins.apimonitor.ContactAccessEvent attribute)}

\begin{fulllineitems}
\phantomsection\label{ambrosia_plugins.apimonitor:ambrosia_plugins.apimonitor.ContactAccessEvent.indices}\pysigline{\bfcode{indices}\strong{ = \{\}}}
\end{fulllineitems}


\end{fulllineitems}

\index{PhoneCallEvent (class in ambrosia\_plugins.apimonitor)}

\begin{fulllineitems}
\phantomsection\label{ambrosia_plugins.apimonitor:ambrosia_plugins.apimonitor.PhoneCallEvent}\pysigline{\strong{class }\code{ambrosia\_plugins.apimonitor.}\bfcode{PhoneCallEvent}}
Bases: {\hyperref[ambrosia.model:ambrosia.model.Event]{\code{ambrosia.model.Event}}}

App calls someone
\index{get\_serializeable\_properties() (ambrosia\_plugins.apimonitor.PhoneCallEvent method)}

\begin{fulllineitems}
\phantomsection\label{ambrosia_plugins.apimonitor:ambrosia_plugins.apimonitor.PhoneCallEvent.get_serializeable_properties}\pysiglinewithargsret{\bfcode{get\_serializeable\_properties}}{}{}
\end{fulllineitems}

\index{indices (ambrosia\_plugins.apimonitor.PhoneCallEvent attribute)}

\begin{fulllineitems}
\phantomsection\label{ambrosia_plugins.apimonitor:ambrosia_plugins.apimonitor.PhoneCallEvent.indices}\pysigline{\bfcode{indices}\strong{ = \{\}}}
\end{fulllineitems}


\end{fulllineitems}

\index{PluginInfo (class in ambrosia\_plugins.apimonitor)}

\begin{fulllineitems}
\phantomsection\label{ambrosia_plugins.apimonitor:ambrosia_plugins.apimonitor.PluginInfo}\pysigline{\strong{class }\code{ambrosia\_plugins.apimonitor.}\bfcode{PluginInfo}}
Bases: {\hyperref[ambrosia.plugins:ambrosia.plugins.PluginInfoTop]{\code{ambrosia.plugins.PluginInfoTop}}}
\index{correlators() (ambrosia\_plugins.apimonitor.PluginInfo static method)}

\begin{fulllineitems}
\phantomsection\label{ambrosia_plugins.apimonitor:ambrosia_plugins.apimonitor.PluginInfo.correlators}\pysiglinewithargsret{\strong{static }\bfcode{correlators}}{}{}
\end{fulllineitems}

\index{parsers() (ambrosia\_plugins.apimonitor.PluginInfo static method)}

\begin{fulllineitems}
\phantomsection\label{ambrosia_plugins.apimonitor:ambrosia_plugins.apimonitor.PluginInfo.parsers}\pysiglinewithargsret{\strong{static }\bfcode{parsers}}{}{}
\end{fulllineitems}


\end{fulllineitems}

\index{SMSAccessEvent (class in ambrosia\_plugins.apimonitor)}

\begin{fulllineitems}
\phantomsection\label{ambrosia_plugins.apimonitor:ambrosia_plugins.apimonitor.SMSAccessEvent}\pysigline{\strong{class }\code{ambrosia\_plugins.apimonitor.}\bfcode{SMSAccessEvent}}
Bases: {\hyperref[ambrosia.model:ambrosia.model.Event]{\code{ambrosia.model.Event}}}

App accesses SMS
\index{get\_serializeable\_properties() (ambrosia\_plugins.apimonitor.SMSAccessEvent method)}

\begin{fulllineitems}
\phantomsection\label{ambrosia_plugins.apimonitor:ambrosia_plugins.apimonitor.SMSAccessEvent.get_serializeable_properties}\pysiglinewithargsret{\bfcode{get\_serializeable\_properties}}{}{}
\end{fulllineitems}

\index{indices (ambrosia\_plugins.apimonitor.SMSAccessEvent attribute)}

\begin{fulllineitems}
\phantomsection\label{ambrosia_plugins.apimonitor:ambrosia_plugins.apimonitor.SMSAccessEvent.indices}\pysigline{\bfcode{indices}\strong{ = \{\}}}
\end{fulllineitems}


\end{fulllineitems}



\paragraph{ambrosia\_plugins.events package}
\label{ambrosia_plugins.events:ambrosia-plugins-events-package}\label{ambrosia_plugins.events::doc}

\subparagraph{Module contents}
\label{ambrosia_plugins.events:module-ambrosia_plugins.events}\label{ambrosia_plugins.events:module-contents}\index{ambrosia\_plugins.events (module)}\index{ANANASEvent (class in ambrosia\_plugins.events)}

\begin{fulllineitems}
\phantomsection\label{ambrosia_plugins.events:ambrosia_plugins.events.ANANASEvent}\pysiglinewithargsret{\strong{class }\code{ambrosia\_plugins.events.}\bfcode{ANANASEvent}}{\emph{name}, \emph{timestamp}, \emph{params}}{}
Bases: {\hyperref[ambrosia.model:ambrosia.model.Event]{\code{ambrosia.model.Event}}}

Represents an event generated by the ANANAS analysis system itself. Any action performed by ANANAS that affects
the emulator (e.g. execution of a command) is recorded in a ANANASEvent.
\begin{quote}\begin{description}
\item[{Parameters}] \leavevmode\begin{itemize}
\item {} 
\textbf{name} (\emph{str}) -- the type of the event

\item {} 
\textbf{timestamp} (\emph{datetime.datetime}) -- the time the event occurred (host clock)

\item {} 
\textbf{params} -- additional parameters (any serializable data structure)

\end{itemize}

\end{description}\end{quote}
\index{get\_serializeable\_properties() (ambrosia\_plugins.events.ANANASEvent method)}

\begin{fulllineitems}
\phantomsection\label{ambrosia_plugins.events:ambrosia_plugins.events.ANANASEvent.get_serializeable_properties}\pysiglinewithargsret{\bfcode{get\_serializeable\_properties}}{}{}
\end{fulllineitems}

\index{indices (ambrosia\_plugins.events.ANANASEvent attribute)}

\begin{fulllineitems}
\phantomsection\label{ambrosia_plugins.events:ambrosia_plugins.events.ANANASEvent.indices}\pysigline{\bfcode{indices}\strong{ = set({[}'start\_ts', `name'{]})}}
\end{fulllineitems}


\end{fulllineitems}

\index{EventParser (class in ambrosia\_plugins.events)}

\begin{fulllineitems}
\phantomsection\label{ambrosia_plugins.events:ambrosia_plugins.events.EventParser}\pysigline{\strong{class }\code{ambrosia\_plugins.events.}\bfcode{EventParser}}
Bases: {\hyperref[ambrosia:ambrosia.ResultParser]{\code{ambrosia.ResultParser}}}
\index{parse() (ambrosia\_plugins.events.EventParser method)}

\begin{fulllineitems}
\phantomsection\label{ambrosia_plugins.events:ambrosia_plugins.events.EventParser.parse}\pysiglinewithargsret{\bfcode{parse}}{\emph{name}, \emph{el}, \emph{context}}{}
\end{fulllineitems}


\end{fulllineitems}

\index{PluginInfo (class in ambrosia\_plugins.events)}

\begin{fulllineitems}
\phantomsection\label{ambrosia_plugins.events:ambrosia_plugins.events.PluginInfo}\pysigline{\strong{class }\code{ambrosia\_plugins.events.}\bfcode{PluginInfo}}
Bases: {\hyperref[ambrosia.plugins:ambrosia.plugins.PluginInfoTop]{\code{ambrosia.plugins.PluginInfoTop}}}
\index{correlators() (ambrosia\_plugins.events.PluginInfo static method)}

\begin{fulllineitems}
\phantomsection\label{ambrosia_plugins.events:ambrosia_plugins.events.PluginInfo.correlators}\pysiglinewithargsret{\strong{static }\bfcode{correlators}}{}{}
\end{fulllineitems}

\index{parsers() (ambrosia\_plugins.events.PluginInfo static method)}

\begin{fulllineitems}
\phantomsection\label{ambrosia_plugins.events:ambrosia_plugins.events.PluginInfo.parsers}\pysiglinewithargsret{\strong{static }\bfcode{parsers}}{}{}
\end{fulllineitems}


\end{fulllineitems}



\paragraph{ambrosia\_plugins.lkm package}
\label{ambrosia_plugins.lkm:ambrosia-plugins-lkm-package}\label{ambrosia_plugins.lkm::doc}

\subparagraph{Submodules}
\label{ambrosia_plugins.lkm:submodules}

\subparagraph{ambrosia\_plugins.lkm.events module}
\label{ambrosia_plugins.lkm:module-ambrosia_plugins.lkm.events}\label{ambrosia_plugins.lkm:ambrosia-plugins-lkm-events-module}\index{ambrosia\_plugins.lkm.events (module)}\index{ANANASAdbShellExecEvent (class in ambrosia\_plugins.lkm.events)}

\begin{fulllineitems}
\phantomsection\label{ambrosia_plugins.lkm:ambrosia_plugins.lkm.events.ANANASAdbShellExecEvent}\pysiglinewithargsret{\strong{class }\code{ambrosia\_plugins.lkm.events.}\bfcode{ANANASAdbShellExecEvent}}{\emph{process}}{}
Bases: {\hyperref[ambrosia.model:ambrosia.model.Event]{\code{ambrosia.model.Event}}}

Represents a command that has been executed by ANANAS
\index{get\_serializeable\_properties() (ambrosia\_plugins.lkm.events.ANANASAdbShellExecEvent method)}

\begin{fulllineitems}
\phantomsection\label{ambrosia_plugins.lkm:ambrosia_plugins.lkm.events.ANANASAdbShellExecEvent.get_serializeable_properties}\pysiglinewithargsret{\bfcode{get\_serializeable\_properties}}{}{}
\end{fulllineitems}

\index{indices (ambrosia\_plugins.lkm.events.ANANASAdbShellExecEvent attribute)}

\begin{fulllineitems}
\phantomsection\label{ambrosia_plugins.lkm:ambrosia_plugins.lkm.events.ANANASAdbShellExecEvent.indices}\pysigline{\bfcode{indices}\strong{ = set({[}{]})}}
\end{fulllineitems}


\end{fulllineitems}

\index{APKInstallEvent (class in ambrosia\_plugins.lkm.events)}

\begin{fulllineitems}
\phantomsection\label{ambrosia_plugins.lkm:ambrosia_plugins.lkm.events.APKInstallEvent}\pysiglinewithargsret{\strong{class }\code{ambrosia\_plugins.lkm.events.}\bfcode{APKInstallEvent}}{\emph{file}, \emph{process}}{}
Bases: {\hyperref[ambrosia.model:ambrosia.model.Event]{\code{ambrosia.model.Event}}}
\index{get\_serializeable\_properties() (ambrosia\_plugins.lkm.events.APKInstallEvent method)}

\begin{fulllineitems}
\phantomsection\label{ambrosia_plugins.lkm:ambrosia_plugins.lkm.events.APKInstallEvent.get_serializeable_properties}\pysiglinewithargsret{\bfcode{get\_serializeable\_properties}}{}{}
\end{fulllineitems}

\index{indices (ambrosia\_plugins.lkm.events.APKInstallEvent attribute)}

\begin{fulllineitems}
\phantomsection\label{ambrosia_plugins.lkm:ambrosia_plugins.lkm.events.APKInstallEvent.indices}\pysigline{\bfcode{indices}\strong{ = set({[}{]})}}
\end{fulllineitems}


\end{fulllineitems}

\index{AnonymousFileEvent (class in ambrosia\_plugins.lkm.events)}

\begin{fulllineitems}
\phantomsection\label{ambrosia_plugins.lkm:ambrosia_plugins.lkm.events.AnonymousFileEvent}\pysiglinewithargsret{\strong{class }\code{ambrosia\_plugins.lkm.events.}\bfcode{AnonymousFileEvent}}{\emph{description}, \emph{process}, \emph{context}, \emph{successful=True}}{}
Bases: {\hyperref[ambrosia_plugins.lkm:ambrosia_plugins.lkm.events.FileEvent]{\code{ambrosia\_plugins.lkm.events.FileEvent}}}

Represents an operation that happens on a file without a name (e.g. an unnamed pipe)
\index{get\_serializeable\_properties() (ambrosia\_plugins.lkm.events.AnonymousFileEvent method)}

\begin{fulllineitems}
\phantomsection\label{ambrosia_plugins.lkm:ambrosia_plugins.lkm.events.AnonymousFileEvent.get_serializeable_properties}\pysiglinewithargsret{\bfcode{get\_serializeable\_properties}}{}{}
\end{fulllineitems}

\index{indices (ambrosia\_plugins.lkm.events.AnonymousFileEvent attribute)}

\begin{fulllineitems}
\phantomsection\label{ambrosia_plugins.lkm:ambrosia_plugins.lkm.events.AnonymousFileEvent.indices}\pysigline{\bfcode{indices}\strong{ = set({[}'process'{]})}}
\end{fulllineitems}


\end{fulllineitems}

\index{CommandExecuteEvent (class in ambrosia\_plugins.lkm.events)}

\begin{fulllineitems}
\phantomsection\label{ambrosia_plugins.lkm:ambrosia_plugins.lkm.events.CommandExecuteEvent}\pysiglinewithargsret{\strong{class }\code{ambrosia\_plugins.lkm.events.}\bfcode{CommandExecuteEvent}}{\emph{path}, \emph{command}, \emph{process}, \emph{execfile}}{}
Bases: {\hyperref[ambrosia.model:ambrosia.model.Event]{\code{ambrosia.model.Event}}}

Represents the execution of a command (including fork, exec, library loads, etc.)
\index{get\_serializeable\_properties() (ambrosia\_plugins.lkm.events.CommandExecuteEvent method)}

\begin{fulllineitems}
\phantomsection\label{ambrosia_plugins.lkm:ambrosia_plugins.lkm.events.CommandExecuteEvent.get_serializeable_properties}\pysiglinewithargsret{\bfcode{get\_serializeable\_properties}}{}{}
\end{fulllineitems}

\index{indices (ambrosia\_plugins.lkm.events.CommandExecuteEvent attribute)}

\begin{fulllineitems}
\phantomsection\label{ambrosia_plugins.lkm:ambrosia_plugins.lkm.events.CommandExecuteEvent.indices}\pysigline{\bfcode{indices}\strong{ = set({[}'process'{]})}}
\end{fulllineitems}


\end{fulllineitems}

\index{CreateDirEvent (class in ambrosia\_plugins.lkm.events)}

\begin{fulllineitems}
\phantomsection\label{ambrosia_plugins.lkm:ambrosia_plugins.lkm.events.CreateDirEvent}\pysiglinewithargsret{\strong{class }\code{ambrosia\_plugins.lkm.events.}\bfcode{CreateDirEvent}}{\emph{start\_ts}, \emph{end\_ts}, \emph{process}, \emph{successful}, \emph{file}}{}
Bases: {\hyperref[ambrosia.model:ambrosia.model.Event]{\code{ambrosia.model.Event}}}

Represents an mkdir() syscall
\index{get\_serializeable\_properties() (ambrosia\_plugins.lkm.events.CreateDirEvent method)}

\begin{fulllineitems}
\phantomsection\label{ambrosia_plugins.lkm:ambrosia_plugins.lkm.events.CreateDirEvent.get_serializeable_properties}\pysiglinewithargsret{\bfcode{get\_serializeable\_properties}}{}{}
\end{fulllineitems}

\index{indices (ambrosia\_plugins.lkm.events.CreateDirEvent attribute)}

\begin{fulllineitems}
\phantomsection\label{ambrosia_plugins.lkm:ambrosia_plugins.lkm.events.CreateDirEvent.indices}\pysigline{\bfcode{indices}\strong{ = set({[}'process'{]})}}
\end{fulllineitems}


\end{fulllineitems}

\index{DeletePathEvent (class in ambrosia\_plugins.lkm.events)}

\begin{fulllineitems}
\phantomsection\label{ambrosia_plugins.lkm:ambrosia_plugins.lkm.events.DeletePathEvent}\pysiglinewithargsret{\strong{class }\code{ambrosia\_plugins.lkm.events.}\bfcode{DeletePathEvent}}{\emph{start\_ts}, \emph{end\_ts}, \emph{successful}, \emph{file}, \emph{process}}{}
Bases: {\hyperref[ambrosia.model:ambrosia.model.Event]{\code{ambrosia.model.Event}}}

Represents an unlink() syscall
\index{get\_serializeable\_properties() (ambrosia\_plugins.lkm.events.DeletePathEvent method)}

\begin{fulllineitems}
\phantomsection\label{ambrosia_plugins.lkm:ambrosia_plugins.lkm.events.DeletePathEvent.get_serializeable_properties}\pysiglinewithargsret{\bfcode{get\_serializeable\_properties}}{}{}
\end{fulllineitems}

\index{indices (ambrosia\_plugins.lkm.events.DeletePathEvent attribute)}

\begin{fulllineitems}
\phantomsection\label{ambrosia_plugins.lkm:ambrosia_plugins.lkm.events.DeletePathEvent.indices}\pysigline{\bfcode{indices}\strong{ = set({[}{]})}}
\end{fulllineitems}


\end{fulllineitems}

\index{ExecEvent (class in ambrosia\_plugins.lkm.events)}

\begin{fulllineitems}
\phantomsection\label{ambrosia_plugins.lkm:ambrosia_plugins.lkm.events.ExecEvent}\pysiglinewithargsret{\strong{class }\code{ambrosia\_plugins.lkm.events.}\bfcode{ExecEvent}}{\emph{start\_ts}, \emph{end\_ts}, \emph{path}, \emph{argv}, \emph{env}, \emph{process}}{}
Bases: {\hyperref[ambrosia.model:ambrosia.model.Event]{\code{ambrosia.model.Event}}}

Represents an execve() syscall
\index{get\_serializeable\_properties() (ambrosia\_plugins.lkm.events.ExecEvent method)}

\begin{fulllineitems}
\phantomsection\label{ambrosia_plugins.lkm:ambrosia_plugins.lkm.events.ExecEvent.get_serializeable_properties}\pysiglinewithargsret{\bfcode{get\_serializeable\_properties}}{}{}
\end{fulllineitems}

\index{indices (ambrosia\_plugins.lkm.events.ExecEvent attribute)}

\begin{fulllineitems}
\phantomsection\label{ambrosia_plugins.lkm:ambrosia_plugins.lkm.events.ExecEvent.indices}\pysigline{\bfcode{indices}\strong{ = set({[}'process'{]})}}
\end{fulllineitems}


\end{fulllineitems}

\index{FileDescriptorEvent (class in ambrosia\_plugins.lkm.events)}

\begin{fulllineitems}
\phantomsection\label{ambrosia_plugins.lkm:ambrosia_plugins.lkm.events.FileDescriptorEvent}\pysiglinewithargsret{\strong{class }\code{ambrosia\_plugins.lkm.events.}\bfcode{FileDescriptorEvent}}{\emph{process}, \emph{successful}}{}
Bases: {\hyperref[ambrosia.model:ambrosia.model.Event]{\code{ambrosia.model.Event}}}

The base event for all file descriptor related events
\index{get\_serializeable\_properties() (ambrosia\_plugins.lkm.events.FileDescriptorEvent method)}

\begin{fulllineitems}
\phantomsection\label{ambrosia_plugins.lkm:ambrosia_plugins.lkm.events.FileDescriptorEvent.get_serializeable_properties}\pysiglinewithargsret{\bfcode{get\_serializeable\_properties}}{}{}
\end{fulllineitems}

\index{indices (ambrosia\_plugins.lkm.events.FileDescriptorEvent attribute)}

\begin{fulllineitems}
\phantomsection\label{ambrosia_plugins.lkm:ambrosia_plugins.lkm.events.FileDescriptorEvent.indices}\pysigline{\bfcode{indices}\strong{ = set({[}'process'{]})}}
\end{fulllineitems}


\end{fulllineitems}

\index{FileEvent (class in ambrosia\_plugins.lkm.events)}

\begin{fulllineitems}
\phantomsection\label{ambrosia_plugins.lkm:ambrosia_plugins.lkm.events.FileEvent}\pysiglinewithargsret{\strong{class }\code{ambrosia\_plugins.lkm.events.}\bfcode{FileEvent}}{\emph{file}, \emph{flags}, \emph{mode}, \emph{process}, \emph{successful}}{}
Bases: {\hyperref[ambrosia_plugins.lkm:ambrosia_plugins.lkm.events.FileDescriptorEvent]{\code{ambrosia\_plugins.lkm.events.FileDescriptorEvent}}}

Represents a normal file operation on a file, directory or pipe
\index{get\_serializeable\_properties() (ambrosia\_plugins.lkm.events.FileEvent method)}

\begin{fulllineitems}
\phantomsection\label{ambrosia_plugins.lkm:ambrosia_plugins.lkm.events.FileEvent.get_serializeable_properties}\pysiglinewithargsret{\bfcode{get\_serializeable\_properties}}{}{}
\end{fulllineitems}

\index{indices (ambrosia\_plugins.lkm.events.FileEvent attribute)}

\begin{fulllineitems}
\phantomsection\label{ambrosia_plugins.lkm:ambrosia_plugins.lkm.events.FileEvent.indices}\pysigline{\bfcode{indices}\strong{ = set({[}'process', `abspath'{]})}}
\end{fulllineitems}

\index{mode\_flags (ambrosia\_plugins.lkm.events.FileEvent attribute)}

\begin{fulllineitems}
\phantomsection\label{ambrosia_plugins.lkm:ambrosia_plugins.lkm.events.FileEvent.mode_flags}\pysigline{\bfcode{mode\_flags}\strong{ = \{`O\_DSYNC': 4096, `O\_DIRECTORY': 65536, `O\_LARGEFILE': 32768, `O\_CREAT': 64, `O\_PATH': 2097152, `O\_EXCL': 128, `O\_CLOEXEC': 524288, `O\_DIRECT': 16384, `O\_APPEND': 1024, `O\_NOCTTY': 256, `O\_NOATIME': 262144, `O\_WRONLY': 1, `O\_NONBLOCK': 2048, `O\_RDWR': 2, `O\_TRUNC': 512, `O\_NOFOLLOW': 131072\}}}
\end{fulllineitems}


\end{fulllineitems}

\index{JavaLibraryLoadEvent (class in ambrosia\_plugins.lkm.events)}

\begin{fulllineitems}
\phantomsection\label{ambrosia_plugins.lkm:ambrosia_plugins.lkm.events.JavaLibraryLoadEvent}\pysiglinewithargsret{\strong{class }\code{ambrosia\_plugins.lkm.events.}\bfcode{JavaLibraryLoadEvent}}{\emph{file}, \emph{process}, \emph{successful}, \emph{system\_library\_load}}{}
Bases: {\hyperref[ambrosia.model:ambrosia.model.Event]{\code{ambrosia.model.Event}}}

Represents dalvik library load operation
\index{get\_serializeable\_properties() (ambrosia\_plugins.lkm.events.JavaLibraryLoadEvent method)}

\begin{fulllineitems}
\phantomsection\label{ambrosia_plugins.lkm:ambrosia_plugins.lkm.events.JavaLibraryLoadEvent.get_serializeable_properties}\pysiglinewithargsret{\bfcode{get\_serializeable\_properties}}{}{}
\end{fulllineitems}

\index{indices (ambrosia\_plugins.lkm.events.JavaLibraryLoadEvent attribute)}

\begin{fulllineitems}
\phantomsection\label{ambrosia_plugins.lkm:ambrosia_plugins.lkm.events.JavaLibraryLoadEvent.indices}\pysigline{\bfcode{indices}\strong{ = set({[}'process'{]})}}
\end{fulllineitems}


\end{fulllineitems}

\index{LibraryLoadEvent (class in ambrosia\_plugins.lkm.events)}

\begin{fulllineitems}
\phantomsection\label{ambrosia_plugins.lkm:ambrosia_plugins.lkm.events.LibraryLoadEvent}\pysiglinewithargsret{\strong{class }\code{ambrosia\_plugins.lkm.events.}\bfcode{LibraryLoadEvent}}{\emph{file}, \emph{process}, \emph{successful}}{}
Bases: {\hyperref[ambrosia.model:ambrosia.model.Event]{\code{ambrosia.model.Event}}}

Represents mmap() operations on a library file
\index{get\_serializeable\_properties() (ambrosia\_plugins.lkm.events.LibraryLoadEvent method)}

\begin{fulllineitems}
\phantomsection\label{ambrosia_plugins.lkm:ambrosia_plugins.lkm.events.LibraryLoadEvent.get_serializeable_properties}\pysiglinewithargsret{\bfcode{get\_serializeable\_properties}}{}{}
\end{fulllineitems}

\index{indices (ambrosia\_plugins.lkm.events.LibraryLoadEvent attribute)}

\begin{fulllineitems}
\phantomsection\label{ambrosia_plugins.lkm:ambrosia_plugins.lkm.events.LibraryLoadEvent.indices}\pysigline{\bfcode{indices}\strong{ = set({[}'process'{]})}}
\end{fulllineitems}


\end{fulllineitems}

\index{MemoryMapEvent (class in ambrosia\_plugins.lkm.events)}

\begin{fulllineitems}
\phantomsection\label{ambrosia_plugins.lkm:ambrosia_plugins.lkm.events.MemoryMapEvent}\pysiglinewithargsret{\strong{class }\code{ambrosia\_plugins.lkm.events.}\bfcode{MemoryMapEvent}}{\emph{flags}, \emph{fd}, \emph{address}, \emph{process}, \emph{return\_value}, \emph{start\_ts}, \emph{end\_ts}}{}
Bases: {\hyperref[ambrosia.model:ambrosia.model.Event]{\code{ambrosia.model.Event}}}

Represents a call to mmap(). It's parent normally is a {\hyperref[ambrosia_plugins.lkm:ambrosia_plugins.lkm.events.FileDescriptorEvent]{\code{ambrosia\_plugins.lkm.events.FileDescriptorEvent}}}
\index{get\_serializeable\_properties() (ambrosia\_plugins.lkm.events.MemoryMapEvent method)}

\begin{fulllineitems}
\phantomsection\label{ambrosia_plugins.lkm:ambrosia_plugins.lkm.events.MemoryMapEvent.get_serializeable_properties}\pysiglinewithargsret{\bfcode{get\_serializeable\_properties}}{}{}
\end{fulllineitems}

\index{indices (ambrosia\_plugins.lkm.events.MemoryMapEvent attribute)}

\begin{fulllineitems}
\phantomsection\label{ambrosia_plugins.lkm:ambrosia_plugins.lkm.events.MemoryMapEvent.indices}\pysigline{\bfcode{indices}\strong{ = set({[}'process'{]})}}
\end{fulllineitems}

\index{mmap\_flags (ambrosia\_plugins.lkm.events.MemoryMapEvent attribute)}

\begin{fulllineitems}
\phantomsection\label{ambrosia_plugins.lkm:ambrosia_plugins.lkm.events.MemoryMapEvent.mmap_flags}\pysigline{\bfcode{mmap\_flags}\strong{ = \{`MAP\_NONBLOCK': 65536, `MAP\_EXECUTABLE': 4096, `MAP\_SHARED': 1, `MAP\_GROWSDOWN': 256, `MAP\_HUGETLB': 262144, `MAP\_STACK': 131072, `MAP\_PRIVATE': 2, `MAP\_DENYWRITE': 2048, `MAP\_LOCKED': 8192, `MAP\_FIXED': 16, `MAP\_NORESERVE': 16384, `MAP\_UNINITIALIZED': 67108864, `MAP\_ANONYMOUS': 32, `MAP\_POPULATE': 32768\}}}
\end{fulllineitems}


\end{fulllineitems}

\index{SendSignalEvent (class in ambrosia\_plugins.lkm.events)}

\begin{fulllineitems}
\phantomsection\label{ambrosia_plugins.lkm:ambrosia_plugins.lkm.events.SendSignalEvent}\pysiglinewithargsret{\strong{class }\code{ambrosia\_plugins.lkm.events.}\bfcode{SendSignalEvent}}{\emph{start\_ts}, \emph{end\_ts}, \emph{number}, \emph{process}, \emph{target\_process}}{}
Bases: {\hyperref[ambrosia.model:ambrosia.model.Event]{\code{ambrosia.model.Event}}}

Represents a kill() syscall
\index{get\_serializeable\_properties() (ambrosia\_plugins.lkm.events.SendSignalEvent method)}

\begin{fulllineitems}
\phantomsection\label{ambrosia_plugins.lkm:ambrosia_plugins.lkm.events.SendSignalEvent.get_serializeable_properties}\pysiglinewithargsret{\bfcode{get\_serializeable\_properties}}{}{}
\end{fulllineitems}

\index{indices (ambrosia\_plugins.lkm.events.SendSignalEvent attribute)}

\begin{fulllineitems}
\phantomsection\label{ambrosia_plugins.lkm:ambrosia_plugins.lkm.events.SendSignalEvent.indices}\pysigline{\bfcode{indices}\strong{ = set({[}{]})}}
\end{fulllineitems}


\end{fulllineitems}

\index{SocketAcceptEvent (class in ambrosia\_plugins.lkm.events)}

\begin{fulllineitems}
\phantomsection\label{ambrosia_plugins.lkm:ambrosia_plugins.lkm.events.SocketAcceptEvent}\pysiglinewithargsret{\strong{class }\code{ambrosia\_plugins.lkm.events.}\bfcode{SocketAcceptEvent}}{\emph{process}, \emph{successful}}{}
Bases: {\hyperref[ambrosia_plugins.lkm:ambrosia_plugins.lkm.events.FileDescriptorEvent]{\code{ambrosia\_plugins.lkm.events.FileDescriptorEvent}}}

Represents a successful accept() on a socket

This event's parent normally is a {\hyperref[ambrosia_plugins.lkm:ambrosia_plugins.lkm.events.SocketEvent]{\code{ambrosia\_plugins.lkm.events.SocketEvent}}} and it is a
{\hyperref[ambrosia_plugins.lkm:ambrosia_plugins.lkm.events.FileDescriptorEvent]{\code{ambrosia\_plugins.lkm.events.FileDescriptorEvent}}} and therefore itself is a file descriptor operation.
\index{get\_serializeable\_properties() (ambrosia\_plugins.lkm.events.SocketAcceptEvent method)}

\begin{fulllineitems}
\phantomsection\label{ambrosia_plugins.lkm:ambrosia_plugins.lkm.events.SocketAcceptEvent.get_serializeable_properties}\pysiglinewithargsret{\bfcode{get\_serializeable\_properties}}{}{}
\end{fulllineitems}

\index{indices (ambrosia\_plugins.lkm.events.SocketAcceptEvent attribute)}

\begin{fulllineitems}
\phantomsection\label{ambrosia_plugins.lkm:ambrosia_plugins.lkm.events.SocketAcceptEvent.indices}\pysigline{\bfcode{indices}\strong{ = set({[}'process'{]})}}
\end{fulllineitems}


\end{fulllineitems}

\index{SocketEvent (class in ambrosia\_plugins.lkm.events)}

\begin{fulllineitems}
\phantomsection\label{ambrosia_plugins.lkm:ambrosia_plugins.lkm.events.SocketEvent}\pysiglinewithargsret{\strong{class }\code{ambrosia\_plugins.lkm.events.}\bfcode{SocketEvent}}{\emph{process}, \emph{successful}}{}
Bases: {\hyperref[ambrosia_plugins.lkm:ambrosia_plugins.lkm.events.FileDescriptorEvent]{\code{ambrosia\_plugins.lkm.events.FileDescriptorEvent}}}

Represents an operation on a socket
\index{address\_families (ambrosia\_plugins.lkm.events.SocketEvent attribute)}

\begin{fulllineitems}
\phantomsection\label{ambrosia_plugins.lkm:ambrosia_plugins.lkm.events.SocketEvent.address_families}\pysigline{\bfcode{address\_families}\strong{ = \{0: `AF\_UNSPEC', 1: `AF\_UNIX', 2: `AF\_INET', 3: `AF\_AX25', 4: `AF\_IPX', 5: `AF\_APPLETALK', 6: `AF\_NETROM', 7: `AF\_BRIDGE', 8: `AF\_ATMPVC', 9: `AF\_X25', 10: `AF\_INET6', 11: `AF\_ROSE', 12: `AF\_DECnet', 13: `AF\_NETBEUI', 14: `AF\_SECURITY', 15: `AF\_KEY', 16: `AF\_NETLINK', 17: `AF\_PACKET', 18: `AF\_ASH', 19: `AF\_ECONET', 20: `AF\_ATMSVC', 21: `AF\_RDS', 22: `AF\_SNA', 23: `AF\_IRDA', 24: `AF\_PPPOX', 25: `AF\_WANPIPE', 26: `AF\_LLC', 27: `AF\_IB', 29: `AF\_CAN', 30: `AF\_TIPC', 31: `AF\_BLUETOOTH', 32: `AF\_IUCV', 33: `AF\_RXRPC', 34: `AF\_ISDN', 35: `AF\_PHONET', 36: `AF\_IEEE802154', 37: `AF\_CAIF', 38: `AF\_ALG', 39: `AF\_NFC', 40: `AF\_VSOCK'\}}}
\end{fulllineitems}

\index{get\_serializeable\_properties() (ambrosia\_plugins.lkm.events.SocketEvent method)}

\begin{fulllineitems}
\phantomsection\label{ambrosia_plugins.lkm:ambrosia_plugins.lkm.events.SocketEvent.get_serializeable_properties}\pysiglinewithargsret{\bfcode{get\_serializeable\_properties}}{}{}
\end{fulllineitems}

\index{indices (ambrosia\_plugins.lkm.events.SocketEvent attribute)}

\begin{fulllineitems}
\phantomsection\label{ambrosia_plugins.lkm:ambrosia_plugins.lkm.events.SocketEvent.indices}\pysigline{\bfcode{indices}\strong{ = set({[}'process'{]})}}
\end{fulllineitems}

\index{sock\_types (ambrosia\_plugins.lkm.events.SocketEvent attribute)}

\begin{fulllineitems}
\phantomsection\label{ambrosia_plugins.lkm:ambrosia_plugins.lkm.events.SocketEvent.sock_types}\pysigline{\bfcode{sock\_types}\strong{ = \{1: `SOCK\_STREAM', 2: `SOCK\_DGRAM', 3: `SOCK\_RAW', 4: `SOCK\_RDM', 5: `SOCK\_SEQPACKET', 6: `SOCK\_DCCP', 10: `SOCK\_PACKET'\}}}
\end{fulllineitems}


\end{fulllineitems}

\index{StartTaskEvent (class in ambrosia\_plugins.lkm.events)}

\begin{fulllineitems}
\phantomsection\label{ambrosia_plugins.lkm:ambrosia_plugins.lkm.events.StartTaskEvent}\pysiglinewithargsret{\strong{class }\code{ambrosia\_plugins.lkm.events.}\bfcode{StartTaskEvent}}{\emph{start\_ts}, \emph{end\_ts}, \emph{process}, \emph{child\_pid}, \emph{spawned\_child}}{}
Bases: {\hyperref[ambrosia.model:ambrosia.model.Event]{\code{ambrosia.model.Event}}}

Represents a fork()-like syscall
\index{get\_serializeable\_properties() (ambrosia\_plugins.lkm.events.StartTaskEvent method)}

\begin{fulllineitems}
\phantomsection\label{ambrosia_plugins.lkm:ambrosia_plugins.lkm.events.StartTaskEvent.get_serializeable_properties}\pysiglinewithargsret{\bfcode{get\_serializeable\_properties}}{}{}
\end{fulllineitems}

\index{indices (ambrosia\_plugins.lkm.events.StartTaskEvent attribute)}

\begin{fulllineitems}
\phantomsection\label{ambrosia_plugins.lkm:ambrosia_plugins.lkm.events.StartTaskEvent.indices}\pysigline{\bfcode{indices}\strong{ = set({[}{]})}}
\end{fulllineitems}


\end{fulllineitems}

\index{SuperUserRequestEvent (class in ambrosia\_plugins.lkm.events)}

\begin{fulllineitems}
\phantomsection\label{ambrosia_plugins.lkm:ambrosia_plugins.lkm.events.SuperUserRequestEvent}\pysiglinewithargsret{\strong{class }\code{ambrosia\_plugins.lkm.events.}\bfcode{SuperUserRequestEvent}}{\emph{start\_ts}, \emph{end\_ts}, \emph{process}}{}
Bases: {\hyperref[ambrosia.model:ambrosia.model.Event]{\code{ambrosia.model.Event}}}

Indicates that the process tried to run ``su''
\index{get\_serializeable\_properties() (ambrosia\_plugins.lkm.events.SuperUserRequestEvent method)}

\begin{fulllineitems}
\phantomsection\label{ambrosia_plugins.lkm:ambrosia_plugins.lkm.events.SuperUserRequestEvent.get_serializeable_properties}\pysiglinewithargsret{\bfcode{get\_serializeable\_properties}}{}{}
\end{fulllineitems}

\index{indices (ambrosia\_plugins.lkm.events.SuperUserRequestEvent attribute)}

\begin{fulllineitems}
\phantomsection\label{ambrosia_plugins.lkm:ambrosia_plugins.lkm.events.SuperUserRequestEvent.indices}\pysigline{\bfcode{indices}\strong{ = set({[}{]})}}
\end{fulllineitems}


\end{fulllineitems}

\index{SyscallEvent (class in ambrosia\_plugins.lkm.events)}

\begin{fulllineitems}
\phantomsection\label{ambrosia_plugins.lkm:ambrosia_plugins.lkm.events.SyscallEvent}\pysiglinewithargsret{\strong{class }\code{ambrosia\_plugins.lkm.events.}\bfcode{SyscallEvent}}{\emph{context}, \emph{props}, \emph{time}, \emph{monotonic\_ts}, \emph{process}, \emph{idx}, \emph{spawned\_child=None}}{}
Bases: {\hyperref[ambrosia.model:ambrosia.model.Event]{\code{ambrosia.model.Event}}}

Represents a system call from lkm
\index{get\_serializeable\_properties() (ambrosia\_plugins.lkm.events.SyscallEvent method)}

\begin{fulllineitems}
\phantomsection\label{ambrosia_plugins.lkm:ambrosia_plugins.lkm.events.SyscallEvent.get_serializeable_properties}\pysiglinewithargsret{\bfcode{get\_serializeable\_properties}}{}{}
\end{fulllineitems}

\index{indices (ambrosia\_plugins.lkm.events.SyscallEvent attribute)}

\begin{fulllineitems}
\phantomsection\label{ambrosia_plugins.lkm:ambrosia_plugins.lkm.events.SyscallEvent.indices}\pysigline{\bfcode{indices}\strong{ = set({[}'index', `name'{]})}}
\end{fulllineitems}


\end{fulllineitems}

\index{UnknownFdEvent (class in ambrosia\_plugins.lkm.events)}

\begin{fulllineitems}
\phantomsection\label{ambrosia_plugins.lkm:ambrosia_plugins.lkm.events.UnknownFdEvent}\pysiglinewithargsret{\strong{class }\code{ambrosia\_plugins.lkm.events.}\bfcode{UnknownFdEvent}}{\emph{process}, \emph{fd\_number}, \emph{successful}}{}
Bases: {\hyperref[ambrosia_plugins.lkm:ambrosia_plugins.lkm.events.FileDescriptorEvent]{\code{ambrosia\_plugins.lkm.events.FileDescriptorEvent}}}

Represents a fd event where no syscall opening the fd has been found.
\index{get\_serializeable\_properties() (ambrosia\_plugins.lkm.events.UnknownFdEvent method)}

\begin{fulllineitems}
\phantomsection\label{ambrosia_plugins.lkm:ambrosia_plugins.lkm.events.UnknownFdEvent.get_serializeable_properties}\pysiglinewithargsret{\bfcode{get\_serializeable\_properties}}{}{}
\end{fulllineitems}


\end{fulllineitems}

\index{ZygoteForkEvent (class in ambrosia\_plugins.lkm.events)}

\begin{fulllineitems}
\phantomsection\label{ambrosia_plugins.lkm:ambrosia_plugins.lkm.events.ZygoteForkEvent}\pysiglinewithargsret{\strong{class }\code{ambrosia\_plugins.lkm.events.}\bfcode{ZygoteForkEvent}}{\emph{process}}{}
Bases: {\hyperref[ambrosia.model:ambrosia.model.Event]{\code{ambrosia.model.Event}}}
\index{get\_serializeable\_properties() (ambrosia\_plugins.lkm.events.ZygoteForkEvent method)}

\begin{fulllineitems}
\phantomsection\label{ambrosia_plugins.lkm:ambrosia_plugins.lkm.events.ZygoteForkEvent.get_serializeable_properties}\pysiglinewithargsret{\bfcode{get\_serializeable\_properties}}{}{}
\end{fulllineitems}

\index{indices (ambrosia\_plugins.lkm.events.ZygoteForkEvent attribute)}

\begin{fulllineitems}
\phantomsection\label{ambrosia_plugins.lkm:ambrosia_plugins.lkm.events.ZygoteForkEvent.indices}\pysigline{\bfcode{indices}\strong{ = set({[}'process'{]})}}
\end{fulllineitems}


\end{fulllineitems}



\subparagraph{Module contents}
\label{ambrosia_plugins.lkm:module-contents}\label{ambrosia_plugins.lkm:module-ambrosia_plugins.lkm}\index{ambrosia\_plugins.lkm (module)}\index{AdbCommandCorrelator (class in ambrosia\_plugins.lkm)}

\begin{fulllineitems}
\phantomsection\label{ambrosia_plugins.lkm:ambrosia_plugins.lkm.AdbCommandCorrelator}\pysiglinewithargsret{\strong{class }\code{ambrosia\_plugins.lkm.}\bfcode{AdbCommandCorrelator}}{\emph{context}}{}
Bases: {\hyperref[ambrosia:ambrosia.Correlator]{\code{ambrosia.Correlator}}}

Find command executions that happen because of ANANAS (through ADB)
\index{correlate() (ambrosia\_plugins.lkm.AdbCommandCorrelator method)}

\begin{fulllineitems}
\phantomsection\label{ambrosia_plugins.lkm:ambrosia_plugins.lkm.AdbCommandCorrelator.correlate}\pysiglinewithargsret{\bfcode{correlate}}{}{}
\end{fulllineitems}


\end{fulllineitems}

\index{CommandExecuteCorrelator (class in ambrosia\_plugins.lkm)}

\begin{fulllineitems}
\phantomsection\label{ambrosia_plugins.lkm:ambrosia_plugins.lkm.CommandExecuteCorrelator}\pysiglinewithargsret{\strong{class }\code{ambrosia\_plugins.lkm.}\bfcode{CommandExecuteCorrelator}}{\emph{context}}{}
Bases: {\hyperref[ambrosia:ambrosia.Correlator]{\code{ambrosia.Correlator}}}

Finds events that form the execution of a command.
\begin{itemize}
\item {} 
{\hyperref[ambrosia_plugins.lkm:ambrosia_plugins.lkm.events.StartTaskEvent]{\code{ambrosia\_plugins.lkm.events.StartTaskEvent}}}: indicate the creation of a new process

\item {} 
{\hyperref[ambrosia_plugins.lkm:ambrosia_plugins.lkm.events.ExecEvent]{\code{ambrosia\_plugins.lkm.events.ExecEvent}}}: commands are started using a fork-and-exec

\item {} \begin{description}
\item[{\code{ambrosia\_plugins.lkm.events.LibraryLoad}: shortly after a fork indicates that a library is loaded that}] \leavevmode
is essential to run the command.

\end{description}

\item {} 
{\hyperref[ambrosia_plugins.lkm:ambrosia_plugins.lkm.events.FileEvent]{\code{ambrosia\_plugins.lkm.events.FileEvent}}}: several file events happen at the begin of a command execution

\end{itemize}
\index{\_find\_file\_events() (ambrosia\_plugins.lkm.CommandExecuteCorrelator method)}

\begin{fulllineitems}
\phantomsection\label{ambrosia_plugins.lkm:ambrosia_plugins.lkm.CommandExecuteCorrelator._find_file_events}\pysiglinewithargsret{\bfcode{\_find\_file\_events}}{\emph{process}, \emph{evt}, \emph{start\_ts}, \emph{matches}}{}
\end{fulllineitems}

\index{\_find\_java\_library\_loads() (ambrosia\_plugins.lkm.CommandExecuteCorrelator method)}

\begin{fulllineitems}
\phantomsection\label{ambrosia_plugins.lkm:ambrosia_plugins.lkm.CommandExecuteCorrelator._find_java_library_loads}\pysiglinewithargsret{\bfcode{\_find\_java\_library\_loads}}{\emph{process}, \emph{evt}, \emph{start\_ts}}{}
\end{fulllineitems}

\index{\_find\_library\_loads() (ambrosia\_plugins.lkm.CommandExecuteCorrelator method)}

\begin{fulllineitems}
\phantomsection\label{ambrosia_plugins.lkm:ambrosia_plugins.lkm.CommandExecuteCorrelator._find_library_loads}\pysiglinewithargsret{\bfcode{\_find\_library\_loads}}{\emph{process}, \emph{evt}, \emph{start\_ts}}{}
\end{fulllineitems}

\index{\_find\_mkdir\_events() (ambrosia\_plugins.lkm.CommandExecuteCorrelator method)}

\begin{fulllineitems}
\phantomsection\label{ambrosia_plugins.lkm:ambrosia_plugins.lkm.CommandExecuteCorrelator._find_mkdir_events}\pysiglinewithargsret{\bfcode{\_find\_mkdir\_events}}{\emph{process}, \emph{evt}, \emph{start\_ts}}{}
\end{fulllineitems}

\index{correlate() (ambrosia\_plugins.lkm.CommandExecuteCorrelator method)}

\begin{fulllineitems}
\phantomsection\label{ambrosia_plugins.lkm:ambrosia_plugins.lkm.CommandExecuteCorrelator.correlate}\pysiglinewithargsret{\bfcode{correlate}}{}{}
\end{fulllineitems}


\end{fulllineitems}

\index{FileEventCorrelator (class in ambrosia\_plugins.lkm)}

\begin{fulllineitems}
\phantomsection\label{ambrosia_plugins.lkm:ambrosia_plugins.lkm.FileEventCorrelator}\pysiglinewithargsret{\strong{class }\code{ambrosia\_plugins.lkm.}\bfcode{FileEventCorrelator}}{\emph{context}}{}
Bases: {\hyperref[ambrosia:ambrosia.Correlator]{\code{ambrosia.Correlator}}}

Finds library load events (mmap to *.so files)
\index{correlate() (ambrosia\_plugins.lkm.FileEventCorrelator method)}

\begin{fulllineitems}
\phantomsection\label{ambrosia_plugins.lkm:ambrosia_plugins.lkm.FileEventCorrelator.correlate}\pysiglinewithargsret{\bfcode{correlate}}{}{}
\end{fulllineitems}


\end{fulllineitems}

\index{InstallCorelator (class in ambrosia\_plugins.lkm)}

\begin{fulllineitems}
\phantomsection\label{ambrosia_plugins.lkm:ambrosia_plugins.lkm.InstallCorelator}\pysiglinewithargsret{\strong{class }\code{ambrosia\_plugins.lkm.}\bfcode{InstallCorelator}}{\emph{context}}{}
Bases: {\hyperref[ambrosia:ambrosia.Correlator]{\code{ambrosia.Correlator}}}
\index{correlate() (ambrosia\_plugins.lkm.InstallCorelator method)}

\begin{fulllineitems}
\phantomsection\label{ambrosia_plugins.lkm:ambrosia_plugins.lkm.InstallCorelator.correlate}\pysiglinewithargsret{\bfcode{correlate}}{}{}
\end{fulllineitems}


\end{fulllineitems}

\index{LkmPluginParser (class in ambrosia\_plugins.lkm)}

\begin{fulllineitems}
\phantomsection\label{ambrosia_plugins.lkm:ambrosia_plugins.lkm.LkmPluginParser}\pysigline{\strong{class }\code{ambrosia\_plugins.lkm.}\bfcode{LkmPluginParser}}
Bases: {\hyperref[ambrosia:ambrosia.ResultParser]{\code{ambrosia.ResultParser}}}

Parses the \emph{process} and \emph{syscalltrace} elements of the result set.
\index{finish() (ambrosia\_plugins.lkm.LkmPluginParser method)}

\begin{fulllineitems}
\phantomsection\label{ambrosia_plugins.lkm:ambrosia_plugins.lkm.LkmPluginParser.finish}\pysiglinewithargsret{\bfcode{finish}}{\emph{context}}{}
Calculate additional information for each process.

This method is executed after all processes have been parsed. This allows to reliably reference other processes
(E.g. when the first process is being parsed no other proccess is known, therefore no other process can be
referenced). The method sets the tg\_leader and the parent. Moreover, it copies the reference to \emph{fds} from the
parent for all threads (in linux a thread \emph{normally} shares FDs with its thread group leader).

\end{fulllineitems}

\index{parse() (ambrosia\_plugins.lkm.LkmPluginParser method)}

\begin{fulllineitems}
\phantomsection\label{ambrosia_plugins.lkm:ambrosia_plugins.lkm.LkmPluginParser.parse}\pysiglinewithargsret{\bfcode{parse}}{\emph{name}, \emph{el}, \emph{context}}{}
Does the actual parsing.
\begin{itemize}
\item {} 
\emph{process} element: All processes reported by the LKM/ANANAS are parsed and
\code{ambrosia\_web.model.entities.Task} entities are created. Moreover, the attributes
* \emph{ananas\_id} (id in the ANANAS db)
* \emph{parent\_id} (the ANANAS db id of the parent task)
* \emph{comm} (description of the process in thekernel)
* \emph{path} (of the executable)
* \emph{type} (the type of the task ANANAS figured out)
* \emph{fds} (a dict of all file descriptors and the path during LKM load)
* \emph{tdgid} (the PID of the task group leader)
* \emph{tg\_leader\_id} (The ANANAS db id of the thread group leader)

\item {} 
\emph{syscalltrace} element: A {\hyperref[ambrosia_plugins.lkm:ambrosia_plugins.lkm.events.SyscallEvent]{\code{ambrosia\_plugins.lkm.events.SyscallEvent}}} event is create for each syscall
using all the information ANANAS provides. Moreover the \code{ambrosia\_web.clocks.ClockSyncer}.translate\_table
attribute is filled. ANANAS records two timestamps for each syscall. There is a \emph{normal} timestamp (which is
the system time when the syscall returned) and the \emph{monotonic} timestamp (which is the time that passed since
the system booted). When the system clock is not changed, the \emph{monotonic} and the \emph{normal} clock are in sync
(e.g. if 10 seconds pass on one clock 10 seconds pass on the other clock). Therefore the \emph{normal} clock is
ahead of the \emph{monotonic} clock (a constant offset = the time the emulator booted). By calculating the
\emph{normal} clock minus the \emph{monotonic} clock we always get this offset. When this offset changes, the system
clock has been altered.
\begin{description}
\item[{This algorithm is implemented using the following variables:}] \leavevmode\begin{itemize}
\item {} 
boot\_time: the actual time the emulator is booted (calculated \emph{normal} - \emph{monotonic} time on the first
syscall = when emulator time and host time are still in sync)

\item {} 
error: how much the expected offset (boot\_time) is off from the acutal offset (\emph{normal} - \emph{monotonic}).
This is also the error of the emulator clock (compared to the host clock)

\item {} 
adjtime: the adjusted time (the captured \emph{normal} time - error).

\item {} 
lasterror: the error of the last syscall. If the error of two consecutive syscall changes, we know that
the system clock has been altered (and we need to make an entry in
\code{ambrosia\_web.clocks.ClockSyncer}.translate\_table). The comparison sees two errors that are at a maximum
of 1 second apart as a clock change. This is because the error is not absolutely precise (the \emph{monotonic}
and \emph{normal} timestamps are not captured at exactly the same time, even a context switch may happen in
between).

\end{itemize}

\end{description}

\end{itemize}

\end{fulllineitems}


\end{fulllineitems}

\index{PluginInfo (class in ambrosia\_plugins.lkm)}

\begin{fulllineitems}
\phantomsection\label{ambrosia_plugins.lkm:ambrosia_plugins.lkm.PluginInfo}\pysigline{\strong{class }\code{ambrosia\_plugins.lkm.}\bfcode{PluginInfo}}
Bases: {\hyperref[ambrosia.plugins:ambrosia.plugins.PluginInfoTop]{\code{ambrosia.plugins.PluginInfoTop}}}
\index{correlators() (ambrosia\_plugins.lkm.PluginInfo static method)}

\begin{fulllineitems}
\phantomsection\label{ambrosia_plugins.lkm:ambrosia_plugins.lkm.PluginInfo.correlators}\pysiglinewithargsret{\strong{static }\bfcode{correlators}}{}{}
\end{fulllineitems}

\index{parsers() (ambrosia\_plugins.lkm.PluginInfo static method)}

\begin{fulllineitems}
\phantomsection\label{ambrosia_plugins.lkm:ambrosia_plugins.lkm.PluginInfo.parsers}\pysiglinewithargsret{\strong{static }\bfcode{parsers}}{}{}
\end{fulllineitems}


\end{fulllineitems}

\index{SyscallCorrelator (class in ambrosia\_plugins.lkm)}

\begin{fulllineitems}
\phantomsection\label{ambrosia_plugins.lkm:ambrosia_plugins.lkm.SyscallCorrelator}\pysiglinewithargsret{\strong{class }\code{ambrosia\_plugins.lkm.}\bfcode{SyscallCorrelator}}{\emph{context}}{}
Bases: {\hyperref[ambrosia:ambrosia.Correlator]{\code{ambrosia.Correlator}}}

Wraps primitive events into higher-level events
\index{\_check\_syscall() (ambrosia\_plugins.lkm.SyscallCorrelator method)}

\begin{fulllineitems}
\phantomsection\label{ambrosia_plugins.lkm:ambrosia_plugins.lkm.SyscallCorrelator._check_syscall}\pysiglinewithargsret{\bfcode{\_check\_syscall}}{\emph{evt}}{}
Wraps a single syscall event into a higher-level event
\begin{quote}\begin{description}
\item[{Parameters}] \leavevmode
\textbf{evt} ({\hyperref[ambrosia_plugins.lkm:ambrosia_plugins.lkm.events.SyscallEvent]{\emph{ambrosia\_plugins.lkm.events.SyscallEvent}}}) -- the syscall event

\end{description}\end{quote}

\end{fulllineitems}

\index{\_generate\_start\_fd\_directory() (ambrosia\_plugins.lkm.SyscallCorrelator method)}

\begin{fulllineitems}
\phantomsection\label{ambrosia_plugins.lkm:ambrosia_plugins.lkm.SyscallCorrelator._generate_start_fd_directory}\pysiglinewithargsret{\bfcode{\_generate\_start\_fd\_directory}}{}{}
Generates the initial fd directory.

Before the correlation is started the fd directory is filed with file descriptor events of processes that
existed before the LKM was loaded.

\end{fulllineitems}

\index{\_get\_del\_fd\_event() (ambrosia\_plugins.lkm.SyscallCorrelator method)}

\begin{fulllineitems}
\phantomsection\label{ambrosia_plugins.lkm:ambrosia_plugins.lkm.SyscallCorrelator._get_del_fd_event}\pysiglinewithargsret{\bfcode{\_get\_del\_fd\_event}}{\emph{fd}, \emph{process}, \emph{success}, \emph{logname}, \emph{clazz=None}}{}
Gets an fd event from the fd directory and deletes it.
\begin{quote}\begin{description}
\item[{Parameters}] \leavevmode\begin{itemize}
\item {} 
\textbf{fd} (\emph{int}) -- the file descriptor number we are searching for

\item {} 
\textbf{process} (\emph{ambrosia\_web.model.entities.Task}) -- the task the fd belongs to

\item {} 
\textbf{clazz} (\emph{class}) -- (optional) only return an event of this type

\item {} 
\textbf{process} -- the task the fd belongs to

\end{itemize}

\end{description}\end{quote}

\end{fulllineitems}

\index{\_get\_dup() (ambrosia\_plugins.lkm.SyscallCorrelator method)}

\begin{fulllineitems}
\phantomsection\label{ambrosia_plugins.lkm:ambrosia_plugins.lkm.SyscallCorrelator._get_dup}\pysiglinewithargsret{\bfcode{\_get\_dup}}{\emph{evt}, \emph{oldfd}, \emph{newfd}, \emph{process}}{}
Duplicate an fd (dup and dup2 syscalls)
\begin{quote}\begin{description}
\item[{Parameters}] \leavevmode\begin{itemize}
\item {} 
\textbf{evt} (\emph{ambrosia\_web.model.Event}) -- the dup syscall event

\item {} 
\textbf{oldfd} (\emph{int}) -- the old file descriptor number

\item {} 
\textbf{newfd} (\emph{int}) -- the new file descriptor number

\end{itemize}

\end{description}\end{quote}

\end{fulllineitems}

\index{\_get\_fd\_event() (ambrosia\_plugins.lkm.SyscallCorrelator method)}

\begin{fulllineitems}
\phantomsection\label{ambrosia_plugins.lkm:ambrosia_plugins.lkm.SyscallCorrelator._get_fd_event}\pysiglinewithargsret{\bfcode{\_get\_fd\_event}}{\emph{fd}, \emph{process}, \emph{success}, \emph{logname}, \emph{clazz=None}, \emph{default\_start\_ts=None}}{}
Get an fd event from the a fd directory entry.

The fd directory (\emph{fd\_directory}) is a dict in the form of

\begin{Verbatim}[commandchars=\\\{\}]
\PYG{p}{\PYGZob{}}
    \PYG{n}{pid}\PYG{p}{:} \PYG{p}{\PYGZob{}}
        \PYG{n}{fd\PYGZus{}number}\PYG{p}{:} \PYG{n}{fd\PYGZus{}event}\PYG{p}{,}
        \PYG{o}{.}\PYG{o}{.}\PYG{o}{.}
    \PYG{p}{\PYGZcb{}}\PYG{p}{,}
    \PYG{o}{.}\PYG{o}{.}\PYG{o}{.}
\PYG{p}{\PYGZcb{}}
\end{Verbatim}

The fd directory represents all file descriptors of the emulator \textbf{at a specific point in time}. This means
that the fd directory is constantly changed as syscalls are being processed (e.g. open() creates an entry, close
removes an entry).

If (for some reason) the fd is not found, this method returns an
{\hyperref[ambrosia_plugins.lkm:ambrosia_plugins.lkm.events.UnknownFdEvent]{\code{ambrosia\_plugins.lkm.events.UnknownFdEvent}}}.

\begin{notice}{note}{Note:}
One value of the fd dictionary dict may be stored under multiple pid keys since tasks (especially threads)
may share file descriptors.
\end{notice}
\begin{quote}\begin{description}
\item[{Parameters}] \leavevmode\begin{itemize}
\item {} 
\textbf{fd} (\emph{int}) -- the file descriptor number we are searching for

\item {} 
\textbf{process} (\emph{ambrosia\_web.model.entities.Task}) -- the task the fd belongs to

\item {} 
\textbf{clazz} (\emph{class}) -- (optional) only return an event of this type

\item {} 
\textbf{default\_start\_ts} (\emph{datetime.datetime}) -- if this fd is unknown, return an event with this start timestamp

\end{itemize}

\end{description}\end{quote}

\end{fulllineitems}

\index{\_parse\_addr\_str() (ambrosia\_plugins.lkm.SyscallCorrelator method)}

\begin{fulllineitems}
\phantomsection\label{ambrosia_plugins.lkm:ambrosia_plugins.lkm.SyscallCorrelator._parse_addr_str}\pysiglinewithargsret{\bfcode{\_parse\_addr\_str}}{\emph{addrstr}, \emph{socket\_evt}}{}
\end{fulllineitems}

\index{correlate() (ambrosia\_plugins.lkm.SyscallCorrelator method)}

\begin{fulllineitems}
\phantomsection\label{ambrosia_plugins.lkm:ambrosia_plugins.lkm.SyscallCorrelator.correlate}\pysiglinewithargsret{\bfcode{correlate}}{}{}
\end{fulllineitems}


\end{fulllineitems}

\index{\_timedelta\_diff() (in module ambrosia\_plugins.lkm)}

\begin{fulllineitems}
\phantomsection\label{ambrosia_plugins.lkm:ambrosia_plugins.lkm._timedelta_diff}\pysiglinewithargsret{\code{ambrosia\_plugins.lkm.}\bfcode{\_timedelta\_diff}}{\emph{td1}, \emph{td2}}{}
\end{fulllineitems}



\paragraph{ambrosia\_plugins.network package}
\label{ambrosia_plugins.network:ambrosia-plugins-network-package}\label{ambrosia_plugins.network::doc}

\subparagraph{Module contents}
\label{ambrosia_plugins.network:module-ambrosia_plugins.network}\label{ambrosia_plugins.network:module-contents}\index{ambrosia\_plugins.network (module)}\index{PluginInfo (class in ambrosia\_plugins.network)}

\begin{fulllineitems}
\phantomsection\label{ambrosia_plugins.network:ambrosia_plugins.network.PluginInfo}\pysigline{\strong{class }\code{ambrosia\_plugins.network.}\bfcode{PluginInfo}}
Bases: {\hyperref[ambrosia.plugins:ambrosia.plugins.PluginInfoTop]{\code{ambrosia.plugins.PluginInfoTop}}}

This plugin is not implemented. Implement as soon as ANANAS properly supports network traffic analysis.

\end{fulllineitems}



\subsubsection{Module contents}
\label{ambrosia_plugins:module-ambrosia_plugins}\label{ambrosia_plugins:module-contents}\index{ambrosia\_plugins (module)}

\subsection{processor module}
\label{processor:module-processor}\label{processor::doc}\label{processor:processor-module}\index{processor (module)}\index{main() (in module processor)}

\begin{fulllineitems}
\phantomsection\label{processor:processor.main}\pysiglinewithargsret{\code{processor.}\bfcode{main}}{}{}
The main method

\end{fulllineitems}



\subsection{Overview}
\label{server:overview}
This section gives a short overview of the internal workings of Ambrosia. For a detailed description please see the
documentation for the modules.

The main function for the Ambrosia server side part is located in {\hyperref[processor:module-processor]{\code{processor}}}. The following shows the usage
of the processor:

\begin{Verbatim}[commandchars=\\\{\}]
usage: processor.py [\PYGZhy{}h] [\PYGZhy{}\PYGZhy{}config CONFIG]
                    [\PYGZhy{}\PYGZhy{}loglevel \PYGZob{}FATAL,ERROR,WARN,INFO,DEBUG\PYGZcb{}]
                    [\PYGZhy{}\PYGZhy{}output OUTPUT]
                    [\PYGZhy{}\PYGZhy{}output\PYGZhy{}type \PYGZob{}serialized,none,tree,interactive\PYGZcb{}]
                    report

process ANANAS report for Ambrosia

positional arguments:
  report                the XML report input

optional arguments:
  \PYGZhy{}h, \PYGZhy{}\PYGZhy{}help            show this help message and exit
  \PYGZhy{}\PYGZhy{}config CONFIG       the config file
  \PYGZhy{}\PYGZhy{}loglevel \PYGZob{}FATAL,ERROR,WARN,INFO,DEBUG\PYGZcb{}
                        the log level for stderr
  \PYGZhy{}\PYGZhy{}output OUTPUT       the output file, default is stdout
  \PYGZhy{}\PYGZhy{}output\PYGZhy{}type \PYGZob{}serialized,none,tree,interactive\PYGZcb{}
                        define what should be printed
\end{Verbatim}

The processor initializes logging (see {\hyperref[ambrosia.util:ambrosia.util.log.init_logging]{\code{ambrosia.util.log.init\_logging()}}}), reads the XML report and creates an
{\hyperref[ambrosia:ambrosia.Ambrosia]{\code{ambrosia.Ambrosia}}} instance. It adjusts the timestamps of events coming from the emulator (see
{\hyperref[ambrosia:ambrosia.Ambrosia.adjust_times]{\code{ambrosia.Ambrosia.adjust\_times()}}}), correlates the events ({\hyperref[ambrosia:ambrosia.Ambrosia.correlate]{\code{ambrosia.Ambrosia.correlate()}}}) and serializes
them ({\hyperref[ambrosia:ambrosia.Ambrosia.serialize]{\code{ambrosia.Ambrosia.serialize()}}}).

All the results are stored in an {\hyperref[ambrosia.model:ambrosia.model.Analysis]{\code{ambrosia.model.Analysis}}} instance. The main two types of entries in the result
are events ({\hyperref[ambrosia.model:ambrosia.model.Event]{\code{ambrosia.model.Event}}}) and entities ({\hyperref[ambrosia.model:ambrosia.model.Entity]{\code{ambrosia.model.Entity}}}). All events and entities
are managed by the {\hyperref[ambrosia.model:ambrosia.model.Analysis]{\code{ambrosia.model.Analysis}}} class. All entities are defined in {\hyperref[ambrosia.model:module-ambrosia.model.entities]{\code{ambrosia.model.entities}}}.
The events are defined by each plugin.


\subsubsection{Plugins}
\label{server:plugins}
All plugins are defined in the {\hyperref[ambrosia_plugins:module-ambrosia_plugins]{\code{ambrosia\_plugins}}} module. A plugin has to specify a
{\hyperref[ambrosia.plugins:ambrosia.plugins.PluginInfoTop]{\code{ambrosia.plugins.PluginInfoTop}}} class called ``PluginInfo''. This class should return all
{\hyperref[ambrosia:ambrosia.Correlator]{\code{ambrosia.Correlator}}} and {\hyperref[ambrosia:ambrosia.ResultParser]{\code{ambrosia.ResultParser}}} classes defined by the plugin.

A {\hyperref[ambrosia:ambrosia.ResultParser]{\code{ambrosia.ResultParser}}} is used to extract events from the report. A {\hyperref[ambrosia:ambrosia.Correlator]{\code{ambrosia.Correlator}}} can be used
to correlate and consolidate events.


\paragraph{\texttt{ambrosia\_plugins.apimonitor}}
\label{server:ambrosia-plugins-apimonitor}
The {\hyperref[ambrosia_plugins.apimonitor:ambrosia_plugins.apimonitor.ApimonitorPluginParser]{\code{ambrosia\_plugins.apimonitor.ApimonitorPluginParser}}} generates events from the report. The
{\hyperref[ambrosia_plugins.apimonitor:ambrosia_plugins.apimonitor.ApiCallCorrelator]{\code{ambrosia\_plugins.apimonitor.ApiCallCorrelator}}} is used to then find known API calls and wrap them into
higher-level events.


\paragraph{\texttt{ambrosia\_plugins.events}}
\label{server:ambrosia-plugins-events}
This simple plugin parses all events created by ANANAS itself.


\paragraph{\texttt{ambrosia\_plugins.lkm}}
\label{server:ambrosia-plugins-lkm}
The lkm plugin handles events originating from the lkm. After the ambrosia\_plugins.lkm.LkmPluginParser has parsed all
the lkm related information from the report, the {\hyperref[ambrosia_plugins.lkm:ambrosia_plugins.lkm.SyscallCorrelator]{\code{ambrosia\_plugins.lkm.SyscallCorrelator}}} wraps primitive syscall
events into higher-level event (like \code{ambrosia\_plugins.lkm.events.SendSignal} or
\code{ambrosia\_plugins.lkm.events.CreateDir}). The {\hyperref[ambrosia_plugins.lkm:ambrosia_plugins.lkm.FileEventCorrelator]{\code{ambrosia\_plugins.lkm.FileEventCorrelator}}} is used to
classify file events (to \code{ambrosia\_plugins.lkm.events.LibraryLoad}). The
{\hyperref[ambrosia_plugins.lkm:ambrosia_plugins.lkm.CommandExecuteCorrelator]{\code{ambrosia\_plugins.lkm.CommandExecuteCorrelator}}} then finds command executions. The
{\hyperref[ambrosia_plugins.lkm:ambrosia_plugins.lkm.AdbCommandCorrelator]{\code{ambrosia\_plugins.lkm.AdbCommandCorrelator}}} is then used to find command executions that have been caused by
ANANAS itself.


\paragraph{\texttt{ambrosia\_plugins.network}}
\label{server:ambrosia-plugins-network}
This plugin is currently not implemented.


\renewcommand{\indexname}{Python Module Index}
\begin{theindex}
\def\bigletter#1{{\Large\sffamily#1}\nopagebreak\vspace{1mm}}
\bigletter{a}
\item {\texttt{ambrosia}}, \pageref{ambrosia:module-ambrosia}
\item {\texttt{ambrosia.clocks}}, \pageref{ambrosia.clocks:module-ambrosia.clocks}
\item {\texttt{ambrosia.config}}, \pageref{ambrosia.config:module-ambrosia.config}
\item {\texttt{ambrosia.context}}, \pageref{ambrosia.context:module-ambrosia.context}
\item {\texttt{ambrosia.db}}, \pageref{ambrosia.db:module-ambrosia.db}
\item {\texttt{ambrosia.model}}, \pageref{ambrosia.model:module-ambrosia.model}
\item {\texttt{ambrosia.model.entities}}, \pageref{ambrosia.model:module-ambrosia.model.entities}
\item {\texttt{ambrosia.plugins}}, \pageref{ambrosia.plugins:module-ambrosia.plugins}
\item {\texttt{ambrosia.util}}, \pageref{ambrosia.util:module-ambrosia.util}
\item {\texttt{ambrosia.util.log}}, \pageref{ambrosia.util:module-ambrosia.util.log}
\item {\texttt{ambrosia\_plugins}}, \pageref{ambrosia_plugins:module-ambrosia_plugins}
\item {\texttt{ambrosia\_plugins.apimonitor}}, \pageref{ambrosia_plugins.apimonitor:module-ambrosia_plugins.apimonitor}
\item {\texttt{ambrosia\_plugins.events}}, \pageref{ambrosia_plugins.events:module-ambrosia_plugins.events}
\item {\texttt{ambrosia\_plugins.lkm}}, \pageref{ambrosia_plugins.lkm:module-ambrosia_plugins.lkm}
\item {\texttt{ambrosia\_plugins.lkm.events}}, \pageref{ambrosia_plugins.lkm:module-ambrosia_plugins.lkm.events}
\item {\texttt{ambrosia\_plugins.network}}, \pageref{ambrosia_plugins.network:module-ambrosia_plugins.network}
\indexspace
\bigletter{p}
\item {\texttt{processor}}, \pageref{processor:module-processor}
\end{theindex}

\renewcommand{\indexname}{Index}
\printindex
\end{document}
